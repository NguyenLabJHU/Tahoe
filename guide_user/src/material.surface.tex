%&LaTeX
% $Id: material.surface.tex,v 1.3 2003-05-21 23:49:03 cjkimme Exp $

\section{Constitutive models for cohesive surface elements}

This section of the guide describes the constitutive models that may
be used as material models specifying the traction vector as a function of
the displacement discontinuity for the cohesive surface elements described in
Section~\ref{section.CSE.laws}. Each of the following subsections provides 
some background information about its respective material model followed by 
a description and example of the input required for analysis. All cohesive
surface models have been implemented in two
dimensions, and most also have three-dimensional implementations. 
All surface materials support implicit and explicit 
analysis; in other words, all models implement consistent tangent matrices 
that produce quadratic convergence near local minima when used with the 
Newton-type, nonlinear solution algorithms (see Section~\ref{sect.solver.nonlinear}).

\begin{table}[h]
\caption{\label{tab.mat.surface} Constitutive models for cohesive surface
elements}
\begin{center}
	
\begin{longtable}[c]{|c|c|}
\hline
 \parbox[c]{0.75in}{\centering \textbf{code}}
&\parbox[c]{4.5in}{\raggedright \textbf{description}}\\
\endhead
\endfoot
\hline
0 & \parbox[c]{4.5in}{\raggedright Xu-Needleman (elastic)}\\
\hline
1 & \parbox[c]{4.5in}{\raggedright Tvergaard-Hutchinson (elastic)}\\
\hline
2 & \parbox[c]{4.5in}{\raggedright Linear Damage}\\
\hline
3 & \parbox[c]{4.5in}{\raggedright Tvergaard-Hutchinson (elastic + viscous dissipation)}\\
\hline
4 & \parbox[c]{4.5in}{\raggedright Tijssens (elastic/viscoplastic)}\\
\hline
5 & \parbox[c]{4.5in}{\raggedright trilinear rate-dependent}\\
\hline
6 & \parbox[c]{4.5in}{\raggedright ``Tied'' potential (elastic)}\\
\hline
7 & \parbox[c]{4.5in}{\raggedright Yoon, Allen, and Searcy (linear viscoelastic)}\\
\hline
8 & \parbox[c]{4.5in}{\raggedright thermomechanical model (not implemented)}\\
\hline
9 & \parbox[c]{4.5in}{\raggedright rate-based ductile fracture model }\\
\hline
10 & \parbox[c]{4.5in}{\raggedright elastoplastic model for geomaterials}\\
\hline
11 & \parbox[c]{4.5in}{\raggedright rigid-plastic model for geomaterials}\\
\hline
\end{longtable}
\end{center}
\end{table}

\subsubsection{Background}
\label{sect.material.surface.common}

The Tahoe input for the cohesive surface elements described in  
Section~\ref{section.CSE.laws} specify the traction-separation relation(s)
for a given cohesive surface element block by specifying a list of material
codes ranging from 0 to 7 along with any required model parameters. In
the following sections,  ``Surface Material $x$'' refers to the constitutive 
relation
specified by a material code of ``$x$'' for a given element block. 
In these constitutive models, the displacement discontinuity between two
opposite crack faces is quantified by the gap vector $\boldsymbol{\Delta}$
whose components $\Delta_t$ and $\Delta_n$ in two dimensions are specified
for directions along and normal to the midplane between the two crack
faces. In three dimensions, the components are $\Delta_{t1}$, $\Delta_{t2}$,
and $\Delta_n$. Each constitutive model described below returns the
traction vector $\mathbf{T}$ as a function of $\boldsymbol{\Delta}$ and any
internal state variables $\{\mathbf{q}_i\}$ the model has.  
Also, many of these models specify 
length scales $\delta_i, i = n,t$ in the normal and tangential directions
directions ($i = n,t1,t2$ in 3D), and one defines a dimensionless vector
$\boldsymbol{\lambda}$ related to the gap vector $\mathbf{\Delta}$ via
\begin{equation}
\lambda \equiv  \frac{\Delta_i}{\delta_i}.
\end{equation}
With this notation many models define the traction
vector as a function of the $\lambda_i$ or indeed of the
scalar opening parameter $\lambda \equiv \sqrt{\boldsymbol{\lambda}
\cdot\boldsymbol{\lambda}}$.

In addition, many of these cohesive models have a penalty stiffness mulitplier
that stiffens the material under compression in order to improve the poor
behavior of cohesive surface elements in this regime. The penalty stiffness
multiplier is a single number multiplying the natural stiffness that would
be calculated by the cohesive surface model. 

\subsection{Surface Material 0: Xu-Needleman (elastic)}
\label{sect.material.surface.xu}

This model due to Xu and Needleman \cite{Needleman1994} is based 
on the universal binding energy
curve and exhibits complete reversibility; consequently, the traction vector
can be derived from a potential $\Phi(\boldsymbol{\Delta})$ via
\begin{equation}
\mathbf{T} = \frac{\partial \Phi}{\partial \boldsymbol{\Delta}}.
\end{equation}
For the Xu-Needleman model, $\Phi$ is a function of the dimensionless
gap vector $\boldsymbol{\lambda}$ described above and is given by 
\begin{equation}
\Phi(\boldsymbol{\Delta}) = \phi_n (1+e^{-\lambda_n}) (1-r+\lambda_n)\frac{1-q}{r-1}-e^{-\lambda_t^2} ( q+\frac{r-q}{r-1}\lambda_n ).
\end{equation}
Here $\phi_n$ is an input parameter specifying the work to fracture and one has
the identity
\begin{equation}
\phi_n = \int \frac{\partial \phi}{\partial \boldsymbol{\Delta}}\cdot \delta 
\boldsymbol{\Delta}.
\end{equation}
Two additional dimensionless parameters appearing in the expression for
$\Phi$ are $q = \frac{\phi_t}{\phi_n}$
relating the normal to shear fracture energies and $r = \frac{\Delta_n^*}{\delta_n}$
which defines the normal opening $\Delta_n^*$ after pure shear separation at zero
normal traction. The exponential decay of this potential implies an infinite range
of interaction between two crack faces, but the Tahoe implementation of this
model truncates the potential at a distance $r_{fail}\,\delta_n$. A plot
of the normal traction as a function of $\Delta_n$ for pure mode I loading
is shown in Figure~\ref{fig:xutdelta} . 

\subsubsection{Input parameters}

The Xu-Needleman model is implemented for both two- and three-dimensionsal
simulations, and the input parameters are independent of the dimensionsality
of the problem.  An example of these input parameters is shown below:
\begin{inputfile}
####### Xu Needleman
0     # surface material code
####### model parameters
1     # q
0     # r
0.1   # dn
0.1   # dt
1.0   # phi_n
100.0 # r_fail
10.0  # penetration stiffness multiplier 
\end{inputfile}

\subsubsection{Output values}
This constitutive model does not have any output values.

\subsection{Surface Material 1: Tvergaard-Hutchinson (elastic)}
\label{sect.material.surface.tv}
As with the Xu-Needleman model, the model of this section due to Tvergaard
and Hutchinson \cite{Tvergaard1992} is compeletely reversible and uses
the dimensionless gap vector $\boldsymbol{\lambda}$ to characterize the
opening displacement. Unlike the Xu-Needleman model, the Tvergaard-Hutchinson
model first specifies a potential density $\phi$ shown in Figure~\ref{fig:tv} 
and given by
\begin{equation}
\phi(\mathbf{\Delta}) = \left\{
\begin{array}{r@{\quad:\quad}l} 
\sigma_{max} \frac{\lambda}{\Lambda_1} & \lambda < \Lambda_1 \\
	\sigma_{max} & \Lambda_1 < \lambda < \Lambda_2 .\\
     \sigma_{max} \frac{1-\lambda}{1-\Lambda_2} & \Lambda_2 < 
     \lambda < \Lambda_{fail}
\end{array}
\right.
\end{equation} 
The potential $\Phi$ can be obtained from the potential density via
\begin{equation}
\Phi = \delta_n \int_0^\lambda \phi(\tilde{\lambda}) \mbox{\ d}\tilde{\lambda},
\end{equation}
and the traction may be concisely expressed as
\begin{equation}
\label{eq.tvt}
T_i = \frac{\partial \Phi}{\partial \Delta_i} = \frac{\delta_n}{\delta_i} 
\Lambda_i \phi(\lambda).
\end{equation}

\subsubsection{Input parameters}
This cohesive model is implemented in two- and three-dimensions, and it should
be noted that the zero-stiffness plateau in the traction-separation relation
when $\Lambda_1 \neq \Lambda_2$ can lead to poorer-than-quadratic or 
convergence or no convergence at all in certain problems. 
An example of the input parameters for the Tvergaard-Hutchinson model 
follows: 
\begin{inputfile}
####### Tvergaard-Hutchinson
1     # material code
####### model parameters
100.0 # sigma 
0.01  # dn
0.01  # dt
0.1   # Lambda_1
0.9   # Lambda_2
1.0   # Lambda_fail (set to 1 if lambda_fail < 1)
0.    # penetration stiffness multiplier
\end{inputfile}

\subsubsection{Output values}
This constitutive model has a single nodal output value; it prints the 
value of $\lambda$.

\subsection{Surface Material 2: Linear Damage Model}

This model is not currently implemented in three-dimensions.

\subsubsection{Input parameters}
An example of input parameters for the linear damage model is shown 
below:
\begin{inputfile}
####### linear damage code
2     # surface material code
####### model parameters
0.01  # dn
0.01  # dt
0.0   # penetration stiffness multiplier
\end{inputfile}

\subsubsection{Output values}
This constitutive model does not have any output values.

\subsection{Surface Material 3: Tvergaard-Huthcinson with viscous dissipation}
\label{sect.material.surface.tvvisc}
This model generalizes the Tvergaard-Hutchinson model 
described in Section~\ref{sect.material.surface.tv} to include the effects of 
viscous dissipation. Consequently, this model does not exhibit reversability 
upon unloading, and may not be completely described by a potential. 
The traction $\mathbf{T}$ is now a function of $\dot{\boldsymbol{\Delta}}$, 
the rate of change of the gap vector, as well as a function of $\boldsymbol{\Delta}$ itself.
It may be expressed as 
\begin{equation}
\mathbf{T} = \mathbf{T}^{tv} + \eta_0 (1-\lambda) \dot{\boldsymbol{\Delta}}
\end{equation}
where $\mathbf{T}^{tv}$ denotes the traction as a function of 
$\boldsymbol{\Delta}$ specified by
the Tvergaard-Hutchinson model in Equation~\eqref{eq.tvt}. 

\subsubsection{Input parameters}
Sample input parameters are shown below:
\begin{inputfile}
####### Tvergaard-Hutchinson with viscous dissipations
3     # surface material code
####### model parameters
100.0 # sigma
0.01  # dn
0.01  # dt
0.10  # lambda_1
0.90  # lambda_2
1.0   # lambda_fail (set to 1 if lambda_fail < 1)
0.01  # eta_0
0.0   # penalty stiffness multiplier
\end{inputfile}

\subsubsection{Output values}
This constitutive model outputs nodal values of $\lambda$ and the energy dissipated
by the viscous term in the traction-separation relation.

\subsection{Surface Material 4: Tijssens' crazing model (elastic-plastic)}
\label{sect.material.surface.tij}

This model is due to Tijssens \etal \cite{Tijssens2001},  
and models
crazing in amorphous polymers. This cohesive law has three distinct regimes
corresponding to elastic deformation until craze initiation, plastic
deformation as the craze fibrils lengthen, and finally
failure as the crazes lose their load-carrying capacity. The elastic
regime is linear elastic with traction-separation relation 
\begin{equation}
T_i = k_i \Delta_i 
\end{equation}
where the normal and tangential stiffnesses $k_i; i = n, t$ are 
parameters of the model and summation over repeated indices is not
implied. Elastic deformation continues
until the stress state in the bulk material surrounding the cohesive
layer satisfies a yield condition. For mode I loading,
the yield condition reduces to specifying a critical normal traction for
craze initiatiation. For more general loading, the stress
tensor may be sampled in the surrounding bulk, and the initiation criterium
is given by a function of $T_n$ and the bulk stress tensor $\mathbf{\sigma}$
\begin{equation}
f(\mathbf{\sigma},T_n) = -\frac{1}{2} \sigma_{ii} - \frac{1}{2} A_\theta - \frac{B_\theta}{2 \sigma_{ii}} - T_n = 0
\end{equation}
The parameters $A_\theta$ and  $B_\theta$ are temperature dependent 
according to
\begin{equation}
X_\theta = X_0 e^{-\frac{Q_X}{k_B \theta}}, X = A, B.
\end{equation}
After craze initiation, the traction-separation relation is specified
in rate form as 
\begin{equation}
\dot{T}_i = 
k_i (\dot{\Delta}_i - \dot{\Delta}_i^{cr}) 
\end{equation}
The components $\Delta_i^{cr}$ represent the span of craze fibrils between
the surfaces of the cohesive layer and evolve according to 
\begin{equation}
\dot{\Delta}_i = \dot{\Delta}_i^0 (e^{-\frac{K^*}{k_B \theta} 
(\tau_i-T_i)}-\delta_{it} e^{-\frac{K^*}{k_B \theta} (\tau_t-T_t)}).
\end{equation}
Here, $k_B \theta$ is Boltzmann's constant multiplied by the absolute
temperature (assumed constant throughout the simulation), $K^*$ and the
$\tau_i$ are input parameters governing the kinetics of craze growth,
and the Kronecker delta, $\delta_{it}$, is included to ensure that $T_t$
is an even function of $\Delta_t$. It is assumed that $\tau_n = \sqrt{3} 
\tau_t$ and $\dot{\Delta}_t^0 = \sqrt{3} \dot{\Delta}_n^0$ in analogy 
to von Mises' stresses. Also, the normal elastic stiffness 
$k_n$ is constant while the tangential elastic stiffness $k_t$ softens
according to 
\begin{equation}
k_t = k_{t0}\, e^{-c_t \Delta_n^{cr}/\Delta^{crit}}.
\end{equation}
Once the craze width 
$\Delta_n^{cr}$ reaches a critical opening $\Delta^{crit}$, the craze
is assumed to have failed and lost its load-carrying capacity. 
In the Tahoe implementation of this model, the traction is reduced to
zero over $n_{fail}$ timesteps with tangential and normal stiffnesses
given by $k_t$ and $k_n$, respectively. The result for pure mode I 
loading at constant rate yields the traction-separation
curves shown in Figure~\ref{fig:tij}. The hardening behavior is seen when 
$\dot{\Delta}_n < \dot{\Delta}_n^0$; otherwise softening behavior is 
observed.

\subsubsection{Input Parameters}
The input parameters for this model are as shown below:
\begin{inputfile}
####### Tijssens
4     # surface material code
####### model parameters
100.0 # k_t^0
100.0 # k_n
50.0  # decay constant for k_t^0
0.005 # critical craze opening width
1.0   # A_0 - mixed-mode initiation parameter
1.0   # B_0 - mixed-mode initiation parameter
1.0   # Q_A - mixed-mode initiation parameter
1.0   # Q_B - mixed-mode initiation parameter
0.001 # deltadot_n^0 - should be of order the opening rate, Delta_dot
10.0  # tau_n
1.0   # K^star
273.0 # absolute temperature
1     # integer > 0 giving bulk element group to obtain stress from
10    # integer number of timesteps n_fail
\end{inputfile}

\subsubsection{Output Values}

This cohesive model outputs four nodal values. The first is the local
normal opening rate $\dot{\Delta}_n$. The second is the tangential
craze width, $\Delta_t^{cr}$, and the third is the normal component
of the same, $\Delta_n^{cr}$. The final output parameter is the initiation
function $f(\sigma,T_n)$.

\subsection{Surface Material 5: trilinear rate dependent}
\label{sect.material.surface.ratedep}

This model is a generalization of the Tvergaard-Hutchinson model of 
Section~\ref{sect.material.surface.tv}; its primary distinction is that
the critical normal
length scale $\delta_n$ varies as a function of the normal opening rate
$\dot{\delta}_n$. The material is assumed to yield when it first reaches the
plateau traction (for pure mode I loading) $\sigma^m$, and at this yield point
$\delta_n$ is set as a function of rate given by
\begin{equation}
\delta_n = \delta_n^b + \delta_n^m \log{\dot{\Delta}_n}
\end{equation} 
where $\log$ denotes the natural logarithm. Before the critical normal
traction is reached, 
the normal length scale is rate-independent and given as an input paramter. 
Once the normal
length scale has been set for a particular integration point, 
the parameter is finalized there and that point is no longer sensitive to 
variations in the local opening rate.  

A second distinction between this trilinear model and Tvergaard and
Hutchinson's is that the traction-separation relation is no longer
constrained to have a plateau (region of zero stiffness); there is
an additional parameter $m_p$ allowing nonzero stiffness and better
numerical performance. The traction on the plateau is determined 
by $m_p$ via
\begin{equation}
\mathbf{T}(\lambda) = \mathbf{T}(\Lambda_1)(1 + m_p (\lambda - \Lambda_1)),
\end{equation}
where $\Lambda_1$ has the same meaning as in the Tvergaard-Hutchinson model
of Section~\ref{sect:material.surface.tv}. Likewise, $\mathbf{T}(\Lambda_1)$
is idential to that determined by Surface Material 1 since the loading rate
effects the plastic failure portion of the model and not the elastic 
loading portion. 

\subsubsection{Input parameters}
An example of the input parameters for the trilinear, rate-dependent model 
is  shown below:
\begin{inputfile}
####### rate-dependent Tvergaard-Hutchinson
5     # surface material code
####### model parameters
100.0 # sigma
0.25  # rate-independent normal length scale, dn
0.25  # dt
0.1   # Lambda_1
0.9   # Lambda_2
1.0   # Lambda_fail (set to 1 if lambda_fail < 1)
0.0   # penalty stiffness multiplier
1.0   # delta_n^b (intercept of rate function)
1.0   # delta_n^m (slope or rate functions)
0.0   # m_p
\end{inputfile}
Setting ``delta\_n\_b'' equal to ``d\_n'' and 
``delta\_n\_m'' and ``m\_p'' to zero in the input 
file reduces this model to the rate-independent Tvergaard-Hutchinson one of 
Section~\ref{sect.material.surface.tv}.

\subsubsection{Output values}
This model outputs nodal values for $\lambda$, each component of the local opening
rate $\mathbf{\dot{\Delta}}$, and the rate-dependent normal length scale $\delta_n$.

\subsection{Surface Material 6: Tied Potential}
\label{sect.material.surface.tied}

This material surface model works in conjunction with the TiedNodes 
KBC Controller 3, and
its behavior is unpredictable unless the TiedNodes controller
is activated. This co-dependency is used in an attempt to
remedy some shortcomings of cohesive surface models. To begin with,
cohesive surface models are in general designed to model the loss of a 
material's load-carrying capacity across a crack face, yet by necesity
the traction-separation relation must also specify the response of this
region of material to loading before this capacity is assumed to have
been lost. Consequently, there is \it a priori \rm no relation between
the bulk material and the initial (presumably) elastic loading portion
of the traction separation relation. The tied nodes KBC Controller constrains
the two faces separated bridged by a cohesive surface to admit no
displacement discontinuity across them until an initiation criterium is
reached, and consequently, the cohesive surface element is inactive until 
its nodes are ``untied'' at initiation. To this end,
the tied potential provides a function 
TiedPotentialT::InitiationQ that 
determines if a critical traction has been reached for yield to occur
and for the material to exhibit cohesive failure. Although a minimal
initiation function is provided and will be described below, this 
function should in general be re-implemented by the end user. Since
the material response is that of the bulk until the nodes are untied,
this model should provide greater numerical stability in situations where
the cohesive layer may reasonably be expected to undergo severe compression
or other deformations for which a cohesive surface element may be 
ill-equipped. 

Currently, two traction-separation laws may be used after the nodes
are untied. The first is the Xu-Needleman model of 
Section~\ref{sect.material.surface.xu} and the second is the 
Tvergaard-Hutchinson model of Section~\ref{sect.material.surface.tv}. The
Xu-Needleman implementation begins the traction separation law at the peak
stress of $\phi_n/(e \delta_n)$ in the notation of 
Section~\ref{sect.material.surface.xu}, so that the materials yield
criterium is taken to be one of maximum normal traction. 

The Tvergaard and Hutchinson tied-potential implementation allows
the user to specify the initiation point in terms of the scalar 
opening parameter $\lambda$ and the model parameters given in
Section~\ref{sect.material.surface.tv}. More concretely, a given
value of $\lambda$ may be inverted to determine a specific traction
that acts as the initiation traction. Since the initiation traction is
not constrained to be the maximum traction, this choice of tied potential
is possibly more robust than the Xu-Needleman choice, but it is 
plagued by poorer numerical performance. 

\subsubsection{Input parameters}
The tied potential input parameters contain a flag that is set to
zero to choose the Tvergaard-Huthcinson model and to one to choose
the Xu-Needleman model. Examples of both sets of input will be shown.

Below is shown sample input for a tied Xu-Needleman model:
\begin{inputfile}
####### Tied potential
6     # surface material code
0. 1. # normal direction to sample stress to obtain traction
####### surface material code for Xu-Needleman
0
####### model parameters
1.0   # q
0.0   # r 
0.001 # dn
0.001 # dt
10.0  # phi
10.0  # r_fail
\end{inputfile}

For the tied Tvergaard-Hutchinson model:
\begin{inputfile}
####### Tied potential
6     # surface material code
0. 1. # normal direction to sample stress to obtain traction
####### surface material code for Tvergaard-Hutchinson
1
####### model parameters
100.0 # sigma
0.001 # dn
0.001 # dt
0.25  # lambda_1
0.75  # lambda_2
1.0   # lambda_3
0.25  # lambda_0, initiate at T = T(lambda_0)
\end{inputfile}

Although the input parameters have been ordered to correspond exactly
with the ones for their respective untied cohesive laws, it should be
noted that there are no penalty stiffness multipliers for interpenetration,
and that the mode I formalism currently used overrides any effects in
the Xu-Needleman specification other than $q = 1$, $r = 0$. 


\subsubsection{Output values}
The tied potentials ouput a single nodal value that reflects a given
nodes status as tied ($\neq 0$) or untied ($0.$) and may be useful 
in conjunction with visualization tools. 

\subsection{Material 7: Yoon, Allen, and Searcy (linear viscoelastic)}
\label{sect.material.surface.yoon}
This model presented initially by Yoon and Allen and later by
Searcy and Allen models a cohesive layer as a linear viscoelastic material
that undergoes damage as it loses its load-carrying capacity. An undamaged
material will respond exactly as a linear viscoelastic material, while
a material undergoing damage will require the specificaton of a 
damage evolution law. The details of the physical interpretation of 
the damage parameter $\alpha$ are given in Reference \cite{Allen1999}, and 
the traction-separation law is given as
\begin{equation}
\mathbf{T} = \boldsymbol{\lambda} (1-\alpha(t))\left[ T_0 + \int_0^t \left( E_\infty + \sum_i^{n_e} (1-E_i 
e^{-\frac{t}{\tau_i}})\right) \frac{\partial \lambda(\tau')}{\partial \tau'} \mbox{ d}
\tau' \right]
\end{equation}
where the $E_i$ and $\tau_i$ are elastic moduli and relaxation times
corresponding to relaxation process $i$, and $E_\infty$ is the equilibrium
elastic modulus as for a Prony series in a standard linear solid. 
The model as published contains an initiation traction $T_0$, but 
calculations using the model set $T_0$ to zero, the same convention
currentely used in Tahoe. There is a choice of
damage evolution laws where $\dot{\alpha}$ is specified as a function
of $\lambda$, $\dot{\lambda}$, and $\alpha$ itself. In all the damage
evolution laws, it is assumed that $\alpha$ does not evolve for closing
cohesive surfaces, i.e when $\dot{\lambda}$ is non-negative. 

The first damage law is given in Reference \cite{Allen2001} as
\begin{equation}
\dot{\alpha} = \alpha_0 \lambda^m,
\end{equation}
and the response of this evolution law to cyclic loading is shown 
in Figure . A second evolution law is given in Reference \cite{Allen1999} as
\begin{equation}
\dot{\alpha} = \alpha_0 (1-\alpha^n) (1- b \lambda^m),
\end{equation}
and the response of this law to identical cyclic loading as before
is shown in Figures .
A third and for the moment final damage evolution law is given by
\begin{equation}
\dot{\alpha} = \alpha_0 \dot{\lambda}^m,
\end{equation}
and its response to cyclic loading is shown in Figure .

\subsubsection{Input parameters}
An example of input parameters for two- or three-dimensional analysis 
is shown below:
\begin{inputfile}
######## Yoon-Allen-Searcy
7      # potential code
0.     # fsigma_0
1.     # d_c^t - critical tangential length scale
1.     # d_c^n - critical normal length scale
10.    # E_infinity - equilibrium modulus
2      # number of elements in prony series (nprony)
10.    # first relaxation modulus
10.    # first relaxation time
100.   # second relaxation modulus
2.     # second relaxation time
 .
 .
 .
100.   # nprony_th relaxation modulus
100.   # nprony_th relaxation time
1      # damage evolution law code
       # 1: alpha_dot = alpha_0*l^alpha_exp
       # 2: alpha_dot = 
       #     ((alpha_0(1-alpha)^alpha_exp)(1-lambda_0)^lambda_exp)
       # 3: alpha_dot = alpha_0*lambda_dot^alpha_exp
2.     # alpha_exp
10.    # alpha_0
#-3.   # lambda_exp # only for damage code 2
#10.   # lambda_0  # only for damage code 2
10.    # penetration stiffness multiplier
\end{inputfile}


\subsubsection{Output values}
This constitutive model has three nodal output values. It outputs values
of $\lambda$, $\dot{\lambda}$, and $\alpha$.

