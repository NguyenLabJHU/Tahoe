%&LaTeX
% $Id: element.tex,v 1.5 2001-06-04 01:32:53 paklein Exp $

\section{Element types}
Within the structure of \tahoe, any formulation that relates nodal 
unknowns to the corresponding conjugate forces falls into the category 
of \textit{elements}.
A list of the element
codes are shown in Table~\ref{tab.element.types}.  
The category of continuum elements is broken
into three parts: Solid, Diffusion, and Mesh Free.  Although mesh free
methods do not use elements, they do need interpolation schemes, these
schemes are discussed in this chapter.
\begin{table}[h]
\caption{\label{tab.element.types} Element types.}
\begin{center}
\begin{tabular}[c]{|c|c|c|c|}
\hline
% table headings
\parbox[b]{1.5in}{\centering \textbf{type}}
&\parbox[b]{1.5in}{\centering \textbf{sub-type}}
&\parbox[b]{0.75in}{\centering \textbf{code}}
&\parbox[b]{0.75in}{\centering \textbf{sub-code}}\\
\hline
% finite element continuum
\multirow{4}{1.5in}{\centering finite element continuum} & small deformation & 2 &
\multirow{4}{0.75in}{\centering N/A}\\
\hhline{~--~}
& updated Lagrangian & 3 & \\
\hhline{~--~}
& total Lagrangian & 17 & \\
\hhline{~--~}
& linear diffusion & 21 & \\
\hline
% meshfree continuum
\multirow{3}{1.5in}{\centering meshfree continuum} & small deformation & 18 &
\multirow{3}{0.75in}{\centering N/A}\\
\hhline{~--~}
& total Lagrangian & 19 & \\
\hhline{~--~}
& strain gradient total Lagrangian & 20 & \\
\hline
% cohesive surface elements
\multirow{3}{1.5in}{\centering finite element cohesive surface} & isotropic & 
\multirow{3}{0.75in}{\centering 11} & 0\\
\hhline{~-~-}
& anisotropic with rotating frame & & 1\\
\hhline{~-~-}
& anisotropic with fixed frame & & 2\\
\hline
% meshfree cohesive surface elements
meshfree cohesive surface & 
anisotropic with rotating frame & 22 & N/A\\
\hline
% contact
\multirow{3}{1.5in}{\centering contact} & penalty & 14& 
\multirow{3}{0.75in}{\centering N/A}\\
\hhline{~--~}
& augmented Lagrangian (2D) & 16 & \\
\hhline{~--~}
& \textsf{ACME}~\cite{ACME} library & 23 & \\
\hline
% pair-potential
\multirow{3}{1.5in}{\centering pair-potential} & predefined & 1 &
\multirow{3}{0.75in}{\centering N/A}\\
\hhline{~--~}
& self-connecting & 8 & \\
\hhline{~--~}
& 8 with periodic BC's & 9 & \\
\hline
% two-body and three-body potentials
\multirow{3}{1.5in}{\centering two- and three-body potential} 
& Stillinger-Weber~\cite{Stillinger1985}  & 6 &
\multirow{3}{0.75in}{\centering N/A}\\
\hhline{~--~}
& mixed Stillinger-Weber & 7 & \\
\hhline{~--~}
& 6 with periodic BC's & 10 & \\
\hline
\end{tabular}
\end{center}
\end{table}

\subsection{Integration domains}
\label{sect.integration.domains}
\tahoe supports a number of Gaussian integration rules for the finite 
element types listed in Table~\ref{tab.element.types}. These rules 
are defined for the geometries listed in Table~\ref{tab.domain.types}.
Additionally, the meshfree formulations make use of a background  grids
composed of ``integration cells'' with the same geometries and 
integration rules.
\begin{table}[h]
\caption{\label{tab.domain.types} Integration domain geometries and 
Gauss rules.}
\begin{center}
\begin{tabular}[c]{|c|c|c|c|}
\hline
% table headings
\parbox[b]{1.5in}{\centering \textbf{geometry}}
&\parbox[b]{0.75in}{\centering \textbf{nodes}}
&\parbox[b]{1.5in}{\centering \textbf{Gauss rule}}
&\parbox[b]{0.75in}{\centering \textbf{code}}\\
\hline
point & 1 & 1 & -1 \\
\hline
line & 2, 3 & 1, 2, 3, 4 & 0 \\
\hline
quadrilateral & 4--8 & 1, 4, 9, 16 & 1 \\
\hline
triangle & 3, 6 & 1, 4, 6 & 2 \\
\hline
hexahedron & 8, 20 & 1, 8, 27, 64 & 3 \\
\hline
tetrahedron & 4, 10 & 1, 4 & 4 \\
\hline
pentahedron & 6, 15 & N/A & 5 \\
\hline
\end{tabular}
\end{center}
\end{table}
Each of these geometries has a corresponding local node numbering. 
The local node numbering, as well as the local ordering of the facets 
is shown in Figures~\ref{fig.line.element} 
through~\ref{fig.pent.element}. Although the pattern of local node 
numbering is not derived from a set of rules, it does display several 
common characteristics for all domain geometries. The vertex 
nodes always have lower local numbers that the mid-side nodes. For 
the three-dimensional domains, the local numbering of the facet nodes 
is always defined in a sense, or line direction, consistent with the 
outward normal to the face. Finally, the local node numbering for the 
follows the convention used by 
\textsf{ABAQUS}~\cite{ABAQUSv56} and \textsf{EnSight}~\cite{EnSight6}, 
which is not the same as the 
convention used by the \textsf{ExodusII}~\cite{ExodusII} database format
for some domain geometries.
However, the numbering of nodes on facets differs from the convention used
by \textsf{ABAQUS} in that element normals point outward.
When reading from or writing to 
an \textsf{ExodusII} database, \tahoe adjusts for 
any differences in the numbering convention.

%% line
\begin{figure}
\begin{center}
\begin{tabular}[c]{c c}
\parbox[c]{2.5in}{
\begin{center}
\includegraphics[scale=1.0]{\dirfilepath{\figpath}{line.eps}}
\end{center}
}
& 
\parbox[c]{2.5in}{\centering
\begin{tabular}[c]{|c|c|}
\hline
\textbf{face} & \textbf{node}\\
\hline
1 & 1 \\
2 & 2 \\
\hline
\end{tabular}
}
\end{tabular}
\end{center}
\hangcaption{Node, face, and node-on-face numbering for line integration domains.}
\label{fig.line.element}
\end{figure}

%% quad
\begin{figure}
\begin{center}
\begin{tabular}[c]{c c}
\parbox[c]{2.5in}{
\begin{center}
\includegraphics[scale=1.0]{\dirfilepath{\figpath}{quad.eps}}
\end{center}
}
& 
\parbox[c]{2.5in}{\centering
\begin{tabular}[c]{|c|c|}
\hline
\textbf{face} & \textbf{nodes}\\
\hline
1 & 1, 2, 5 \\
2 & 2, 3, 6 \\
3 & 3, 4, 7 \\
4 & 4, 1, 8 \\
\hline
\end{tabular}
} \\
\end{tabular}
\end{center}
\hangcaption{Node, face, and node-on-face numbering for quadrilateral integration 
domains.}
\label{fig.quad.element}
\end{figure}

%% tri
\begin{figure}
\begin{center}
\begin{tabular}[c]{c c}
\parbox[c]{2.5in}{
\begin{center}
\includegraphics[scale=1.0]{\dirfilepath{\figpath}{tri.eps}}
\end{center}
}
& 
\parbox[c]{2.5in}{\centering
\begin{tabular}[c]{|c|c|}
\hline
\textbf{face} & \textbf{nodes}\\
\hline
1 & 1, 2, 4 \\
2 & 2, 3, 5 \\
3 & 3, 4, 6 \\
\hline
\end{tabular}
} 
\end{tabular}
\end{center}
\hangcaption{Node, face, and node-on-face numbering for triangular integration 
domains.}
\label{fig.tri.element}
\end{figure}

%% hex
\begin{figure}
\begin{center}
\begin{tabular}[c]{c c}
\parbox[c]{2.5in}{
\begin{center}
\includegraphics[scale=1.0]{\dirfilepath{\figpath}{hex.eps}}
\end{center}
}
& 
\parbox[c]{2.5in}{\centering
\begin{tabular}[c]{|c|c|}
\hline
\textbf{face} & \textbf{nodes}\\
\hline
1 & 1, 4, 3, 2, 12, 11, 10, 9 \\
2 & 5, 6, 7, 8, 13, 14, 15, 16 \\
3 & 1, 2, 6, 5, 9, 18, 13, 17 \\
4 & 2, 3, 7, 6, 10, 19, 14, 18 \\
5 & 3, 4, 8, 7, 11, 20, 15, 19 \\
6 & 4, 1, 5, 8, 12, 17, 16, 20 \\
\hline
\end{tabular}
} \\
\end{tabular}
\end{center}
\hangcaption{Node, face, and node-on-face numbering for hexahedral integration 
domains.}
\label{fig.hex.element}
\end{figure}

%% hex
\begin{figure}
\begin{center}
\begin{tabular}[c]{c c}
\parbox[c]{2.5in}{
\begin{center}
\includegraphics[scale=1.0]{\dirfilepath{\figpath}{tet.eps}}
\end{center}
}
& 
\parbox[c]{2.5in}{\centering
\begin{tabular}[c]{|c|c|}
\hline
\textbf{face} & \textbf{nodes}\\
\hline
1 & 1, 2, 4, 5, 9, 8 \\
2 & 2, 3, 4, 6, 10, 9 \\
3 & 3, 1, 4, 7, 8, 10 \\
4 & 1, 3, 2, 7, 6, 5 \\
\hline
\end{tabular}
} \\
\end{tabular}
\end{center}
\hangcaption{Node, face, and node-on-face numbering for tetrahedral integration 
domains.}
\label{fig.tet.element}
\end{figure}

%% hex
\begin{figure}
\begin{center}
\begin{tabular}[c]{c c}
\parbox[c]{2.5in}{
\begin{center}
\includegraphics[scale=1.0]{\dirfilepath{\figpath}{pent.eps}}
\end{center}
}
& 
\parbox[c]{2.5in}{\centering
\begin{tabular}[c]{|c|c|}
\hline
\textbf{face} & \textbf{nodes}\\
\hline
1 & 1, 3, 2, 9, 8, 7 \\
2 & 4, 5, 6, 10, 11, 12 \\
3 & 1, 2, 5, 4, 7, 14, 10, 13\\
4 & 2, 3, 6, 5, 8, 15, 11, 14 \\
5 & 3, 1, 4, 6, 9, 13, 12, 15 \\
\hline
\end{tabular}
} \\
\end{tabular}
\end{center}
\hangcaption{Node, face, and node-on-face numbering for pentahedral integration 
domains.}
\label{fig.pent.element}
\end{figure}

\subsection{Meshfree methods}
\label{section.meshfree.methods}
Recently, there has been intense interest in the development of 
meshfree numerical simulation methods. 
The methods used in \tahoe
have their origins in Smoothed Particle Hydrodynamics (SPH) 
proposed by Gringold and Monaghan\cite{Gingold1977}.
A number of methods have developed that employ
variants of the Moving Least Squares fitting procedure
of Lancaster and Salkauskas\cite{Lancaster1981}.
These include, but are not limited to,
the Diffuse Element Method of
Nayroles \etal\cite{Nayroles1992},
the Element Free Galerkin method of 
Belytschko \etal\cite{Belytschko1994b},
the Reproducing Kernel Particle Method of
Liu and Chen\cite{Liu1995a},
and the Meshless Local Petrov-Galerkin method of
Atluri and Zhu\cite{Atluri1998a}. 
A number of overviews and reviews
have appeared regarding these 
methods\cite{Belytschko1996c,Liu1996b,Liu1996c,Liu1999d,Liu2000b}, 
though the interested reader will
find no shortage of new developments.

\subsubsection{Reproducing Kernel Particle Method (RKPM)}
\label{section.RKPM}
RKPM belongs to class of methods for which the
approximation, or ``image'', of a signal is given by a kernel 
expression. Without loss of generality, we can consider the 
expression for the approximation in one dimension
\begin{equation}
u^{R_{a}}\farg{x} = \int \limits_{-\infty}^{+\infty}
    \phi_{a}\farg{x-y}\,u\farg{y}\,dy,
\end{equation}
where $\phi_{a}$ is alternately called a weight, kernel, or smoothing 
function. From the analogy to signal processing, 
$\phi_{a}$ may be viewed as a customizable low pass filter between 
the original signal, or data, $u\farg{y}$ and its reproduced image.
This function is positive, even, and has compact support characterized
by the dilation parameter $a$. Liu and Chen\cite{Liu1995a} improved the
accuracy of the method by modifying the window function with a
correction to yield a reproducing condition as
\begin{equation}
\label{eq.RKPM.RK}
u^{R_{a}}\farg{x} = \int \limits_{-\infty}^{+\infty}
    \overline{\phi}_{a}\farg{x-y}\,u\farg{y}\,dy,
\end{equation}
where the modified window function
\begin{equation}
\overline{\phi}_{a}\farg{x-y} = C\farg{x; x-y}\,\phi_{a}\farg{x-y}
\end{equation}
incorporates the polynomial
\begin{equation}
C\farg{x; x-y} = b_{0}\farg{x} +
    b_{1}\farg{x}\,\rbrkt{x-y} +
    b_{2}\farg{x}\,\rbrkt{x-y}^{2} + \ldots +
    b_{m}\farg{x}\,\rbrkt{x-y}^{m}
\end{equation}
which ensures that the approximation can exactly represent
polynomials of order $m$. In general,
we may express the correction function as
\begin{equation}
C\farg{\mathbf{x}; \mathbf{x}-\mathbf{y}} =     
    \mathbf{b}\farg{\mathbf{x}}\cdot
    \mathbf{P}\farg{\mathbf{x}-\mathbf{y}}
\end{equation}
where $\mathbf{P}$ is a basis of polynomials
that possess the desired degree of completeness and 
$\mathbf{b}\farg{\mathbf{x}}$ is an vector of unknown
coefficients determined from the reproducing condition~\eqref{eq.RKPM.RK}.
The requirement that each member of the basis be reproduced
follows from~\eqref{eq.RKPM.RK} as
\begin{equation}
\mathbf{P}\farg{\mathbf{0}}= 
    \int \limits_{-\infty}^{+\infty}
    \mathbf{b}\farg{\mathbf{x}}\cdot
    \mathbf{P}\farg{\mathbf{x}-\mathbf{y}}\,
    \phi_{a}\farg{\mathbf{x}-\mathbf{y}}\,
    \mathbf{P}\farg{\mathbf{x}-\mathbf{y}}\,d\mathbf{y}.
\end{equation}
The vector of coefficients follows as
\begin{equation}
\label{eq.RKPM.b}
\mathbf{b}\farg{\mathbf{x}}=\mathbf{M}_{a}^{-1}\farg{\mathbf{x}}
    \mathbf{P}\farg{\mathbf{0}},
\end{equation} 
where
\begin{equation}
\label{eq.RKPM.moment}
\mathbf{M}_{a}\farg{\mathbf{x}} =
    \int \limits_{-\infty}^{+\infty}
    \mathbf{P}\farg{\mathbf{x}-\mathbf{y}}
    \otimes
    \mathbf{P}\farg{\mathbf{x}-\mathbf{y}}\,
    \phi_{a}\farg{\mathbf{x}-\mathbf{y}} d\mathbf{y}
\end{equation} 
is known as the moment matrix. Using the result from~\eqref{eq.RKPM.b},
the reproducing condition~\eqref{eq.RKPM.RK} can be written as
\begin{equation}
\label{eq.RKPM.RK2}
\mathbf{u}^{R_{a}}\farg{\mathbf{x}}= \int \limits_{-\infty}^{+\infty}
    \mathbf{b}\farg{\mathbf{x}}
    \cdot \mathbf{P}\farg{\mathbf{x}-\mathbf{y}} 
    \phi_{a}\farg{\mathbf{x}-\mathbf{y}}\,
    \mathbf{u}\farg{\mathbf{y}}\,d\mathbf{y}.
\end{equation}
First and higher order gradients of the field representation follow 
from~\eqref{eq.RKPM.b}--\eqref{eq.RKPM.RK2}
through simple differentiation. For clarity, we present the expressions 
for derivatives. We note that these derivatives may be evaluated
exactly though some early work in the meshfree field advocated
approximate expressions for these derivatives\cite{Nayroles1992}. The
gradient of the field may be expressed as
\begin{equation}
\label{eq.RKPM.du}
\begin{split}
\frac{\partial \mathbf{u}^{R_{a}}\farg{\mathbf{x}}}
     {\partial \mathbf{x}}=
 \int \limits_{-\infty}^{+\infty}
    \mathbf{u}\farg{\mathbf{y}}\otimes
    & \left[ \frac{\partial \mathbf{b}\farg{\mathbf{x}}}{\partial \mathbf{x}} 
    \right.
    \,\mathbf{P}\farg{\mathbf{x}-\mathbf{y}}
    \,\phi_{a}\farg{\mathbf{x}-\mathbf{y}} + \\
    &\mathbf{b}\farg{\mathbf{x}}\,
    \frac{\partial \mathbf{P}\farg{\mathbf{x}-\mathbf{y}}}{\partial \mathbf{x}} 
    \,\phi_{a}\farg{\mathbf{x}-\mathbf{y}} + \\
    &\mathbf{b}\farg{\mathbf{x}}
    \cdot \mathbf{P}\farg{\mathbf{x}-\mathbf{y}}\,
    \left. \frac{\partial \phi_{a}\farg{\mathbf{x}-\mathbf{y}}}{\partial 
    \mathbf{x}} \right]
    d\mathbf{y},
\end{split}
\end{equation}    
where the gradient of the coefficient vector
\begin{equation}
\label{eq.RKPM.db}    
\frac{\partial \mathbf{b}\farg{\mathbf{x}}}{\partial \mathbf{x}}=
-\mathbf{M}_{a}^{-1}\farg{\mathbf{x}}
\frac{\partial \mathbf{M}_{a}\farg{\mathbf{x}}}{\partial \mathbf{x}}\,
\mathbf{b}\farg{\mathbf{x}}
\end{equation}
follows from~\eqref{eq.RKPM.b} making use of the relation
\begin{equation}
\label{eq.RKPM.dMinv}    
\frac{\partial \sbrkt{\mathbf{M}_{a}^{-1}}_{ij}}{\partial x_{k}}=
-\sbrkt{\mathbf{M}_{a}^{-1}}_{ir} 
\frac{\partial \sbrkt{\mathbf{M}_{a}}_{rs}}{\partial x_{k}}\,
\sbrkt{\mathbf{M}_{a}^{-1}}_{sj}.
\end{equation}    
Higher-order gradients may be calculated by further differentiation 
of~\eqref{eq.RKPM.du}--\eqref{eq.RKPM.dMinv}.
Notably, the expressions for the representation of the unknown field
variables are general for an arbitrary number of field and spatial
dimensions, including the expressions for the first and higher order
gradients of the field.

In evaluating the representations for the field numerically, we 
discretize the integrals in the previous expressions. The discrete reproducing 
condition follows from~\eqref{eq.RKPM.RK2} as
\begin{equation}
\label{eq.RKPM.u.discrete}    
\mathbf{u}^{R_{a}^{{h}}}\farg{\mathbf{x}}= \sum \limits_{I=1}^{N_{p}}
    \mathbf{b}\farg{\mathbf{x}}
    \cdot \mathbf{P}\farg{\mathbf{x}-\mathbf{x}_{I}} 
    \phi_{a}\farg{\mathbf{x}-\mathbf{x}_{I}}\,
    \mathbf{u}_{I}\,\Delta V_{I},
\end{equation}    
where $N_{p}$ is the number of sampling points, or particles, under 
consideration, and $\mathbf{x}_{I}$,
$\mathbf{u}_{I}=\mathbf{u}\farg{\mathbf{x}_{I}}$,
and $\Delta V_{I}$ are the coordinates, field value, and
integration weight (volume) associated with particle $I$,
respectively. 
From~\eqref{eq.RKPM.u.discrete}, we can identify the RKPM nodal 
shape functions as
\begin{equation}
\label{eq.RKPM.shape.expansion}    
\mathbf{u}^{R_{a}^{{h}}}\farg{\mathbf{x}}= \sum \limits_{I=1}^{N_{p}}
\Phi_{aI}\farg{\mathbf{x}}\,\mathbf{u}_{I},
\end{equation}    
where
\begin{equation}
\label{eq.RKPM.shape}    
\Phi_{aI}\farg{\mathbf{x}}=
    \mathbf{b}\farg{\mathbf{x}}
    \cdot \mathbf{P}\farg{\mathbf{x}-\mathbf{x}_{I}} 
    \phi_{a}\farg{\mathbf{x}-\mathbf{x}_{I}}\,\Delta V_{I}.
\end{equation}    
Unlike the shape functions derived for standard finite element methods,
the RKPM shape functions~\eqref{eq.RKPM.shape} do not in general possess the 
so-called Kronecker delta property, that is
\begin{equation}
\Phi_{aI}\farg{\mathbf{x}_{J}} \not = \delta_{IJ}.
\end{equation}
The lack of this property complicates the enforcement of essential 
boundary conditions. Furthermore, this characteristic affects the 
enforcement of contact constraint and the imposition of cohesive 
tractions.

\todo{User must specify\\
(1) the window function\\
(2) basis functions\\
(3) order of completeness for polynomial basis functions}

\subsection{Solid deformation}

\subsubsection{Output variables}
The values that can be written to output as nodal or element 
quantities are listed in Tables~\ref{tab.solid.output.node} 
and~\ref{tab.solid.output.element}, respectively. The element output 
of integration point values have the labels
\[
\texttt{ip}N\texttt{.}\sbrkt{\textit{label}} \qquad
N=1,\ldots,n_{\textit{ip}},
\]
where $n_{\textit{ip}}$ is the number of element integration points.

\begin{table}[h]
\caption{\label{tab.solid.output.node} Nodal output values for solid 
deformation.}
\begin{center}
\begin{tabular}[c]{|c|c|c|c|}
\hline
% table headings
 \parbox[b]{0.25in}{\centering  \textbf{\#}}
&\parbox[b]{0.5in}{\centering   \textbf{var.}}
&\parbox[b]{2.5in}{\raggedright \textbf{description}}
&\parbox[b]{2.5in}{\raggedright  \textbf{labels}}\\
\hline
1 & $\mathbf{X}$ 
  & \parbox[b]{2.5in}{\raggedright reference coordinates} 
  & \parbox[b]{2.5in}{\raggedright 
  $\texttt{x}\sbrkt{i} \quad i = 0, \ldots, n_{\textit{sd}}$ }\\
\hline
2 & $\mathbf{d}$ 
  & \parbox[b]{2.5in}{\raggedright displacements} 
  & \parbox[b]{2.5in}{\raggedright 
  $ \texttt{D\_}\sbrkt{i} \quad i = \texttt{X}, \texttt{Y}, \ldots, 
  n_{\textit{dof}}$ }\\
\hline
3 & $\boldsymbol{\sigma}$ 
  & \parbox[c]{2.5in}{\raggedright extrapolated stresses} 
  & \parbox[c]{2.5in}{\raggedright \vspace{0.025in}
  2D: \texttt{s11}, \texttt{s22}, \texttt{s12} \\
  3D: \texttt{s11}, \texttt{s22}, \texttt{s33}, 
      \texttt{s23}, \texttt{s13}, \texttt{s12}}\\
\hline
4 & $\hat{\boldsymbol{\sigma}}$ 
  & \parbox[b]{2.5in}{\raggedright extrapolated principal stresses} 
  & \parbox[b]{2.5in}{\raggedright
  $\texttt{s}\sbrkt{i} \quad i = 0, \ldots, n_{\textit{sd}}$ }\\
\hline
5 & $\phi$ 
  & \parbox[c]{2.5in}{\raggedright extrapolated strain energy density} 
  & \parbox[c]{2.5in}{\raggedright \texttt{phi}} \\
\hline
6 & $c_{d}$, $c_{s}$ 
  & \parbox[c]{2.5in}{\raggedright extrapolated acoustical wave speeds} 
  & \parbox[c]{2.5in}{\raggedright \vspace{0.025in}
  2D: \texttt{cd}, \texttt{cs}\\
  3D: \texttt{cd}, \texttt{cs\_min}, \texttt{cs\_max}} \\
\hline
7 &  
  & \parbox[c]{2.5in}{\raggedright extrapolated material model output} 
  & \parbox[c]{2.5in}{\raggedright determined by model} \\
\hline
\end{tabular}
\end{center}
\end{table}

\begin{table}[h]
\caption{\label{tab.solid.output.element} Element output values for 
solid deformation.}
\begin{center}
\begin{tabular}[c]{|c|c|c|c|}
\hline
% table headings
 \parbox[b]{0.25in}{\centering  \textbf{\#}}
&\parbox[b]{0.5in}{\centering   \textbf{var.}}
&\parbox[b]{2.5in}{\raggedright \textbf{description}}
&\parbox[b]{2.5in}{\raggedright  \textbf{labels}}\\
\hline
1 & $\overline{\mathbf{X}}$ 
  & \parbox[b]{2.5in}{\raggedright reference centroid coordinates} 
  & \parbox[b]{2.5in}{\raggedright 
  $\texttt{xc\_}\sbrkt{i} \quad i = 0, \ldots, n_{\textit{sd}}$ }\\
\hline
2 & $m$ 
  & \parbox[b]{2.5in}{\raggedright integrated mass} 
  & \parbox[b]{2.5in}{\raggedright \texttt{mass} }\\
\hline
3 & $\phi$ 
  & \parbox[c]{2.5in}{\raggedright integrated strain energy} 
  & \parbox[c]{2.5in}{\raggedright \texttt{U}}\\
\hline
4 & $T$ 
  & \parbox[c]{2.5in}{\raggedright integrated kinetic energy} 
  & \parbox[c]{2.5in}{\raggedright \texttt{T}}\\
\hline
5 & $\mathbf{L}$ 
  & \parbox[c]{2.5in}{\raggedright integrated linear momentum} 
  & \parbox[c]{2.5in}{\raggedright 
  $\texttt{L\_}\sbrkt{i} \quad i=\texttt{X}, \texttt{Y}, \ldots, 
  n_{\textit{dof}}$} \\
\hline
6 & $\boldsymbol{\sigma}$ 
  & \parbox[c]{2.5in}{\raggedright integration point stresses} 
  & \parbox[c]{2.5in}{\raggedright \vspace{0.025in}
  2D: \texttt{s11}, \texttt{s22}, \texttt{s12} \\
  3D: \texttt{s11}, \texttt{s22}, \texttt{s33}, 
      \texttt{s23}, \texttt{s13}, \texttt{s12}}\\
\hline
7 &  
  & \parbox[c]{2.5in}{\raggedright integration point material output} 
  & \parbox[c]{2.5in}{\raggedright determined by model} \\
\hline
\end{tabular}
\end{center}
\end{table}

\subsubsection{Constitutive models}
The materials available for analysis of solid deformation are listed 
in Table~\ref{tab.mat.solid}.

\subsection{Diffusion}

\subsubsection{Output variables}

\subsubsection{Constitutive models}

\subsection{Cohesive surface elements}
\label{section.CSE.element}
At every point along a cohesive surface, we define a local
coordinate frame which resolves the opening displacements across
the surface into normal and shear components. As shown in 
Figure~\ref{fig.CSE}, this coordinate frame is not uniquely
defined for finite deformations.
\begin{figure}[h]
\centerline{\includegraphics[scale = 1.0]
{\dirfilepath{\figpath}{CSE_local.eps}}}
\hangcaption
[Local coordinate frame]
{Local coordinate frame $\tilde{\mathbf{e}}_{i}$ constructed
 along the surface where $\mathbf{e}_{i}$ is the global
 cartesian frame.\label{fig.CSE}}
\end{figure}
We choose to define this coordinate system with respect to
the mid-plane between the displaced surfaces. Alternately, this
frame may be fixed to the upper or lower facet.
The discussion that follows is independent of the particular choice
of local coordinate system though we present specific expressions for
the mid-plane. Let the transformation of the gap 
vector 
from its representation in the global coordinate
frame  $\boldsymbol{\Delta}$ 
to its local representation $\tilde{\boldsymbol{\Delta}}$ 
in a frame defined with respect to the cohesive surface be
given by
\labeleq{eq.CSE.1}{
\tilde{\boldsymbol\Delta} = \mathbf{Q}^{\text{T}}\,\boldsymbol{\Delta}.
}
The transformation resolves the opening displacement into
normal and shear components relative to the current orientation
of the surface.
This distinction allows for the definition of mixed-mode, surface 
constitutive relations that show a dependence on the character of the 
opening displacement.
In a variational setting, the contribution to the virtual work from
cohesive surface elements is
\labeleq{eq.CSE.2}{
\delta w^{\text{CSE}} = \int\limits_{\Gamma_0} \mathbf{T} \cdot 
    \delta \boldsymbol{\Delta}
    \,d\Gamma,
}
where the integration domain $\Gamma_{0}$ is defined in the 
undeformed configuration and $\delta \boldsymbol{\Delta}$ is the 
variation in the surface opening displacement. We review the
formulation of several surface constitutive models in 
Section~\ref{section.CSE.laws}. The discussion that follows
applies to all displacement-driven cohesive relations
$\tilde{\mathbf{T}}(\tilde{\boldsymbol{\Delta}})$
defined in the local surface frame.
From~\eqref{eq.CSE.1}, the traction in the global coordinate frame
is given by the transformation
\labeleq{eq.CSE.4}{
\mathbf{T} = \mathbf{Q}\,\tilde{\mathbf{T}}.
}

\subsubsection{Output variables}
Nodal and element output variables are prescribed in the input file 
by specifying a \texttt{1} for variables to be written and a \texttt{0}
for variables not to be written. The sequence and definition of the 
nodal and element output variables are described in the next sections.

\paragraph{Nodal output variables}
The output variables written by cohesive surface elements are listed 
in Tables~\ref{tab.CSE.iso.node.output} and 
Table~\ref{tab.CSE.aniso.node.output}.
Output variables defined at the integration points are projected to 
the nodes using the algorithm proposed by Zinkiewickz \etal~\cite{Zinkiewickz}.

The isotropic and anisotropic 
element write the same variables; however, some of the variables 
returned as scalars for the isotropic formulation are returned as 
vectors for the anisotropic case.
\begin{table}[h]
\caption{\label{tab.CSE.iso.node.output} Nodal output variables for isotropic 
cohesive surface elements.}
\begin{center}
\begin{tabular}[c]{|c|c|c|c|}
\hline
% table headings
 \parbox[b]{0.50in}{\centering  \textbf{\#}}
&\parbox[b]{0.5in}{\centering   \textbf{var.}}
&\parbox[b]{2.5in}{\raggedright \textbf{description}}
&\parbox[b]{2.0in}{\raggedright  \textbf{labels}}\\
\hline
1 & $\mathbf{X}$ 
  & \parbox[b]{2.5in}{\raggedright reference coordinates} 
  & \parbox[b]{2.0in}{\raggedright 
  $\texttt{x}\sbrkt{i} \quad i = 0, \ldots, n_{\textit{sd}}$  }\\
\hline
2 & $\mathbf{d}$ 
  & \parbox[b]{2.5in}{\raggedright displacements} 
  & \parbox[b]{2.0in}{\raggedright 
  $\texttt{D\_}\sbrkt{i} \quad
  i = \texttt{X}, \texttt{Y}, \ldots, n_{\textit{dof}}$  }\\
\hline
3 & $||\Delta||$ 
  & \parbox[b]{2.5in}{\raggedright opening displacements} 
  & \parbox[b]{2.0in}{\texttt{jump}}\\
\hline
4 & $||\mathbf{T}||$ 
  & \parbox[b]{2.5in}{\raggedright traction} 
  & \parbox[b]{2.0in}{\raggedright \texttt{Tmag}}\\
\hline
5 &  
  & \parbox[b]{2.5in}{\raggedright constitutive output variables} 
  & \parbox[b]{2.0in}{\raggedright determined by model} \\
\hline
\end{tabular}
\end{center}
\end{table}

\begin{table}[h]
\caption{\label{tab.CSE.aniso.node.output} Nodal output variables for anisotropic 
cohesive surface elements.}
\begin{center}
\begin{tabular}[c]{|c|c|c|c|}
\hline
% table headings
 \parbox[b]{0.50in}{\centering  \textbf{\#}}
&\parbox[b]{0.5in}{\centering   \textbf{var.}}
&\parbox[b]{2.5in}{\raggedright \textbf{description}}
&\parbox[b]{2.0in}{\raggedright  \textbf{labels}}\\
\hline
1 & $\mathbf{X}$ 
  & \parbox[b]{2.5in}{\raggedright reference coordinates} 
  & \parbox[b]{2.0in}{\raggedright 
  $\texttt{x}\sbrkt{i}\quad i = 0, \ldots, n_{\textit{sd}}$  }\\
\hline
2 & $\mathbf{d}$ 
  & \parbox[b]{2.5in}{\raggedright displacements} 
  & \parbox[b]{2.0in}{\raggedright 
  $\texttt{D\_}\sbrkt{i} i = \texttt{X}, \texttt{Y}, \ldots, n_{\textit{dof}}$  }\\
\hline
3 & $\tilde{\mathbf{\Delta}}$ 
  & \parbox[b]{2.5in}{\raggedright local opening displacements} 
  & \parbox[c]{2.0in}{\raggedright
      \vspace{0.025in}
      2D: \texttt{d\_t}, \texttt{d\_n}\\
	  3D: \texttt{d\_t1}, \texttt{d\_t2}, \texttt{d\_n}}\\
\hline
4 & $\tilde{\mathbf{T}}$ 
  & \parbox[c]{2.5in}{\raggedright local traction} 
  & \parbox[c]{2.0in}{\raggedright
      \vspace{0.025in}
      2D: \texttt{T\_t}, \texttt{T\_n}\\
	  3D: \texttt{T\_t1}, \texttt{T\_t2}, \texttt{T\_n}}\\
\hline
5 &  
  & \parbox[b]{2.5in}{\raggedright constitutive output variables} 
  & \parbox[b]{2.0in}{\raggedright determined by model} \\
\hline
\end{tabular}
\end{center}
\end{table}

Note that the normal component for the output of the opening 
displacement and the traction is always the last value. The tangent 
directions, denoted with the \texttt{t1} and \texttt{t2}, are 
determined by the coordinate mapping over surface element and 
generally vary from element to element.

\paragraph{Element output variables}
The element output variables for both isotropic and anisotropic 
cohesive surface elements are listed in 
Table~\ref{tab.CSE.element.output}.
\begin{table}[h]
\caption{\label{tab.CSE.element.output} Element output variables for
cohesive surface elements.}
\begin{center}
\begin{tabular}[c]{|c|c|c|c|}
\hline
% table headings
 \parbox[b]{0.50in}{\centering  \textbf{\#}}
&\parbox[b]{0.5in}{\centering   \textbf{var.}}
&\parbox[b]{2.5in}{\raggedright \textbf{description}}
&\parbox[b]{2.0in}{\raggedright  \textbf{labels}}\\
\hline
1 & $\overline{\mathbf{X}}$ 
  & \parbox[b]{2.5in}{\raggedright centroid in reference coordinates} 
  & \parbox[b]{2.0in}{\raggedright
  $\texttt{xc\_}\sbrkt{i} \quad i = 0, \ldots, n_{\textit{sd}}$  }\\
\hline
2 & $\phi$ 
  & \parbox[b]{2.5in}{\raggedright integrated cohesive energy} 
  & \parbox[b]{2.0in}{\raggedright \texttt{phi}}\\
\hline
3 & $\overline{t}$ 
  & \parbox[c]{2.5in}{\raggedright average traction} 
  & \parbox[c]{2.0in}{\raggedright   
      \vspace{0.025in}
      2D: \texttt{T\_t}, \texttt{T\_n}\\
	  3D: \texttt{T\_t1}, \texttt{T\_t2}, \texttt{T\_n}}\\
\hline
\end{tabular}
\end{center}
\end{table}
Note that some of the variables are element-averaged quantities, 
while others are integrated over the element.

\subsubsection{Constitutive models}
\label{section.CSE.laws}
The element formulation presented in Section~\ref{section.CSE.element}
makes use of a local coordinate frame defined with respect the 
deforming facets of the cohesive surface. This construction allows 
all surface constitutive relations to define the evolution of
the local traction $\tilde{\mathbf{T}}$ 
with respect to the local opening displacement
$\tilde{\boldsymbol{\Delta}}$, 
while the kinematics of large deformation and surface rotations are 
accounted for at the element level.
For the case of an elastic, or reversible, cohesive relation, the 
traction is derived from a free energy
potential $\varphi$ as
\labeleq{eq.CSE.10a}{
\tilde{\mathbf{T}}=\frac{\partial \varphi}{\partial 
\tilde{\boldsymbol{\Delta}}} 
}
while the surface stiffness may be expressed as
\labeleq{eq.CSE.10b}{
\frac{\partial \tilde{\mathbf{T}}}{\partial \tilde{\boldsymbol{\Delta}}} 
= \frac{\partial^{2} \varphi}{\partial \tilde{\boldsymbol{\Delta}}
\partial \tilde{\boldsymbol{\Delta}}}.
}
%% 
 % The use of $\varphi$ in here differs from its definition in 
 % Section~\ref{section.kinematics}. Throughout this section we will
 % use $\varphi$ to denote a cohesive potential.
 % Using~\eqref{eq.CSE.10b}, we see that the first term contributing to 
 % the element stiffness~\eqref{eq.CSE.10} is symmetric, while the remaining 
 % terms are generally nonsymmetric.
 %%

Motivated by the universal binding energy 
curves of Rose \etal\cite{Rose1981} from atomistic modeling,
Xu and Needleman\cite{Needleman1994} proposed the mixed-mode cohesive potential
\begin{equation}
\label{eq.XN.1}    
\varphi\farg{\tilde{\boldsymbol{\Delta}}}=
    \phi_{n} +
    \phi_{n}\,\exp\farg{-\frac{\tilde{\Delta}_{n}}{\delta_{n}}}\,
    \rbrkt{
        \sbrkt{1 - r + \frac{\tilde{\Delta}_{n}}{\delta_{n}}}\,
        \frac{1 - q}{r - 1} -
	\exp\farg{-\frac{\tilde{\Delta}_{t}^{2}}{\delta_{t}^{2}}}\,
	\sbrkt{q + \rbrkt{\frac{r - q}{r - 1}}\,
	\frac{\tilde{\Delta}_{n}}{\delta_{n}}} 
    }.
\end{equation}
From~\eqref{eq.CSE.10a}, the components of the traction
are
\begin{equation}
\label{eq.XN.2}    
\tilde{T}_{n}\farg{\tilde{\boldsymbol{\Delta}}}=
    \frac{\phi_{n}}{\delta_{n}}\,
    \exp\farg{-\frac{\tilde{\Delta}_{n}}{\delta_{n}}}\,
    \rbrkt{
        \frac{\tilde{\Delta}_{n}}{\delta_{n}}\,
	\exp\farg{-\frac{\tilde{\Delta}_{t}^{2}}{\delta_{t}^{2}}}+
        \frac{1 - q}{r - 1}\,
	\sbrkt{
	    1 - \exp\farg{-\frac{\tilde{\Delta}_{t}^{2}}{\delta_{t}^{2}}}
	}\,
	\sbrkt{r - \frac{\tilde{\Delta}_{n}}{\delta_{n}}}
    }
\end{equation}
and
\begin{equation}
\label{eq.XN.3}    
\tilde{T}_{t}\farg{\tilde{\boldsymbol{\Delta}}}=
    2\,\phi_{n}\,\frac{\tilde{\Delta}_{t}}{\delta_{t}^{2}}\,
    \rbrkt{q + \sbrkt{\frac{r - q}{r - 1}}\,
    \frac{\tilde{\Delta}_{n}}{\delta_{n}}}\,
    \exp\farg{-\frac{\tilde{\Delta}_{n}}{\delta_{n}}}\,
    \exp\farg{-\frac{\tilde{\Delta}_{t}^{2}}{\delta_{t}^{2}}}.
\end{equation}
In \eqref{eq.XN.1}--\eqref{eq.XN.3},
the ratio $q=\frac{\phi_{t}}{\phi_{n}}$ relates the pure normal 
mode to shear mode fracture energy and 
$r=\frac{\tilde{\Delta}^{*}_{n}}{\delta_{n}}$ defines
$\tilde{\Delta}^{*}_{n}$, the value of $\tilde{\Delta}_{n}$ after 
complete shear separation with $\tilde{T}_{n} = 0$.
The effect of $r$ vanishes when $q=1$, which is the
value used for all calculations in this study.
The normal and tangential openings are defined as
$\tilde{\Delta}_{n}=\tilde{\boldsymbol{\Delta}}\cdot
\tilde{\mathbf{n}}$ and
$\tilde{\Delta}_{n}=\tilde{\boldsymbol{\Delta}}\cdot
\tilde{\mathbf{t}}$ with respect to the local normal and 
tangent. Xu and Needleman developed this cohesive relation
in two dimensions. We extend this relation to three dimensions by 
assuming the response to be isotropic with respect to the tangential
opening. That is, the response is expressed only in terms of the
magnitude of the tangential opening, which can also be written as
\begin{equation}
\tilde{\Delta}_{t}=\left|    
   \tilde{\boldsymbol{\Delta}}-
   \tilde{\Delta}_{n}\,\tilde{\mathbf{n}}
   \right|,
\end{equation}    
to emphasize that it is independent of $\tilde{\mathbf{t}}$.
For the three-dimensional model, we did not attempt to incorporate any 
additional effects from the relative rotation of the surfaces about
the normal.

\subsection{Cohesive surface elements for meshfree domains}
\label{section.element.CSE.MF}

\subsubsection{Constitutive models}
\subsubsection{Parameters for adaptive insertion}

\subsection{Contact}
In the input section for contact elements, the user
specifies a regularization constant and the entities involved with 
contact. Contact entities are specified in two sections, contact 
surfaces and contact, or striker, nodes.
\subsubsection{Penalty formulation}
With this contact formulation, the penetration of a striker node into a 
contact surface is penalized with a quadratic potential as
\begin{equation}
U_{c}= \frac{1}{2}\,k_{c}\,g^{2},
\end{equation}
where $k_{c}$ is the contact stiffness, or regularization parameter,
and $g$ is the penetration 
distance. When used in conjunction with cohesive surface elements, a 
reasonable choice for the contact stiffness is
\begin{equation}
k_{c}=k_{\text{CSE}}\,A,
\end{equation}
where $A$ is the area of the cohesive surface element, and 
$k_{\text{CSE}}$ is the tangent stiffness of the surface 
constitutive relation. For the potential proposed by 
Xu-Needleman potential~\ref{eq.XN.1},
this stiffness is
\begin{equation}
k_{\text{CSE}} = \frac{\phi}{\delta_{n}^{2}}.
\end{equation}

\subsubsection{Augmented Lagrangian formulation}
The augmented Lagrangian formulation in \tahoe is based on the 
formulation by Heegaard and Curnier~\cite{Heegaard1993}.

\subsection{Pair-potential}

\subsection{Two- and three-body potential}

