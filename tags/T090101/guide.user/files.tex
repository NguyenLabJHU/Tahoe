%&LaTeX
% $Id: files.tex,v 1.4 2001-04-23 05:07:41 paklein Exp $

\section{Geometry and results files}
\label{sect.files}
The user input consists of a analysis parameters
and a geometry specification. 
Analysis parameters are specified in a plain text file. Several 
options exist for the specification of the geometry.
Table~\ref{tab.input.file.type} lists the
file types available to specify the geometry.
These will be described in greater detail in 
Section~\ref{sect.file.geometry}.
Table~\ref{tab.output.file.type} lists the
file types available for the output of results.
Note that some of the file types may be used for both
input and results, while others are limited to one or the other.
\begin{table}[h]
\caption{\label{tab.input.file.type} Geometry input file formats.}
\begin{center}
\begin{tabular}[c]{|c|c|}
\hline
 \parbox[b]{0.75in}{\centering \textbf{code}}
&\parbox[b]{2.0in}{\raggedright \textbf{description}}\\
\hline
\parbox[b]{0.75in}{\centering 0} & 
\parbox[b]{2.0in}{\raggedright TahoeI}\\
\hline
\parbox[b]{0.75in}{\centering 1}  & 
\parbox[b]{2.0in}{\raggedright TahoeII}\\
\hline
\parbox[b]{0.75in}{\centering 5}  & 
\parbox[b]{2.0in}{\raggedright \textsf{ExodusII}~\cite{ExodusII}}\\
\hline
\end{tabular}
\end{center}
\end{table}
\begin{table}[h]
\caption{\label{tab.output.file.type} Result output file formats.}
\begin{center}
\begin{tabular}[c]{|c|c|}
\hline
 \parbox[c]{0.75in}{\centering \textbf{code}}
&\parbox[c]{2.0in}{\raggedright \textbf{description}}\\
\hline
\parbox[c]{0.75in}{\centering 0} & 
\parbox[c]{2.0in}{\raggedright TahoeI}\\
\hline
\parbox[c]{0.75in}{\centering 2}  & 
\parbox[c]{2.0in}{\raggedright \textsf{TecPlot}~\cite{TecPlot8} version 7}\\
\hline
\parbox[c]{0.75in}{\centering 3}  & 
\parbox[c]{2.0in}{\raggedright \vspace{2pt} \textsf{EnSight}~\cite{EnSight6} Gold ASCII 
version 6 \vspace{2pt}}\\
\hline
\parbox[c]{0.75in}{\centering 4}  & 
\parbox[c]{2.0in}{\raggedright \vspace{2pt} \textsf{EnSight} Gold 
Binary version 6 \vspace{2pt}}\\
\hline
\parbox[c]{0.75in}{\centering 5}  & 
\parbox[c]{2.0in}{\raggedright \textsf{ExodusII}~\cite{ExodusII}}\\
\hline
\end{tabular}
\end{center}
\end{table}

\subsection{Geometry file formats}
\label{sect.file.geometry}
Input data is allowed to be one of three database formats listed in
Table~\ref{tab.input.file.type}: 
\begin{itemize}
\item[(1)] TahoeI: With this format, geometry data may be supplied 
embedded in the parameters file or in files separate from, or external 
to, the parameters files. When specified in external files, the block 
of geometry data in the parameters file is replaced by the name of 
the external file where the data resides. The format of the data 
itself is the same in both cases.
\item[(2)] TahoeII: This format allows random access to the geometry 
data which is stored in one or more plain text files.
\item[(3)] \textsf{ExodusII}: 
This format allows random access to the geometry 
data which is stored in platform independent binary file.
\end{itemize}
All \tahoe plain text input files are white space delineated, that is, 
values are separated by white space, but the arrangement of white space
is ignored.
The \texttt{\#} symbol denotes that the remainder of a given line of text 
should be ignored, or treated as a comment.

\subsubsection{TahoeI format}
\label{sect.file.tahoeI}
The TahoeI format includes specifications for model geometry as well 
as kinematic and force boundary conditions. Each of these input types 
is described in greater detail in the sections that follow. All of the 
input types precede the actual input data with some indication of the 
dimensions of the data. When beginning to read a block of input data,
\tahoe will determine whether the block begins with an integer or
with a character string.
A leading integer denotes the dimensions of an embedded block of
input data. If \tahoe finds a character string, it assumes the string 
is the name of an external file in which the input data resides. 
\tahoe cannot distinguish file names beginning with an integer from 
the dimensions of an embedded data block.

\paragraph{Coordinate list}
\label{sect.TahoeI.coordinates}
The format of coordinate data is shown below:
\begin{align}
& \texttt{\# dimensions} \notag \\
& \sbrkt{n_{nd} = \textit{number of points}} \notag \\
& \sbrkt{n_{sd} = \textit{number of spatial dimensions}} \notag \\
& \texttt{\# coordinates} \notag \\
& \begin{matrix}
\texttt{1} & \sbrkt{x_{11}} & \dots  & \sbrkt{x_{1n_{sd}}} \\
\vdots \\
\sbrkt{n_{nd}}  & \sbrkt{x_{n_{nd}1}} & \dots & \sbrkt{x_{n_{nd}n_{sd}}}
\end{matrix} \notag
\end{align}
The coordinate of each point is preceded by its index. Coordinates 
may appear in any order; however, \tahoe does not verify that all 
$n_{nd}$ points have been assigned coordinates exactly once.

\paragraph{Element block}
\label{sect.TahoeI.elements}
The format of element block data is shown below:
\begin{align}
& \texttt{\# dimensions} \notag \\
& \sbrkt{n_{el} = \textit{number of elements}} \notag \\
& \sbrkt{n_{en} = \textit{number of nodes per element}} \notag \\
& \texttt{\# connectivities} \notag \\
& \begin{matrix}
\texttt{1} & \sbrkt{n_{11}} & \dots  & \sbrkt{n_{1n_{en}}} \\
\vdots \\
\sbrkt{n_{el}} & \sbrkt{n_{n_{el}1}} & \dots & \sbrkt{n_{n_{el}n_{en}}}
\end{matrix} \notag
\end{align}
The connectivity of each element is preceded by its index. 
Connectivities may appear in any order; 
however, \tahoe does not verify that all $n_{el}$ 
elements have been assigned connectivities exactly once.

\paragraph{Node set}
\label{sect.TahoeI.nodeset}
The format of node set data is shown below:
\begin{align}
& \texttt{\# dimension} \notag \\
& \sbrkt{n_{nd} = \textit{number of nodes}} \notag \\
& \texttt{\# list} \notag \\
& \begin{matrix}
\sbrkt{n_{1}} & \dots  & \sbrkt{n_{n_{nd}}} \\
\end{matrix} \notag
\end{align}

\paragraph{Side set}
\label{sect.TahoeI.sideset}
The format of side set data is shown below:
\begin{align}
& \texttt{\# dimension} \notag \\
& \sbrkt{n_{s} = \textit{number of facets}} \notag \\
& \texttt{\# facet list} \notag \\
& \begin{matrix}
\sbrkt{e_{1}} & \sbrkt{s_{1}} \\
\vdots \\
\sbrkt{e_{n_{s}}} & \sbrkt{s_{n_{s}}}
\end{matrix} \notag
\end{align}
Facets are specified by two numbers, the element number $e$ and the
side $s$. The side numbering convention for the various integration 
domain geometries are shown in Figures~\ref{fig.line.element} 
through~\ref{fig.pent.element}.

\subsubsection{TahoeII format}
\label{sect.file.tahoeII}
The TahoeII format allows random access to geometry data in a plain 
text file format. The file format uses the keywords listed
in Table~\ref{tab.TahoeII.keywords}. The keywords must preceded by an
asterisk (\texttt{*}).
\begin{table}[h]
\caption{\label{tab.TahoeII.keywords} Keywords for the TahoeII file format.}
\begin{center}
\begin{tabular}[c]{cc}
\parbox[b]{2.0in}{\centering \texttt{version}} &
\parbox[b]{2.0in}{\centering \texttt{title}}\\
\parbox[b]{2.0in}{\centering \texttt{dimensions}} & 
\parbox[b]{2.0in}{\centering \texttt{nodes}}\\
\parbox[b]{2.0in}{\centering \texttt{elements}}  & 
\parbox[b]{2.0in}{\centering \texttt{nodesets}}\\
\parbox[b]{2.0in}{\centering \texttt{sidesets}}  & 
\parbox[b]{2.0in}{\centering \texttt{set}}\\
\end{tabular}
\end{center}
\end{table}
Aside from the keyword \texttt{set}, each keyword denotes a separate 
type of data. These keywords may appear in any order in a TahoeII file.
However, since the file is scanned whenever data is requested, it is 
more efficient to place longer blocks of data toward the end of the 
file. Alternately, blocks of data may be specified in files separate 
from the primary TahoeII file.
When specified in an externally, the block 
of geometry data in the file is replaced by the path of 
the external file where the data resides. The format of the data 
itself is the same in both cases.

A TahoeII file contains a single set of coordinates and an arbitrary 
number of element sets, node sets, and side sets. 
The format of the sets is similar to the TahoeI format described in 
Section~\ref{sect.file.tahoeI}.
The sets are 
identified by an ID. ID's may be any positive integer and do not need 
to be consecutive. ID's for each type of set are stored separately.
As mentioned above, TahoeII files are plain text and
white space delineated, that is, 
values are separated by white space, but the arrangement of white space
is ignored.
The \texttt{\#} symbol denotes that the remainder of a given line of text 
should be ignored, or treated as a comment.
The current TahoeII version is \texttt{1.0}. The format of each type of
data in a TahoeII file is described in the sections that follow.

\paragraph{Version}
The \texttt{version} keyword is used to check the compatibility of the 
database with the current version of the TahoeII code. 
The current version is \texttt{1.0}.
An example of 
this section is shown below:
\begin{align}
& \texttt{\# TahoeII file version} \notag \\
& \texttt{*version 1.0} \notag
\end{align}

\paragraph{Title}
This section begins with the \texttt{version} keyword.
A one line title may be assigned to the database:
\begin{align}
& \texttt{\# database title} \notag \\
& \texttt{*title} \notag \\
& \sbrkt{\textrm{\textit{one line title of 254 characters or less}}}
\end{align}
\paragraph{Dimensions}
\label{section.TahoeII.dimensions}
This section begins with the \texttt{dimensions} keyword.
It contains the dimensions of the nodal coordinate data, and
declares the ID and dimension of each element, node, and side set.
This sequence of data in this 
section must follow the order shown in the example below:
\begin{align}
& \texttt{\# dimensions and IDs} \notag \\\
& \texttt{*dimensions} \notag \\
& \texttt{\# dimension of coordinate data} \notag \\
& \sbrkt{n_{nd}=\textit{number of nodes}} \notag \\
& \sbrkt{n_{sd}=\textit{number of spatial dimensions}} \notag \\
& \texttt{\# element set declarations} \notag \\
& \sbrkt{n_{es}=\textit{number of element sets}} \notag \\
& \begin{matrix}
\sbrkt{\textit{ID}_{1}} & 
\sbrkt{n_{el1} = \textit{number of elements}} & 
\sbrkt{n_{en1} = \textit{number of element nodes}} \\
\vdots \\
\sbrkt{\textit{ID}_{n_{es}}} & 
\sbrkt{n_{el\,n_{es}}} & \sbrkt{n_{en\,n_{es}}}
\end{matrix} \notag \\
& \texttt{\# node set declarations} \notag \\
& \sbrkt{n_{ns} = \textit{number of node sets}} \notag \\
& \begin{matrix}
\sbrkt{\textit{ID}_{1}} & \sbrkt{n_{nd1} = \textit{number of set nodes}} \\
\vdots \\
\sbrkt{\textit{ID}_{n_{ns}}} & \sbrkt{n_{nd\,n_{ns}}}
\end{matrix} \notag \\
& \texttt{\# side set declarations} \notag \\
& \sbrkt{n_{ss} = \textit{number of side sets}} \notag \\
& \begin{matrix}
\sbrkt{\textit{ID}_{1}} & \sbrkt{n_{s1} = \textit{number of facets}} \\
\vdots \\
\sbrkt{\textit{ID}_{n_{ss}}} & \sbrkt{n_{s\,n_{ss}}}
\end{matrix} \notag
\end{align}
If the number of element sets $n_{es}$, node sets $n_{ns}$, or side sets 
$n_{ss}$ is zero, the subsequent list of ID's and dimensions is empty. 

\paragraph{Node sets}
This section begins with the \texttt{nodesets} keyword.
The node sets must appear in the order they have been specified in 
the dimensions section~\ref{section.TahoeII.dimensions}. The 
beginning of the data for each node set is marked by the \texttt{set} 
keyword. The data for each set is comprised of the number of nodes in 
the set and list of nodes themselves. This data for each set may be 
defined in line or in an external file. If the data is defined in an 
external file, the path to the file replaces the data itself in the 
TahoeII file. In either case, the format of the data itself is the 
same. The format of each node set specification is the same as the TahoeI 
format for node sets~\ref{sect.TahoeI.nodeset}. An example appears 
below:
\begin{align}
& \texttt{\# node set specifications} \notag \\
& \texttt{*nodesets} \notag \\
& \texttt{*set} \notag \\
& \sbrkt{n_{nd1} = \textit{number of nodes in set~}1} \notag \\
& \begin{matrix}
\sbrkt{n_{1}} & \dots  & \sbrkt{n_{n_{nd1}}} \\
\end{matrix} \notag \\
& \vdots \notag \\
& \texttt{*set} \notag \\
& \sbrkt{n_{nd\,n_{ns}} = \textit{number of nodes in set~}n_{ns}} \notag \\
& \begin{matrix}
\sbrkt{n_{1}} & \dots  & \sbrkt{n_{n_{nd\,n_{ns}}}} \\
\end{matrix} \notag
\end{align}
The set subscripts for the nodes in each set have been omitted for 
clarity.
Note that the dimensions of the set are from the from the dimensions 
section are repeated but that the ID of each set is not repeated.
When specifying the data for a node set in an external file, the 
primary TahoeII contains the path to the external file, as shown below:
\[
\texttt{*set~} \sbrkt{\textrm{\textit{path to external file}}}
\]

\paragraph{Side sets}
This section begins with the \texttt{sidesets} keyword.
The side sets must appear in the order they have been specified in 
the dimensions section~\ref{section.TahoeII.dimensions}. The 
beginning of the data for each side set is marked by the \texttt{set} 
keyword. The data for each set is comprised of the number of each in 
the set and list of facet specifications themselves. 
Facets are specified by two numbers, the element number $e$ and the
side $s$. The side numbering convention for the various integration 
domain geometries are shown in Figures~\ref{fig.line.element} 
through~\ref{fig.pent.element}.
This data for each set may be 
defined in line or in an external file. If the data is defined in an 
external file, the path to the file replaces the data itself in the 
TahoeII file. In either case, the format of the data itself is the 
same. The format of each side set specification is the same as the TahoeI 
format for side sets~\ref{sect.TahoeI.sideset}. An example appears 
below:
\begin{align}
& \texttt{\# side set specifications} \notag \\
& \texttt{*sidesets} \notag \\
& \texttt{*set} \notag \\
& \sbrkt{n_{s1} = \textit{number of facets in set~}1} \notag \\
& \begin{matrix}
\sbrkt{e_{1}} & \dots  & \sbrkt{s_{1}} \\
\vdots \\
\sbrkt{e_{n_{s1}}} & \dots  & \sbrkt{s_{n_{s1}}}
\end{matrix} \notag \\
& \vdots \notag \\
& \texttt{*set} \notag \\
& \sbrkt{n_{s\,n_{ss}} = \textit{number of facets in set~}n_{ss}} \notag \\
& \begin{matrix}
\sbrkt{e_{1}} & \dots  & \sbrkt{s_{1}} \\
\vdots \\
\sbrkt{e_{n_{s\,n_{ss}}}} & \dots  & \sbrkt{s_{n_{s\,n_{ss}}}}
\end{matrix} \notag
\end{align}
The set subscripts for the element and sides in each set have been
omitted for clarity.
Note that the dimensions of the set are from the from the dimensions 
section are repeated but that the ID of each set is not repeated.
When specifying the data for a side set in an external file, the 
primary TahoeII contains the path to the external file, as shown below:
\begin{align}
\texttt{*set~} \sbrkt{\textrm{\textit{path to external file}}}
\end{align}

\paragraph{Element sets}
This section begins with the \texttt{elements} keyword.
The element sets must appear in the order they have been specified in 
the dimensions section~\ref{section.TahoeII.dimensions}. The 
beginning of the data for each element set is marked by the \texttt{set} 
keyword. The data for each set is comprised of the number of each in 
the set and list of facet specifications themselves. 
This data for each set may be 
defined in line or in an external file. If the data is defined in an 
external file, the path to the file replaces the data itself in the 
TahoeII file. In either case, the format of the data itself is the 
same. The format of each element set specification is the same as the TahoeI 
format for element sets~\ref{sect.TahoeI.elements}. An example appears 
below:
\begin{align}
& \texttt{\# element set specifications} \notag \\
& \texttt{*elements} \notag \\
& \texttt{*set} \notag \\
& \sbrkt{n_{el1} = \textit{number of elements in set~}1} \notag \\
& \begin{matrix}
\texttt{1} & \sbrkt{n_{11}} & \dots  & \sbrkt{n_{1n_{en1}}} \\
\vdots \\
\sbrkt{n_{el1}} & \sbrkt{n_{n_{el1}1}} & \dots  & \sbrkt{n_{n_{el1}n_{en1}}}
\end{matrix} \notag \\
& \vdots \notag \\
& \texttt{*set} \notag \\
& \sbrkt{n_{el\,n_{es}} = \textit{number of elements in set~}n_{es}} \notag \\
& \begin{matrix}
\texttt{1} & \sbrkt{n_{11}} & \dots  & \sbrkt{n_{1n_{en\,n_{es}}}} \\
\vdots \\
\sbrkt{n_{el\,n_{es}}} & \sbrkt{n_{n_{el\,n_{es}}1}} & \dots  & 
\sbrkt{n_{n_{el\,n_{es}}n_{en\,n_{es}}}}
\end{matrix} \notag
\end{align}
The set subscript on the nodes in each element has been omitted for 
clarity.
The connectivity of each element is preceded by its index. 
Connectivities may appear in any order; 
however, \tahoe does not verify that all $n_{el}$ 
elements have been assigned connectivities exactly once.
Note that the dimensions of the set are from the from the dimensions 
section are repeated but that the ID of each set is not repeated.
When specifying the data for a side set in an external file, the 
primary TahoeII contains the path to the external file, as shown below:
\[
\texttt{*set~} \sbrkt{\textrm{\textit{path to external file}}}
\]

\paragraph{Nodal coordinates}
The coordinate specification is marked by the \texttt{nodes} keyword.
The format of coordinate data is the TahoeI format described in 
Section~\ref{sect.TahoeI.coordinates}.
An example is shown below:
\begin{align}
& \texttt{\# nodal coordinates} \notag \\
& \texttt{*nodes} \notag \\
& \texttt{\# dimensions} \notag \\
& \sbrkt{n_{nd} = \textit{number of points}} \notag \\
& \sbrkt{n_{sd} = \textit{number of spatial dimensions}} \notag \\
& \texttt{\# coordinates} \notag \\
& \begin{matrix}
\texttt{1} & \sbrkt{x_{11}} & \dots  & \sbrkt{x_{1n_{sd}}} \\
\vdots \\
\sbrkt{n_{nd}}  & \sbrkt{x_{n_{nd}1}} & \dots & \sbrkt{x_{n_{nd}n_{sd}}}
\end{matrix} \notag
\end{align}
The coordinate of each point is preceded by its index. Coordinates 
may appear in any order; however, \tahoe does not verify that all 
$n_{nd}$ points have been assigned coordinates exactly once.
Coordinate data may be specified in the primary TahoeII file or in an 
external file.
When specifying the data for a side set in an external file, the 
primary TahoeII contains the path to the external file, as shown below:
\begin{align}
& \texttt{\# nodal coordinates} \notag \\
& \texttt{*nodes~} \sbrkt{\textrm{\textit{path to external file}}}
\end{align}

\subsubsection{\textsf{ExodusII/Genesis} format}
\label{sect.file.exodusII}
The \textsf{ExodusII}~\cite{ExodusII} finite element geometry and results file 
format was developed at Sandia National Laboratories. This file format 
is export-controlled, and therefore is not available to all users.
A given database should contain a coordinate list, element blocks,
and any node set or sides sets references from the 
parameters file.  \tahoe does not read material data from the 
\textsf{ExodusII} database. 
Node and element maps are currently ignored and items are assumed to be
consecutively numbered.  To refer to a specific set or block in the
database use the ID value.  The blocks and sets do not need to have
consecutive ID values.

\subsection{Results file formats}
\label{sect.file.results}
Output data is associated with an output group.  An output group is not
necessarily the same as the element group.  Element groups can have more
than one connectivity set for post-processing purposes.  Some element
groups have no output.  These connectivity sets are not the element blocks
listed in the parameters file.  For instance, to visualize an internal 3D
crack, those elements that represent the crack (localized elements or
cohesive surfaces) are separated into an additional connectivity within the
element group.  This allows post-processors to easily show the crack
without having to create isosurfaces or other body representations.

Variable output data is either associated with a node or an element.  Nodal
data are variable values defined at the nodes within the output group. 
Element data are variables values associated with the elements.  Each
element variable has one value per element.  All element quadrature data is
projected to the nodes using the smoothing algorithm proposed by 
Zienkiewicz~\cite{Zienkiewicz}.

For the purposes of writing output files, the output print increment is a
value starting at one and is consecutively incremented.  Thus the print
increment is not the same as the computational increments used by the
program.  This accounts for changes in the frequency with which
computational increments are printed and some I/O databases demand
consecutive numbering.

Depending upon the type of output database chosen, different output files
are created.  File extensions are listed in 
Table~\ref{tab.file.extensions}. The log/output file
is always created and contains an echo of the input data.  If multiple
loading sequences are used, the extensions for geometry and variable data
will be lengthened to include sequence numbers.  \textsf{EnSight}
uses the variable
labels as file extensions on its variable files.  For 
\textsf{ExodusII} and
\textsf{TecPlot}, the variable labels are used as variable names.
The nodes are broken up into those used by the connectivities.  Node
numbers coincide with the internal consecutive node numbering.
\begin{table}[h]
\caption{\label{tab.file.extensions} File extensions for results files.}
\begin{center}
\begin{tabular}[c]{|c|c|c|}
\hline
\parbox[b]{1.0in}{\centering \textbf{file type}} &
\parbox[b]{2.0in}{\raggedright \textbf{extension}} &
\parbox[b]{2.0in}{\raggedright \textbf{description}} \\
\hline
\multirow{4}{1.0in}{\centering plain text} &
\parbox[c]{2.0in}{\raggedright \texttt{.out}} &
\parbox[c]{2.0in}{\raggedright input echo/runtime log} \\
\hhline{~--}
&
\parbox[c]{2.0in}{\raggedright \texttt{.io}$\sbrkt{\textit{group}}$\texttt{.fracture}} &
\parbox[c]{2.0in}{\raggedright fracture surface area} \\
\hhline{~--}
&
\parbox[c]{2.0in}{\raggedright \texttt{.nd}$\sbrkt{\textit{node}}$\texttt{.hst}} &
\parbox[c]{2.0in}{\raggedright nodal history} \\
\hhline{~--}
&
\parbox[b]{2.0in}{\raggedright \texttt{.rs}$\sbrkt{\textit{step}}$} &
\parbox[b]{2.0in}{\raggedright restart data} \\
\hline
\multirow{2}{1.0in}{\centering TahoeI} &
\parbox[b]{2.0in}{\raggedright \texttt{.geo}} &
\parbox[b]{2.0in}{\raggedright output geometry} \\
\hhline{~--}
&
\parbox[b]{2.0in}{\raggedright \texttt{.run}} &
\parbox[b]{2.0in}{\raggedright output results} \\
\hline
\parbox[b]{1.0in}{\centering \textsf{TecPlot}} &
\parbox[b]{2.0in}{\raggedright \texttt{.io}$\sbrkt{\textit{group}}$\texttt{.dat}} &
\parbox[b]{2.0in}{\raggedright output results} \\
\hline
\multirow{4}{1.0in}{\centering \textsf{EnSight}} &
\parbox[b]{2.0in}{\raggedright \texttt{.io}$\sbrkt{\textit{group}}$\texttt{.case}} &
\parbox[b]{2.0in}{\raggedright case file} \\
\hhline{~--}
&
\parbox[b]{2.0in}{\raggedright \texttt{.io}$\sbrkt{\textit{group}}$\texttt{.geo}} &
\parbox[b]{2.0in}{\raggedright output geometry} \\
\hhline{~--}
&
\parbox[b]{2.0in}{\raggedright 
\texttt{.io}$\sbrkt{\textit{group}}$\texttt{.ps}$\sbrkt{\textit{step}}$\texttt{.geo}} &
\parbox[b]{2.0in}{\raggedright changing output geometry} \\
\hhline{~--}
&
\parbox[b]{2.0in}{\raggedright
\texttt{.io}$\sbrkt{\textit{group}}$\texttt{.ps}$\sbrkt{\textit{step}}$\texttt{.}%
$\sbrkt{\textit{label}}$} &
\parbox[b]{2.0in}{\raggedright output variable} \\
\hline
\multirow{2}{1.0in}{\centering \textsf{ExodusII}} &
\parbox[b]{2.0in}{\raggedright 
\texttt{.io}$\sbrkt{\textit{group}}$\texttt{.exo}} &
\parbox[b]{2.0in}{\raggedright output results} \\
\hhline{~--}
&
\parbox[b]{2.0in}{\raggedright 
\texttt{.io}$\sbrkt{\textit{group}}$\texttt{.ps}$\sbrkt{\textit{step}}$\texttt{.exo}} &
\parbox[b]{2.0in}{\raggedright changing geometry output} \\
\hline
\end{tabular}
\end{center}
\end{table}
As described in Section~\ref{sect.parallel.analysis}, \tahoe has two 
output modes for parallel execution. 
\tahoe can join the results data from 
all the processors at run time, or the distributed results data may be joined 
during a separate post-processing step. With distributed output files,
the extension \texttt{.p}$\sbrkt{\textit{processor}}$ is prepended to
each of the extensions listed in Table~\ref{tab.file.extensions}.

\subsubsection{\textsf{TecPlot}}
The \textsf{TecPlot}~\cite{TecPlot8} files are currently written in version 7.5 
plain text format. 
Element data is ignored and not written to an output file.  Element data
must be averaged or projected to the nodes. \textsf{TecPlot}
plain text files can be
converted to binary files, which are generally portable, using the 
\textsf{PrePlot}
executable supplied by \textsf{TecPlot}.

\subsubsection{\textsf{EnSight}}
\textsf{EnSight}~\cite{EnSight6} 
files are written in version 6 Gold plain text or binary format. 
The \textsf{EnSight} binary files are generally portable between some platforms. 
Users should conduct a preliminary test to verify this portability. 
Version 6 Gold files are readable by \textsf{EnSight} 
version 6 Gold and by \textsf{EnSight}
version 7 or later.  The \textsf{EnSight} case file is written at each print
increment to account for unexpected machine crashes.  Each output group is
treated as a part and is written to a separate geometry and case file. 
Separate files are produced so only the part that changes will be printed
at each print increment.  Thus reducing the amount of file space needed for
large meshes.  Element and node tags are not given.  Any vector variable
internally named with the
$\cbrkt{\texttt{\_{X}}, \texttt{\_{Y}}, \texttt{\_{Z}}}$
convention will be written to a
vector variable file instead of a scalar variable file.

\subsubsection{\textsf{ExodusII}}
\textsf{ExodusII} files are portable to all platforms that have the library
installed on them.  Each output group is treated as a block and is written
to a separate file.  Separate files are produced so only the block that
changes will be printed at each print increment.  Separate files are also
produced so that nodal data from different output groups does not need to
be merged.  Element numbering and ordering maps are not given.  The nodes
are broken up into those used by the connectivities.  
\textsf{ExodusII} files, from
multiple restarts, can be joined together using the \texttt{conex} executable.

\subsubsection{Fracture surface area}
The fracture surface area file contains columnar data, in the order listed
as follows: 
\begin{itemize}
\item[(1)] time
\item[(2)] fracture surface area
\end{itemize}

\subsubsection{Node history output}
\label{sect.file.node.history}
The history node file contains columnar data, in the order listed as 
follows. \\
\begin{itemize}
\item[(1)] time
\item[(2)] displacement vector
\item[(3)] velocity vector (if dynamic or transient)
\item[(4)] acceleration vector (if dynamic)
\item[(5)] force vector (if prescribed DOF exists)
\end{itemize}
