%&LaTeX
% $Id: ICandBC.tex,v 1.4 2001-04-27 22:43:55 paklein Exp $

\section{Initial conditions and boundary conditions}
\label{set.ICandBC}
Initial conditions and boundary conditions may be specified on 
selected nodes. Additionally, the variational form 
of governing equations leads to natural boundary conditions that may 
be specified on selected parts of the domain boundary. \tahoe 
supports a number of geometry database types. These are listed in 
Table~\ref{tab.input.file.type}. The database format determines how 
nodes or sets of nodes are identified and how parts of the domain 
boundary are identified.
\begin{table}[h]
\begin{center}
\caption{\label{tab.BC.type} Identification of nodes and 
boundaries.}
\begin{tabular}[c]{|c|c|c|}
\hline
 \parbox[b]{1.25in}{\centering \textbf{database}}
&\parbox[b]{1.5in}{\centering \textbf{node ID}}
&\parbox[b]{1.5in}{\centering \textbf{side ID}}\\
\hline
  \parbox[b]{1.25in}{\centering TahoeI}
& \parbox[b]{1.5in}{\centering node}
& \parbox[b]{1.5in}{\centering element,facet} \\
\hline
  \parbox[b]{1.25in}{\centering TahoeII}
& \parbox[b]{1.5in}{\centering node set ID}
& \parbox[b]{1.5in}{\centering side set ID} \\
\hline
  \parbox[b]{1.25in}{\centering \textsf{ExodusII}~\cite{ExodusII}}
& \parbox[b]{1.5in}{\centering node set ID}
& \parbox[b]{1.5in}{\centering side set ID} \\
\hline
\end{tabular}
\end{center}
\end{table}
Table~\ref{tab.BC.type} lists how nodes and element side are 
identified for the different database types. Their use will be 
explained with examples below. For the TahoeI format, boundary 
segments are explicitly specified with an element and facet number. 
Canonical facet numbering for the different integration domain 
geometries supported by \tahoe are listed in 
Section~\ref{sect.integration.domains}. For TahoeII and 
\textsf{ExodusII} databases, groups of nodes may be collected into 
node sets and groups of element facets are collected into side sets. 
Both types of sets are identified with ID numbers.

Many of the boundary conditions may be specified as functions of 
time. Their values have the form
\begin{equation}
\label{eq.func.time}
    g\farg{t} = g\,f^{\rbrkt{i}}\farg{t},
\end{equation}
where $g$ is a scalar coefficient and $f^{\rbrkt{i}}\farg{t}$ is a
schedule function. Schedule functions are explained in 
Section~\ref{sect.sched.param}. Time varying functions in the input 
file are specified by providing $g$ and $i$, as defined 
in~\eqref{eq.func.time}.

\subsection{Nodal initial conditions}
\label{set.ICandBC.IC}
Each initial condition is specified as
\[
\sbrkt{\textit{node ID}} \quad 
\sbrkt{\textit{degree of freedom}} \quad
\sbrkt{\textit{value}}
\]
The definition of the \textit{node ID} depends on the geometry 
database type as listed in Table~\ref{tab.BC.type}. A \textit{node ID} 
of \texttt{-1} applies the boundary condition to \textit{all} nodes.

\subsection{Nodal kinematic boundary conditions}
\label{set.ICandBC.KBC}
Each nodal kinematic boundary condition is specified as
\[
\sbrkt{\textit{node ID}} \quad 
\sbrkt{\textit{degree of freedom}} \quad
\sbrkt{\textit{code}} \quad
\sbrkt{\textit{schedule}} \quad
\sbrkt{\textit{value}} 
\]
The definition of the \textit{node ID} depends on the geometry 
database type as listed in Table~\ref{tab.BC.type}. The \textit{code} 
specifies if the condition is applied to the nodal unknown $u$, or one 
of its time derivatives $\dot{u}$, $\ddot{u}$, or higher. The codes 
are listed in Table~\ref{tab.KBC.code}.
\begin{table}[h]
\begin{center}
\caption{\label{tab.KBC.code} Codes for nodal kinematic boundary 
conditions.}
\begin{tabular}[c]{|c|c|}
\hline
 \parbox[b]{0.75in}{\centering \textbf{code}}
&\parbox[b]{0.75in}{\centering \textbf{type}}\\
\hline
 \parbox[b]{0.75in}{\centering 0}
&\parbox[b]{0.75in}{\centering \textit{fixed}}\\
\hline
 \parbox[b]{0.75in}{\centering 1}
&\parbox[b]{0.75in}{\centering $u$}\\
\hline
 \parbox[b]{0.75in}{\centering 2}
&\parbox[b]{0.75in}{\centering $\dot{u}$}\\
\hline
 \parbox[b]{0.75in}{\centering 3}
&\parbox[b]{0.75in}{\centering $\ddot{u}$}\\
\hline
\end{tabular}
\end{center}
\end{table}
The \textit{value} and \textit{schedule} in the nodal kinematic 
boundary condition specification refer to 
$g$ and $i$ in~\eqref{eq.func.time}, respectively, for the variation 
of the boundary condition in time.

\subsection{Kinematic boundary condition controllers}
\label{set.ICandBC.KBC.controller}
\todo{table of codes}
\subsubsection{K-field}
\subsubsection{Interface K-field}
\subsubsection{Periodic displacement}
\subsection{Applied nodal forces}
\label{set.ICandBC.FBC}
Each nodal force boundary condition is specified as
\[
\sbrkt{\textit{node ID}} \quad 
\sbrkt{\textit{degree of freedom}} \quad
\sbrkt{\textit{schedule}} \quad
\sbrkt{\textit{value}}
\]
The definition of the \textit{node ID} depends on the geometry 
database type as listed in Table~\ref{tab.BC.type}. 
The \textit{value} and \textit{schedule} in the nodal kinematic 
boundary condition specification refer to 
$g$ and $i$ in~\eqref{eq.func.time}, respectively, for the variation 
of the boundary condition in time.

\subsection{Force controllers}
\label{set.ICandBC.FBC.controller}
\todo{mention that this boundary condition type actually behaves more like 
``elements'' because they do produce a contribution to the stiffness
matrix, i.e., they act as ``follower forces'' and are not simply 
defined as functions of time.}\\
\todo{table of codes}
\subsubsection{Rigid wall (penalty method)}
\subsubsection{Rigid sphere (penalty method)}
\subsubsection{Rigid sphere (augmented Lagrangian method)}
\subsubsection{Rigid sphere for meshfree domains (penalty method)}

\subsection{Body forces}
\label{set.ICandBC.body}
Certain element formulations include body forces.
Applied body forces are specified with the vector analog 
of~\eqref{eq.func.time}.
Their values have the form
\begin{equation}
\label{eq.vec.func.time}
    \mathbf{b}\farg{t} = \mathbf{b}\,f^{\rbrkt{i}}\farg{t},
\end{equation}
where $\mathbf{b}$ is a vector coefficient of length $n_{dof}$, the 
number of degrees of freedom per node, and $f^{\rbrkt{i}}\farg{t}$ 
is a scalar schedule function. Schedule functions are explained in 
Section~\ref{sect.sched.param}.
Body forces are specified as
\[  
\sbrkt{\textit{schedule}} \quad 
\sbrkt{b_{1}} \quad \ldots \quad
\sbrkt{b_{n_{dof}}}
\]

\subsection{Natural boundary conditions}
\label{set.ICandBC.traction}
Recasting balance laws in weak form leads to natural boundary 
conditions involving flux terms across domain surfaces. These fluxes
are specified as part of the input of element parameters. Fluxes are 
associated with the faces of elements. As shown in Table~\ref{tab.BC.type}, 
the description of element faces differs depending on the geometry 
database format. Each natural boundary condition requires specification
of the flux vector at every node on the selected element face, or set 
of faces. Canonical number of element faces and node numbering over 
each face are described in Section~\ref{sect.integration.domains}.
Natural boundary conditions are specified as
\begin{align}
& \sbrkt{\textit{side ID}} \quad 
\sbrkt{\textit{schedule}} \quad 
\sbrkt{\textit{coordinate system}} \notag \\
& \begin{matrix}
\sbrkt{t_{11}} & \ldots & \sbrkt{t_{1n_{dof}}} \\
\vdots \\
\sbrkt{t_{n_{fn}1}} & \ldots & \sbrkt{t_{n_{fn}n_{dof}}}
\end{matrix} \notag
\end{align}
Each flux vector is length $n_{dof}$, and a vector must be specified 
for all $n_{fn}$ nodes on the element face.
The \textit{coordinate system} is identified by the values listed in 
Table~\ref{tab.BC.coord.sys}.
\begin{table}[h]
\begin{center}
\caption{\label{tab.BC.coord.sys} Coordinate systems for natural 
boundary conditions.}
\begin{tabular}[c]{|c|c|}
\hline
 \parbox[b]{0.75in}{\centering \textbf{code}}
&\parbox[b]{1.5in}{\centering \textbf{description}} \\
\hline
 \parbox[b]{0.75in}{\centering 0}
&\parbox[b]{1.5in}{\centering Cartesian} \\
\hline
 \parbox[b]{0.75in}{\centering 1}
&\parbox[b]{1.5in}{\centering facet local} \\
\hline
\end{tabular}
\end{center}
\end{table}
The facet local coordinate frame is constructed such that the last 
component of the flux vector $t_{n_{dof}}$ corresponds to flux in the 
outward normal direction. The tangential components of the flux vector a 
constructed from the mapping of the face to its corresponding parent 
integration domain.

