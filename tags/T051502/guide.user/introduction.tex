%&LaTeX
% $Id: introduction.tex,v 1.4 2002-04-13 02:17:29 paklein Exp $

\section{Introduction}
\tahoe is a general purpose simulation code.  It originally included only
finite element methods, but has been extended to include meshfree methods,
atomistics, and cohesive surface approaches to modeling fracture.  Aside
from interfaces for adding element formulations and constitutive models,
\tahoe also supports specialized boundary conditions such as $K$-fields and
rigid boundaries of various geometries.  The code has also been extended
parallel execution.

The purpose of this guide is to explain how to run \tahoe in its various
modes.  \tahoe execution is driven by a parameters file.  The layout of this
file is described in Section~\ref{sect.inputfile}.  
The parameters in the file are scanned
sequentially.  The interceding chapters provide some additional information
about the various formulations available for execution.  This guide is not
intended as a guide of developers.  Although some references to specific
classes do appear throughout this guide, the structure of the code itself
is not explained in any detail.  Additional information about the code
itself can be obtained from the \tahoe Development Server at
\begin{center}
\href{http://tahoe.ca.sandia.gov}{\texttt{\textless http://tahoe.ca.sandia.gov\textgreater}}
\end{center}
If you are interested in getting access to the code repository for \tahoe, please contact
\begin{center}
\href{mailto:tahoe-help@sandia.gov}{\texttt{tahoe-help@sandia.gov}}
\end{center}
This guide contains hypertext links.  These can be used to navigate the
guide if your document viewer support links.

\subsection{Conventions}
\todo{describe conventions used for this guide.}

\subsection{Overview of \tahoe components}
\label{sect.components.overview}
Although this guide will not discuss the implementation of \tahoe in 
any detail, some discussion of its internal organization is included 
to provide a rationale for the organization of input parameters. 
Internally, \tahoe is organized into the components shown in 
Figure~\ref{fig.tahoe.components}.
\begin{figure}[h]
\centerline{\includegraphics[scale = 1.0]
{\dirfilepath{\figpath}{manager.pdf}}}
\hangcaption
[\tahoe components]
{Division of functionality and data within \tahoe.\label{fig.tahoe.components}}
\end{figure}
Data and parameters are ``owned'' by specific components of the 
code. A brief description of the components follows. For further 
information on a given capability, identify the component whose scope 
covers the capability and seek more information in the later sections of 
this guide.

\tahoe is a hierarchical collection of components, or managers. Each 
manager is responsible for specific data and functionality. The 
configuration of each manager is set in the parameters input file. A 
brief description of each follows.

\subsubsection{\tahoe manager}
The \texttt{Tahoe manager} sits at the top of the manager hierarchy. 
It is responsible of for first configuring other managers during the 
initialization phase of a simulation, checking 
for consistency and compatibility of input parameters, and second 
serving as the back bone for communication between the managers 
during the solution phase. Additionally, the
\texttt{Tahoe manager} is the ``last hope'' for trapping and 
correcting exceptional conditions that occur during the solution phase.

\subsubsection{Time manager}
The \texttt{time manager} is responsible for all parameters related 
to the solution progress variable, or time. This includes information 
about the initial time, final time, and time step. 
Boundary conditions may be specified as functions of time. These 
schedules are specified to the \texttt{time manager}. With solution 
procedures for which it is supported, the \texttt{time manager} 
handles automatic time step control.

\subsubsection{Controller}
\tahoe supports a number of time integration schemes that are handled 
by the \texttt{controller}. There are no input parameters for 
directly affecting the configuration of the \texttt{controller}. The 
time integration scheme is determined by parameters specified to the 
\texttt{Tahoe manager} and the \texttt{time manager}.

\subsubsection{Node manager}
The \texttt{node manager} is responsible for handling the fields of 
unknowns, i.e. $\mathbf{u}\farg{\mathbf{X},t}$. Depending on the 
type of analysis, this data includes the field variables and their 
derivatives at the current time step $t_{n}$, the last time 
$t_{n-1}$, and at other times $t_{n-N}$ for recovery from
exceptional conditions that occur during the solution phase.
The \texttt{node manager} handles essential boundary conditions and 
specification of the externally applied conjugate forces.

\subsubsection{Element groups}
An \texttt{element group} in \tahoe generally refers to the 
implementation of any formulation that translates nodal field 
variables to their conjugate forces. For most simulations, a majority 
of the parameters provided to \tahoe are concerned with the 
\texttt{element groups}.  For finite element types, these parameters 
would include specifications of the element geometry and integration 
scheme for the element-level variational equations. The specification 
of constitutive parameters is embedded in the element data although 
the types of constitutive models will depend on the particular element.
The elements supported by \tahoe are listed in Table~\ref{tab.element.types}.

\subsubsection{I/O manager}
The \texttt{I/O manager} is responsible for supporting the various input 
formats for geometry information and the various output formats for 
results data. There are no input parameters for directly affecting the 
\texttt{I/O manager}. I/O file formats are specified in the parameters for the 
\texttt{Tahoe manager}. A description of supported file formats is given in 
Section~\ref{sect.files}.

\subsubsection{Solver}
The \texttt{solver} is responsible bringing the system to equilibrium
\[
\mathbf{F}^{\textrm{int}}\farg{\mathbf{u}} =
\mathbf{F}^{\textrm{ext}}\farg{t}.
\]
Depending on the type of analysis, the \texttt{solver} may require 
specifications of the nonlinear solution algorithm and 
convergence tolerances. 
Parameters for solvers for linear system of equations is embedded in 
the \texttt{solver} data, including installed third-party libraries.

\subsection{Supported platforms}
\todo{provide table of supported architectures, compilers, and 
extensions, i.e., MPI.}