%&LaTeX
% $Id: input.tex,v 1.3 2001-04-23 05:07:41 paklein Exp $
\section{Input file version \inputversion}
\label{sect.inputfile}
\begin{figure}[h]
\centerline{\includegraphics[scale = 1.0]
{\dirfilepath{\figpath}{input_eg.eps}}}
\hangcaption
{Layout of a \tahoe parameters file.
\label{fig.inputfile.layout}}
\end{figure}
Simulation parameters for \tahoe are specified in
white space delineated, plain text files.
Input values are separated by white space, but the arrangement of white space
is ignored. 
White space includes blank spaces, tabs, new line characters, and 
carriage returns. \tahoe uses calls to the platform-specific, 
standard \texttt{C/C++} libraries to handle input from text files.
Plain text files contain platform-specific line ending characters
that may cause compatibility problems when files are transferred 
between platforms unless some file translation is applied.
The \texttt{\#} symbol denotes that the remainder 
of a given line of text should be ignored, or treated as a comment.
Values are read from the parameters file sequentially.
Therefore, all parameters must be specified, and there are no default 
values. The only exceptions to these conventions may come in 
specifications to the third-party libraries used by \tahoe for certain 
types of analysis.

As discussed in Section~\ref{sect.components.overview}, \tahoe is 
internally divided into components. Data and parameters are owned by 
each component. Figure~\ref{fig.inputfile.layout} shows how the 
parameters file may be divided into sections of parameters for the 
different components. 
\begin{itemize}
\item[(1)] ``Global'' analysis parameters
\item[(2)] Time parameters
\item[(3)] Nodal parameters
\item[(4)] Element group(s) parameters
\item[(5)] Solver parameters
\end{itemize}
Details of the parameters in each block follow 
below.

\subsection{Global analysis parameters}
\label{sect.analysis.params}
The format of the global analysis parameters is shown below:
\begin{inputfile}
%    # job file marker
\inputversion  # input file version marker
# title
\invar{one line of at most 254 characters}
#### analysis parameters
\invar{analysis code}
\invar{input format}
\invar{output format}
\invar{read restart flag}
\invar{write restart increment}
\invar{echo input flag}
\end{inputfile}
The first character in a parameters file must be \texttt{\%}. The 
input file version is used by \tahoe to determine the compatibility of 
the input file with the version of the executable. \tahoe issues a 
warning if the input file version is older than the current. 
Developers are encouraged to make changes to the input file format 
backward compatible; however, backward compatibility cannot be 
maintained in all cases. Changes to the input file format since 
version 3.1 are logged in Section~\ref{sect.changes.log}.
Values of the \textit{analysis code} are listed in 
Table~\ref{tab.analysis.types}. Table~\ref{tab.input.file.type} lists 
the values for the \textit{input format}. For input file formats using 
external geometry files, the code is followed by the relative path to 
the geometry database, as in:
\begin{inputfile}
# input format
5    # ExodusII format
\invar{path to geometry database}
\end{inputfile}	
Table~\ref{tab.output.file.type} lists the values for the 
\textit{output format}. The \textit{read restart flag} is boolean of either 
\texttt{0} or \texttt{1}. A \texttt{1} indicates that \tahoe should 
initialize the system using information in a restart file. In this 
case, the flag is followed by the relative path to the restart file, 
as in:
\begin{inputfile}
# read restart flag
1   \invar{path to restart file}
\end{inputfile}
A \textit{write restart increment} of $N > 0$ indicates that \tahoe 
should write a restart file every $N$ steps.
The \textit{echo input flag} is a boolean flag of 
either \texttt{0} or \texttt{1}. A value of \texttt{1} causes verbose 
echoing of input parameters and geometry information.

\subsection{Time parameters}
\label{sect.time.parameters}
The time parameters are grouped into three sections:
\begin{align}
& \texttt{\#\#\#\# time parameters} \notag \\
& \texttt{\# list of time sequences} \notag \\
& \sbrkt{n_{seq} = \textit{number of time sequences}} \notag \\
& \begin{matrix}
\texttt{1} & \sbrkt{\textit{parameters for sequence~} 1} \\
\vdots \\
\sbrkt{n_{seq}}  & \sbrkt{\textit{parameters for sequence~} n_{seq}}
\end{matrix} \notag \\
& \texttt{\# list of schedule functions} \notag \\
& \sbrkt{n_{scf} = \textit{number of schedule functions}} \notag \\
& \begin{matrix}
\texttt{1} & \sbrkt{\textit{parameters for schedule function~} 1} \\
\vdots \\
\sbrkt{n_{scf}}  & \sbrkt{\textit{parameters for schedule function~} n_{scf}}
\end{matrix} \notag \\
& \texttt{\# time integration parameters} \notag \\
& \invar{\textit{parameter list}} \notag
\end{align}
The numbered sets of time sequence parameters and numbered schedule 
functions may appear in any order in their respective lists. The 
format of time sequence parameters is described in 
Section~\ref{sect.tseq.param}, while the format of schedule functions 
in described in Section~\ref{sect.sched.param}. Most of the time 
integration schemes do not have any input parameters. Only the time 
integration scheme used for implicit elastodynamics 
(see Table~\ref{tab.analysis.types}) requires additional parameters.
The HHT-$\alpha$~\cite{Hughes1977} integrator takes the single 
parameter $0 \leq \alpha \leq 1$, where $\alpha > 0$ introduces 
numerical dissipation. Dissipation is not supported for nonlinear 
analysis.

\subsubsection{Time sequence parameters}
\label{sect.tseq.param}
A time sequence is specified by the following list of parameters:
\begin{align}
& \texttt{\# time sequence parameters} \notag \\
& \sbrkt{\textit{number of steps}} \notag \\
& \sbrkt{\textit{output print increment}} \notag \\
& \sbrkt{\textit{maximum number of step cuts}} \notag \\
& \sbrkt{\textit{initial step size}} \notag
\end{align}
If the \textit{number of steps} is specified as zero, \tahoe will only 
complete the initialization phase of a simulation, reading all 
values from the parameters file. An \textit{output print increment}
specified as $N > 0$ causes results data to be written 
every $N$ steps of the initial step size.
An \textit{output print increment}
specified as $N < 0$ produces output every $N$ steps of the current step size.
Some analysis times support automatic time step control. The 
\textit{maximum number of step cuts} controls how many times the 
initial time step can be reduced by a factor of two before the 
simulation is aborted.

\subsubsection{Schedule functions}
\label{sect.sched.param}
Schedule functions are used to specify the time-varying behavior of 
boundary conditions or other analysis-specific parameters such as the 
thermal expansion in elastostatic and elastodynamic analysis. A
function $f\farg{t}$ is specified with a list of
ordered pairs $\cbrkt{t_{i}, f\farg{t_{i}}}, i=1,\ldots,n_{pts}$. 
The function is linearly interpolated between the data provided. 
The function is extended at zero slope beyond the range of time 
provided, that is
\[
f\farg{t} =
\begin{cases}
f\farg{t_{1}} & \textrm{if~} t < t_{1}, \\
f\farg{t_{i}} + \rbrkt{t - t_{i}} 
\frac{f\rbrkt{t_{i+1}} - f\rbrkt{t_{i}}}{t_{i+1} - t_{i}} & 
\textrm{if~} t_{i} < t <  t_{i+1} \textrm{~for~} i=1,\ldots,n_{pts}-1, \\
f\farg{t_{n_{pts}}} & \textrm{if~}t > t_{n_{pts}}.
\end{cases}	
\]
\todo{add figure}
An example of the input format for schedule function is shown below:
\begin{align}
& \sbrkt{n_{pts} = \textit{number of points on the schedule}} \notag \\
& \begin{matrix}
\sbrkt{t_{1}} & \sbrkt{f\farg{t_{1}}} \\
\vdots \\
\sbrkt{t_{n_{pts}}} & \sbrkt{f\farg{t_{n_{pts}}}}
\end{matrix} \notag
\end{align}
The ordered pairs on the schedule function must be specified with 
time in sequentially increasing order.
\subsection{Nodal parameters}
The nodal parameters of the input file contains the 
following sections:
\begin{align}
& \texttt{\#\#\#\# nodal data} \notag \\
& \sbrkt{\textit{coordinate data}} \notag \\
& \sbrkt{\textit{list of initial conditions}} \notag \\
& \sbrkt{\textit{list of kinematic boundary conditions}} \notag \\
& \sbrkt{\textit{list of kinematic boundary condition controllers}} \notag \\
& \sbrkt{\textit{list of force boundary conditions}} \notag \\
& \sbrkt{\textit{list of force boundary condition controllers}} \notag \\
& \sbrkt{\textit{list of history node specifications}} \notag
\end{align}
The \textit{coordinate data} only appears in the nodal section if
geometry format TahoeI is specified in the global parameters 
section, described in Section~\ref{sect.analysis.params}. For all 
other geometry file formats listed in 
Table~\ref{tab.input.file.type}, the nodal coordinate data resides in 
an external file, and this section of the nodal parameters is omitted.
The format for all of the lists specified in this section is similar. 
The list is initiated with its length and followed by the entries in 
the list if the length is nonzero.
The \textit{list of initial conditions} is either empty
\[ \texttt{0 \# initial conditions}\]	
or contains a list specified as
\begin{align}
& \texttt{\# initial conditions} \notag \\
& \sbrkt{n_{ic} = \textit{number of initial conditions}} \notag \\
& \sbrkt{\textit{parameters for initial condition~} 1} \notag \\
& \quad \vdots \notag \\
& \sbrkt{\textit{parameters for initial condition~} n_{ic}} \notag
\end{align}
The format of initial condition specifications is described in
Section~\ref{set.ICandBC.IC}.
The \textit{list of kinematic boundary conditions}, or essential 
boundary conditions, is either empty
\[ \texttt{0 \# kinematic boundary conditions}\]	
or contains a list specified as
\begin{align}
& \texttt{\# kinematic boundary conditions} \notag \\
& \sbrkt{n_{kbc} = \textit{number of kinematic boundary conditions}} \notag \\
& \sbrkt{\textit{parameters for kinematic boundary condition~} 1} \notag \\
& \quad \vdots \notag \\
& \sbrkt{\textit{parameters for kinematic boundary condition~} n_{kbc}} \notag
\end{align}
The format of kinematic boundary condition specifications is described in
Section~\ref{set.ICandBC.KBC}.
Kinematic boundary conditions allow specification of the primal 
variables along the cartesian axes as a function of time.
Kinematic boundary condition controllers allow specification the field at 
groups of nodes to follow specific functional forms, such as the 
displacements surrounding a crack tip based on the asymptotic $K$-field 
solution from isotropic linear elasticity.
The \textit{list of kinematic boundary condition controllers}
is either empty
\[ \texttt{0 \# kinematic boundary condition controllers}\]	
or contains a list specified as
\begin{align}
& \texttt{\# kinematic boundary condition controllers} \notag \\
& \sbrkt{n_{kctl} = \textit{number of kinematic boundary conditions}} \notag \\
& \sbrkt{\textit{parameters for kinematic boundary condition controller~} 1} \notag \\
& \quad \vdots \notag \\
& \sbrkt{\textit{parameters for kinematic boundary condition controller~} n_{kctl}} \notag
\end{align}
The format of kinematic boundary condition controllers specifications is described in
Section~\ref{set.ICandBC.KBC.controller}.
The \textit{list of force boundary conditions} is either empty
\[ \texttt{0 \# nodal force boundary conditions}\]	
or contains a list specified as
\begin{align}
& \texttt{\# nodal force boundary conditions} \notag \\
& \sbrkt{n_{fbc} = \textit{number of nodal force boundary conditions}} \notag \\
& \sbrkt{\textit{parameters for force boundary condition~} 1} \notag \\
& \quad \vdots \notag \\
& \sbrkt{\textit{parameters for force boundary condition~} n_{fbc}} \notag
\end{align}
The format of nodal force boundary condition specifications is described in
Section~\ref{set.ICandBC.FBC}.
The \textit{list of force boundary condition controllers} is either empty
\[ \texttt{0 \# force boundary condition controllers}\]	
or contains a list specified as
\begin{align}
& \texttt{\# force boundary condition controllers} \notag \\
& \sbrkt{n_{fctl} = \textit{number of force boundary condition controllers}} \notag \\
& \sbrkt{\textit{parameters for force controller~} 1} \notag \\
& \quad \vdots \notag \\
& \sbrkt{\textit{parameters for force controller~} n_{fctl}} \notag
\end{align}
The format of nodal force boundary condition controllers specifications is described in
Section~\ref{set.ICandBC.FBC.controller}.
Force boundary condition controllers implement a number of ``rigid'' 
objects that generate forces on nodes through contact. For example, 
the rigid sphere controllers can be used to represent an indenter 
with a motion specified as a function of time.

Nodes, or sets of nodes, can be specified to produce more detailed 
output. The format of history node output is described in 
Section~\ref{sect.file.node.history}.
The \textit{list of history node specifications} is either empty
\[ \texttt{0 \# number of history nodes}\]	
or contains a list specified as
\begin{align}
& \texttt{\# history nodes} \notag \\
& \sbrkt{n_{hst} = \textit{number of history node ID's}} \notag \\
& \sbrkt{\textit{node ID}_{1}} 
\qquad \ldots \qquad 
\sbrkt{\textit{node ID}_{n_{hst}}} \notag
\end{align}
The \textit{node ID} differs depending on geometry file format (see 
Table~\ref{tab.input.file.type}). For TahoeI format, the 
\textit{node ID} is simply the node number. Some of the geometry 
database formats supported by \tahoe allow nodes to be grouped into 
sets identified by an \textit{ID}. With these database formats, the 
\textit{node ID} corresponds to the node set \textit{ID}.

\subsection{Element parameters}

\subsection{Solver parameters}

\subsection{Batch file format}
