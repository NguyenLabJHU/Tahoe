%&LaTeX
% $Id: analysis.tex,v 1.2 2001-04-23 05:07:41 paklein Exp $

\section{Analysis types}
\label{sect.analysis.types}
Table~\ref{tab.analysis.types} shows a list of analysis available in \tahoe.
\begin{table}[h]
\caption{\label{tab.analysis.types} This is the list of types.}
\begin{center}
\begin{tabular}[c]{|l|c|c|}
\hline
 \parbox[b]{2.0in}{\textbf{analysis type}}
&\parbox[b]{1.0in}{\centering \textbf{linear}}
&\parbox[b]{1.0in}{\centering \textbf{nonlinear}}\\
\hline
elastostatic &1 &3\\
\hline
implicit elastodynamic &2 &4\\
\hline
explicit elastodynamic &6 &7\\
\hline
steady-state diffusion &19 &N/A\\
\hline
transient diffusion &20 &N/A \\
\hline
\end{tabular}
\end{center}
\end{table}

\subsection{Analysis types}
Linear analysis implies that the governing equations are linear in the
unknown variables over a given time increment.  Nonlinear analysis implies
that some iterative procedure is required to determine the solution in the
unknown variables.  Most cases require the nonlinear solution procedures. 
Specifying a nonlinear analysis type for a linear problem will produce a
solution after a single iteration of the solution procedure.

Whether linear or nonlinear, the solution procedure ultimately results in a
linear system of equation.  A number of different storage schemes are
available for the global matrix associated with this linear system.  The
different types of matrix storage are listed in 
Table~\ref{tab.matrix.types}.  The choice of
matrix depends on the analysis type as is outlined below.  Moreover, all
matrix types may be used for serial execution, while parallel execution
must use either the diagonal, \textsf{Aztec}~\cite{Aztecv11}, 
or \textsf{SPOOLES}~\cite{SPOOLESv22} matrices.
\begin{table}[h]
\caption{\label{tab.matrix.types} Matrix types for the global system 
of equations.}
\begin{center}
\begin{tabular}[c]{|c|c|}
\hline
 \parbox[b]{2.0in}{\centering \textbf{matrix type}}
&\parbox[b]{1.0in}{\centering \textbf{code}}\\
\hline
diagonal &0\\
\hline
profile &1\\
\hline
full &2\\
\hline
\textsf{Aztec}~\cite{Aztecv11} &3\\
\hline
\textsf{SPOOLES}~\cite{SPOOLESv22} &5\\
\hline
\end{tabular}
\end{center}
\end{table}

\subsubsection{Elastostatic analysis}
This analysis can be used with profile, full, \textsf{Aztec}, or 
\textsf{SPOOLES} global matrix types.

\subsubsection{Implicit elastodynamic analysis}
This analysis uses HHT-$\alpha$ time integration developed by Hilber 
\etal~\cite{Hughes1977} and can be used with
profile, full, \textsf{Aztec}, 
or \textsf{SPOOLES}  global matrix types.

\subsubsection{Explicit elastodynamic analysis}
This analysis uses the explicit, central-difference time integration scheme
from the classic Newmark family of methods. In order to make this method
``matrix-free'', the diagonal matrix storage should be used.

\subsubsection{Steady-state and transient diffusion}
Transient head conduction uses a trapezoidal time integration scheme, and
can be used with profile, full, \textsf{Aztec}, or 
\textsf{SPOOLES}  global
matrix types.

\subsection{Solution techniques}

\subsubsection{Linear}
The linear solution technique is used with all of the linear analyses and
with explicit dynamic analysis.  Explicit dynamic analysis may involve
large, nonlinear kinematics of deformation, nonlinear constitutive
response, and contact; however, as a result of the time discretization, the
incremental accelerations are linearly related to the driving forces over a
single time increment.

\subsubsection{Nonlinear}
\label{sect.solver.nonlinear}
Nonlinear solution techniques are used with all nonlinear analyses except
with nonlinear explicit dynamics analysis.  Table~\ref{tab.solution.types}
shows a list of
nonlinear solution techniques supported.  As mentioned above, the nonlinear
solution procedures iteratively search for the solution until some
convergence tolerance is satisfied.  In general, a solution procedure
involves determining two aspects of the search for the solution.  The
procedure must determine the ``length'' and ``direction'' of the update to the
current best guess of the solution.  The methods listed in
Table~\ref{tab.solution.types} differ
in how they determine these two parameters.  The various nonlinear solution
methods use two criteria for assessing convergence based on the norm of the
residual forces.  The first is an ``absolute'' tolerance while the second is
a ``relative'' tolerance.  The ``absolute'' tolerance is an indication of
``small'' in terms of the numerical precision displayed for a given
calculation.  Depending on the platform, the various formulations used
throughout a calculation, and details of the implementation, the level of
numerical ``noise'' in a calculation may limit how close to absolute zero the
solution procedure is able to drive the norm of the residual.  The second
tolerance examines the ``relative'' error.  This compares the norm of the
residual at the end of an iteration with the norm of the residual at the
first iteration.  If one examines the console output during a simulation,
the absolute error is written to the output after incremental boundary
conditions are applied followed by the relative error after each iteration. 
The solution procedure stops if either condition is satisfied.
\begin{table}[h]
\caption{\label{tab.solution.types} Nonlinear solution techniques.}
\begin{center}
\begin{tabular}[c]{|c|c|}
\hline
 \parbox[b]{2.0in}{\centering \textbf{solution method}}
&\parbox[b]{1.0in}{\centering \textbf{code}}\\
\hline
standard Newton &0\\
\hline
initial tangent &1\\
\hline
modified Newton &2\\
\hline
dynamic relaxation~\cite{Underwood1983} &3\\
\hline
Newton with line search &4\\
\hline
preconditioned nonlinear conjugate  gradient &5\\
\hline
interactive Newton &6\\
\hline
\end{tabular}
\end{center}
\end{table}

All of the nonlinear solution techniques automatically adjust the time, or
load, increment depending on the behavior of the iterative solution
procedure.  The solvers have user-adjustable parameters for determining
whether the increment should increase or decrease.  A cut in the increment
is triggered if the number of iterations exceeds to the user-defined value
or if the ``relative'' error exceeds a user-defined value.  Additionally, the
increment is decreased if ``overload'' conditions are detected in
calculations anywhere in the code.  Once the increment has been cut, \tahoe
begins to monitor two parameters in combination to determine if the
increment should increased.  The user can prescribe the number of
iterations in a ``quick'' solution and the number of sequential ``quick''
solutions that must pass before the increment is increased.  An increment
adjustment changes the step size by a factor of two.

\paragraph{Standard Newton}
Newton's method is a second order optimization technique that makes use of
a tangent matrix to determine the update vector.  This solver reforms a new
tangent matrix at every iteration, and does not rescale the update vector. 
The convergence behavior of Newton methods is known to be poor ``far'' from
the solution, but the method does display locally quadratic convergence
``near'' the solution.

\paragraph{Initial tangent}
This solver differs from the standard Newton in that a new tangent matrix
is calculated only once per time, or load, increment.  The tangent matrix
is formed for the initial iteration and is subsequently reused.  This
approach results in slower rates of convergence, but may result in faster
solution times since the number of times the tangent matrix is formed and
factorized is reduced.  This method assumes that the matrix type selected
allows the factorized matrix to be reused at.  This method may also provide
better performance in cases in which the tangent matrix becomes rank
deficient.

\paragraph{Modified Newton}
This solver attempts to determine conditions under which to reform or reuse
the tangent matrix based on a series of rules.  This solver may be
considered experimental.

\paragraph{Dynamic relaxation}
Dynamic relaxation~\cite{Underwood1983} 
is a first order optimization technique that
wraps the problem of finding the solution with a set of damped, second
order differential equations.  The solution is determined by dissipating
the transient behavior resulting from residual forces.

\paragraph{Newton with line search}
This solver augments the standard Newton solver with a line search to
determine the length of the update vector.  User-specified parameters
control the effort expended for the line search. The goal of the line 
search is to determine the optimal distance along the update direction.
User-defined parameters control the maximum number of iterations
expended to determine this optimal distance. The optimal distance is
defined by the point along the update vector at which the residual 
vector is orthogonal to the update vector. An \textit{exact} line
search finds this point exactly. A second input parameter provides
an approximation to the exact search.

\paragraph{Preconditioned nonlinear conjugate gradient}
This solver incorporates the preconditioned, conjugate gradient 
first order optimization method. User-specified parameters control the 
number of conjugate gradient iterations between ``restarts'', as well 
as specifications for the line search.

\paragraph{Interactive Newton}
This solver allows the user to interactively control the solution 
process. The underlying solver is a Newton solver with line search. 
The user can step forward or backward in time, modify solver parameters,
and write output data. Additionally, the value of other parameters in 
the simulation may be viewed and changed by addressing specific 
\textit{scopes} in \tahoe. The console commands also change with scope.
