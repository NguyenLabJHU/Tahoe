%&LaTeX
% $Id: tahoe.user.tex,v 1.8 2001-06-04 01:32:53 paklein Exp $
\documentclass[12pt]{article}
%\documentclass[11pt,twoside]{article}
\usepackage{amsmath}
\usepackage{alltt}
\usepackage{xspace}
%\usepackage{array}
\usepackage{multirow}
\usepackage{hhline}
\usepackage{longtable}
\usepackage{ulem}
\usepackage{graphicx}
\usepackage{color}
%\usepackage{overcite}
\usepackage{hangcaption}
\usepackage{fancyhdr}
\usepackage[textures,%
    bookmarks=true,%
	colorlinks=true,%
	bookmarksnumbered=true]{hyperref}
%-------------------------- macros ----------------------
%- NOTE: latex2html has trouble expanding nested macros -
\newcommand{\mathvec}[1]{\mbox{\boldmath $#1$}}
\newcommand{\labeleq}[2]{\begin{equation} \label{#1} #2 \end{equation}}
\newcommand{\rbrkt}[1]{\left(#1\right)}
\newcommand{\sbrkt}[1]{\left[#1\right]}
\newcommand{\cbrkt}[1]{\left\{#1\right\}}
\newcommand{\abrkt}[1]{\left<#1\right>}
\newcommand{\farg}[1]{\negmedspace \left(#1\right)}
\newcommand{\etal}[0]{\textit{et al.}\@ }
\newcommand{\dbyd}[2]{\frac{\partial#1}{\partial#2}}
\newcommand{\ddbydd}[3]{\frac{\partial^{2}#1}{\partial#2\partial#3}}
\newcommand{\invar}[1]{\textrm{[\textit{#1}]}}
%-------------------------- macros ----------------------

%----------------------- environments -------------------
% inputfile: requires the package alltt because the 
% standard verbatim environment fails to find the end of
% inputfile environment
\newenvironment{inputfile}{\\%
\begin{minipage}[t]{\linewidth}%
\begin{minipage}[t]{0.2\linewidth} ~ \end{minipage}%
\begin{minipage}[t]{0.8\linewidth}%
\vspace{\parskip}%
\begin{alltt}}
{\end{alltt}%
\vspace{\parskip}%
\end{minipage}%
\end{minipage}}
%----------------------- environments -------------------

%------------------------- headers ----------------------
\lhead[\fancyplain{}{}]%
      {\fancyplain{}{\slshape\rightmark}}
\rhead[\fancyplain{}{\slshape\leftmark}]%
      {\fancyplain{}{}}
%--- draft      
%\chead{{\textcolor{red}{\textbf{\textsf{DRAFT}}}}}
%\lfoot{{\textcolor{red}{\textbf{\textsf{DRAFT}}}}}
%\rfoot{{\textcolor{red}{\textbf{\textsf{DRAFT}}}}}
%--- draft
%------------------------- headers ----------------------

%---------------------- table macros --------------------
% needed to modify formats of tabular material,
% stands for Preserve Back Slash
\newcommand{\PBS}[1]{\let\temp=\\#1\let\\=\temp}
% fixed width text boxes for tables
\newcommand{\tabcell}[2]{\parbox{#1}{\vspace{0.025in}#2\vspace{0.025in}}}
%---------------------- table macros --------------------

%---------------------- editing macros ------------------
\newcommand{\todo}[1]{\textcolor{red}{\textit{[To do: #1]}}}
\newcommand{\strike}[1]{\textcolor{red}{\sout{#1}}}
%---------------------- editing macros ------------------

%----------------------- figure path --------------------
\newcommand{\figpath}{fig.color}
%\newcommand{\figpath}{fig.grayscale}
%----------------------- figure path --------------------

%----------------- platform dependent macros ------------
%-- Mac pathnames
\newcommand{\filepath}[1]{:#1}
\newcommand{\dirfilepath}[2]{:#1:#2}
\newcommand{\dirdirfilepath}[3]{:#1:#2:#3}

%-- UNIX pathnames
%\newcommand{\filepath}[1]{./#1}
%\newcommand{\dirfilepath}[2]{./#1/#2}
%\newcommand{\dirdirfilepath}[3]{./#1/#2/#3}
%----------------- platform dependent macros ------------

%------------------------- format -----------------------
%-- page layout
\setlength{\headsep}{20pt}
\setlength{\topmargin}{0pt}
\setlength{\headheight}{14pt}
\setlength{\textheight}{8.5in}
\setlength{\textwidth}{6.5in}
\setlength{\hoffset}{0.0in}
\setlength{\voffset}{-0.25in}
\setlength{\evensidemargin}{0.0in}
\setlength{\oddsidemargin}{0.0in}
%------------------------- format -----------------------

%------------------ selective typesetting ---------------
%\includeonly{\filepath{input}, \filepath{files}, \filepath{ICandBC}}
%------------------ selective typesetting ---------------

%------------------- document variables -----------------
\newcommand{\inputversion}{3.4.1}
\newcommand{\tahoe}{\textsf{Tahoe}\xspace}
%------------------- document variables -----------------

\begin{document}
\setcounter{tocdepth}{5}
\setcounter{secnumdepth}{5}
\numberwithin{equation}{section}
\numberwithin{figure}{section}
\numberwithin{table}{section}
\normalem %-- restore normal behavior for \em
%----------------------- front matter -------------------
\begin{center}
\vspace{0.25in}
{\Huge \tahoe \\}
{\LARGE User Guide \\}
\vspace{0.25in}
Input Version \inputversion \\
\vspace{0.10in}
last updated \today \\
\end{center}
%------------------------ abstract ----------------------
\begin{figure}[h]
\centerline{\includegraphics[scale = 0.75]
{\dirfilepath{\figpath}{Tahoe.eps}}}
\end{figure}
\protect\vspace{0.15in}
\begin{abstract}
\tahoe is a general purpose simulation code incorporating finite 
element, meshfree methods, and atomistic methods. The code is
capable of execution on parallel platforms and includes interfaces
to third-party libraries that can solve linear systems of equations
in parallel. This document is intended as a guide for users. It
does not contain the detailed descriptions of the code required by
developers. This guide is a work in progress. Please send questions
and bug reports to 
\href{mailto:tahoe-help@sandia.gov}{\texttt{tahoe-help@sandia.gov}}.
\end{abstract}
\protect\vspace{0.15in}
%------------------------ abstract ----------------------
\newpage
\tableofcontents
\newpage
\listoffigures
\listoftables
\newpage
%----------------------- front matter -------------------
%-------------------------- body ------------------------
\pagestyle{fancy}
\include{\filepath{introduction}}
\include{\filepath{execution}}
\include{\filepath{regression}}
\include{\filepath{input}}
\include{\filepath{analysis}}
\include{\filepath{ICandBC}}
\include{\filepath{element}}
\include{\filepath{material.solid}}
\include{\filepath{files}}
\include{\filepath{changes}}
%-------------------------- body ------------------------
%----------------------- back matter --------------------
%\appendix
% bibliography
\stepcounter{section}
\addcontentsline{toc}{section}{\protect\numberline{\thesection}References}
\bibliographystyle{amsplain}
\bibliography{%
   \dirfilepath{bibliography}{general},%
   \dirfilepath{bibliography}{fracture},%
   \dirfilepath{bibliography}{fracture.experiments},%
   \dirfilepath{bibliography}{atomistic},%
   \dirfilepath{bibliography}{meshfree}}
\end{document}
