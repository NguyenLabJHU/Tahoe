%&LaTeX
% $Id: introduction.tex,v 1.2 2001-04-10 01:57:49 paklein Exp $

\section{Introduction}
\tahoe is a general purpose simulation code.  It originally included only
finite element methods, but has been extended to include meshfree methods,
atomistics, and cohesive surface approaches to modeling fracture.  Aside
from interfaces for adding element formulations and constitutive models,
\tahoe also supports specialized boundary conditions such as K-fields and
rigid boundaries of various geometries.  The code has also been extended
parallel execution.

The purpose of this guide is to explain how to run \tahoe in its various
modes.  \tahoe execution is driven by a parameters file.  The layout of this
file is described in Section~\ref{sect.inputfile}.  
The parameters in the file are scanned
sequentially.  The interceding chapters provide some additional information
about the various formulations available for execution.  This guide is not
intended as a guide of developers.  Although some references to specific
classes do appear throughout this guide, the structure of the code itself
is not explained in any detail.  Additional information about the code
itself can be obtained from the \tahoe Development Server at
\begin{center}
\href{http://tahoe.ca.sandia.gov}{\texttt{\textless http://tahoe.ca.sandia.gov\textgreater}}
\end{center}
If you are interested in getting access to the code repository for \tahoe, please contact
\begin{center}
\href{mailto:tahoe-help@sandia.gov}{\texttt{tahoe-help@sandia.gov}}
\end{center}
This guide contains hypertext links.  These can be used to navigate the
guide if your document viewer support links.

\subsection{Conventions}

\subsection{Overview of \tahoe components}

\begin{figure}[h]
\centerline{\includegraphics[scale = 1.0]
{\dirfilepath{\figpath}{manager.eps}}}
\hangcaption
[\tahoe components]
{Division of functionality and data within \tahoe.\label{fig.tahoe.components}}
\end{figure}

\subsection{Supported platforms}
