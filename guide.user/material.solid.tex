%&LaTeX
% $Id: material.solid.tex,v 1.7 2002-04-24 22:44:52 dzeigle Exp $

\section{Constitutive models for solid mechanics}
This section of the guide describes the constitutive models available 
for analysis of displacements and stresses. A separate section is 
dedicated to each material model. Each section provides some 
background information about the model, followed up a description of 
the input required for three- and two-dimensional analysis.
All materials support both implicit as well as explicit 
analysis. That is, most models implement consistent tangent matrix 
that should produce locally quadratic convergence when used with 
Newton-type, nonlinear solution algorithms 
(see Section~\ref{sect.solver.nonlinear}).
As will be noted in the sections that follow, some models only provide 
approximate tangents that will result in
slower convergence rates with Newton-type solvers.

\begin{table}[h]
\caption{\label{tab.mat.solid} Materials for analysis of 
displacements and stress.}
\begin{center}
	
\begin{longtable}[c]{|c|c|}
\hline
 \parbox[c]{0.75in}{\centering \textbf{code}}
&\parbox[c]{4.5in}{\raggedright \textbf{description}}\\
\endhead
\endfoot
\hline
1 & \parbox[c]{4.5in}{\raggedright small deformation Kirchhoff-St.~Venant}\\
\hline
2 & \parbox[c]{4.5in}{\raggedright finite deformation Kirchhoff-St.~Venant}\\
\hline
3 & \parbox[c]{4.5in}{\raggedright small deformation cubic}\\
\hline
4 & \parbox[c]{4.5in}{\raggedright finite deformation cubic}\\
\hline
5 & \parbox[c]{4.5in}{\raggedright uncoupled volumetric/deviatoric}\\
\hline
6 & \parbox[c]{4.5in}{\raggedright quadratic-logarithmic (Simo)}\\
\hline
7 & \parbox[c]{4.5in}{\raggedright quadratic-logarithmic (Ogden)}\\
\hline
8 & \parbox[c]{4.5in}{\raggedright small deformation $J_{2}$ plasticity}\\
\hline
9 & \parbox[c]{4.5in}{\raggedright finite deformation $J_{2}$ plasticity}\\
\hline
10 & \parbox[c]{4.5in}{\raggedright finite deformation $J_{2}$ plasticity in 
principal stretches}\\
\hline
11 & \parbox[c]{4.5in}{\raggedright small deformation Drucker-Prager plasticity}\\
\hline
12 & \parbox[c]{4.5in}{\raggedright 2D hexagonal Lennard-Jones lattice}\\
\hline
13 & \parbox[c]{4.5in}{\raggedright plane strain FCC Lennard-Jones lattice}\\
\hline
14 & \parbox[c]{4.5in}{\raggedright FCC lattice with EAM potentials}\\
\hline
15 & \parbox[c]{4.5in}{\raggedright Stillinger-Weber diamond cubic}\\
\hline
16 & \parbox[c]{4.5in}{\raggedright Virtual Internal Bond model}\\
\hline
17 & \parbox[c]{4.5in}{\raggedright isotropic Virtual 
Internal Bond model (Simo)}\\
\hline
18 & \parbox[c]{4.5in}{\raggedright isotropic Virtual 
Internal Bond model (Ogden)}\\
\hline
19 & \parbox[c]{4.5in}{\raggedright isotropic Virtual 
Internal Bond model with $J_{2}$ plasticity}\\
\hline
80 & \parbox[c]{4.5in}{\raggedright \textsf{ABAQUS/Standard~\cite{ABAQUSv56} UMAT} BCJ}\\
\hline
\end{longtable}
\end{center}
\end{table}

\subsubsection{Common input parameters}
\label{sect.material.solid.common}
All constitutive models for solid mechanics share some common input 
parameters. An example of these parameters is shown below:
\begin{inputfile}
# Rayleigh damping parameters
0.0   # mass proportionate damping    
0.0   # stiffness proportionate damping
# mass density
1.0   # mass/reference volume
# thermal expansion
0     # schedule number
0.0   # % expansion
\end{inputfile}

The force for Rayleigh damping is defined as
\begin{equation}
	\mathbf{F}_{R} = \mathbf{C}_{R}\,\mathbf{v}
\end{equation}
where the damping matrix $\mathbf{C}_{R}$ is a linear combination
of the mass and stiffness matrices
\begin{equation}
	\mathbf{C}_{R} = a\,\mathbf{M} + b\,\mathbf{K},
\end{equation}
where $a$ and $b$ are the parameters specified in the input file.

Thermal expansion may be defined as a function of time with 
two parameters. The schedule number 
corresponds to the schedules defined in the time parameters block of 
the input file~\ref{sect.time.parameters}. The second parameter defines the 
volume change as a percentage. Thermal expansion is disabled by 
specifying a value of \texttt{0} for the schedule number. 
Thermal strains are imposed 
differently for small strain and finite 
strain material models. For small strain materials, an additive decomposition 
of the strain is assumed
\begin{equation}
\boldsymbol{\epsilon} = 
    \boldsymbol{\epsilon}^{\theta} + 
	\boldsymbol{\epsilon}^{\mathrm{mechanical}},
\end{equation}
where the thermal strain is 
\begin{equation}
\boldsymbol{\epsilon}^{\theta}\farg{t} = \varepsilon\farg{t} \mathbf{1}
\end{equation}
For materials formulated at finite strains, the multiplicative split 
\begin{equation}
\mathbf{F} = \mathbf{F}^{\mathrm{mechanical}}\,
\mathbf{F}^{\theta},
\end{equation}
of the deformation 
gradient is assumed,
where the thermal deformation gradient is
\begin{equation}
\mathbf{F}^{\theta}\farg{t} = \rbrkt{1 + \varepsilon\farg{t}} \mathbf{1}.
\end{equation}

\subsubsection{Common input parameters for 2D analysis}
\label{sect.material.2D.common}
If constructed using the standard \tahoe materials model toolbox, all 
materials for two-dimensional analysis contain two common parameters.
An example of these parameters is shown below:
\begin{inputfile}
####### common 2D parameters
1.0   # thickness
2     # constraint option
\end{inputfile}
The thickness $t$ is used to transform the differential volume element
as
\begin{equation}
	d\Omega = t\,dx_{1}\,dx_{2}.
\end{equation}
The acceptable values for the constraint option are shown in 
Table~\ref{tab.2Dconstraint}.
\begin{table}[h]
\caption{\label{tab.2Dconstraint} Constraint options for 
two-dimensional analysis.}
\begin{center}
\begin{tabular}[c]{|l|c|c|}
\hline
 \parbox[b]{0.75in}{\centering \textbf{code}}
&\parbox[b]{2.0in}{\raggedright \textbf{description}}\\
\hline
\parbox[b]{0.75in}{\centering 1} & 
\parbox[b]{2.0in}{\raggedright plane stress}\\
\hline
\parbox[b]{0.75in}{\centering 2}  & 
\parbox[b]{2.0in}{\raggedright plane strain}\\
\hline
\end{tabular}
\end{center}
\end{table}
Some materials models have been specifically formulated as plane 
stress or plane strain. These materials will silently 
override the user-specified value of the constraint option.

\subsection{Material 1: small deformation Kirchhoff-St.~Venant}
\label{sect.mat.SSKStV}
The stress response of this isotropic model is 
governed by the strain energy function
\begin{equation}
\Phi\farg{\boldsymbol{\epsilon}}=
\frac{1}{2} \epsilon_{ij}\,\mathsf{C}_{ijkl}\,\epsilon_{kl},
\end{equation}
where $\boldsymbol{\epsilon}$ is the infinitesimal strain tensor
and $\mathvec{\mathsf{C}}$
is the isotropic tensor of elastic constants which may be
written as
\begin{equation}
\label{eq.modulus.isotropic}
	\mathvec{\mathsf{C}} =
	\begin{bmatrix}
		\lambda + 2 \mu & \lambda & \lambda & 0 & 0 & 0 \\
		\lambda & \lambda + 2 \mu & \lambda & 0 & 0 & 0 \\
		\lambda & \lambda & \lambda + 2 \mu & 0 & 0 & 0 \\
		0 & 0 & 0 & \mu & 0 & 0 \\
		0 & 0 & 0 & 0 & \mu & 0 \\
		0 & 0 & 0 & 0 & 0 & \mu
	\end{bmatrix}
\end{equation}
for the three-dimensional case. The Lam\a'e constants $\lambda$ and 
$\mu$ are related to Young's modulus $E$ 
and Poisson's ratio $\nu$ by
\begin{equation}
\label{eq.lambda}
\lambda = \frac{\nu\,E}{\rbrkt{1 + \nu}\rbrkt{1 - 2\,\nu}}
\end{equation}
and
\begin{equation}
\label{eq.mu}
\mu = \frac{E}{2 \rbrkt{1 + \nu}}.
\end{equation}

\subsubsection{3D input parameters}
An example of input parameters for three-dimensional analysis is shown 
below:
\begin{inputfile}
####### Kirchhoff-St.Venant
1     # material code
####### common material parameters
0.0    0.0    1.0
0      0.0
####### moduli
100.0 # Young's modulus
0.25  # Poisson's ratio
\end{inputfile}
After the common parameters described in 
Section~\ref{sect.material.solid.common}, the 
model requires specification of Young's modulus 
$E$ and Poisson's 
ratio $\nu$.

\subsubsection{2D input parameters}
An example of input parameters for two-dimensional analysis is shown 
below:
\begin{inputfile}
####### Kirchhoff-St.Venant
1     # material code
####### common material parameters
0.0    0.0    1.0
0      0.0
####### moduli
100.0 # Young's modulus
0.25  # Poisson's ratio
####### 2D parameters
1.0   # thickness
2     # constraint option
\end{inputfile}
After the common parameters~\ref{sect.material.solid.common}, the 
model requires specification of Young's modulus $E$ and Poisson's 
ratio $\nu$. The material thickness and 
constraint option from Table~\ref{tab.2Dconstraint} 
follow the 
parameters for three-dimensional analysis.

\subsubsection{Output values}
This constitutive model does not have any output values.

\subsection{Material 2: finite deformation Kirchhoff-St.~Venant}
\label{sect.mat.FDKStV}
This model is the finite deformation version of the material model
described in Section~\ref{sect.mat.SSKStV}.
The stress response of this isotropic model is 
governed by the strain energy function
\begin{equation}
\Phi\farg{\mathbf{E}}=
\frac{1}{2} E_{ij}\,\mathsf{C}_{ijkl}\,E_{kl},
\end{equation}
where $\mathbf{E}$ is the Green Lagrangian strain tensor
\begin{equation}
	\mathbf{E} = \frac{1}{2}\rbrkt{
	\mathbf{F}^{\text{T}}\mathbf{F} - \mathbf{1}},
\end{equation}
where $\mathbf{F}$ is the deformation gradient
and $\mathvec{\mathsf{C}}$ is the isotropic tensor of 
elastic constants given by~\eqref{eq.modulus.isotropic}.

\subsubsection{3D input parameters}
An example of input parameters for three-dimensional analysis is shown 
below:
\begin{inputfile}
####### Kirchhoff-St.Venant
2     # material code
####### common material parameters
0.0    0.0    1.0
0      0.0
####### moduli
100.0 # Young's modulus
0.25  # Poisson's ratio
\end{inputfile}
After the common parameters described in 
Section~\ref{sect.material.solid.common}, the 
model requires specification of Young's modulus 
$E$ and Poisson's ratio $\nu$.

\subsubsection{2D input parameters}
An example of input parameters for two-dimensional analysis is shown 
below:
\begin{inputfile}
####### Kirchhoff-St.Venant
2     # material code
####### common material parameters
0.0    0.0    1.0
0      0.0
####### moduli
100.0 # Young's modulus
0.25  # Poisson's ratio
####### 2D parameters
1.0   # thickness
2     # constraint option
\end{inputfile}
After the common parameters~\ref{sect.material.solid.common}, the 
model requires specification of Young's modulus $E$ and Poisson's 
ratio $\nu$. The material thickness and 
constraint option from Table~\ref{tab.2Dconstraint} 
follow the 
parameters for three-dimensional analysis.

\subsubsection{Output values}
This constitutive model does not have any output values.

\subsection{Material 3: small deformation cubic}
\label{sect.mat.SSCubic}
The stress response of this model is 
governed by the strain energy function
\begin{equation}
\Phi\farg{\boldsymbol{\epsilon}}=
\frac{1}{2} \epsilon_{ij}\,\mathsf{C}_{ijkl}\,\epsilon_{kl},
\end{equation}
where $\boldsymbol{\epsilon}$ is the infinitesimal strain tensor
and $\mathvec{\mathsf{C}}$ 
is the tensor of elastic constants which may be
written as
\begin{equation}
\label{eq.modulus.cubic}
	\mathvec{\mathsf{C}} =
	\begin{bmatrix}
		\mathsf{C}_{11} & \mathsf{C}_{12} & \mathsf{C}_{12} & 0 & 0 & 0 \\
	    \mathsf{C}_{12} & \mathsf{C}_{11} & \mathsf{C}_{12} & 0 & 0 & 0 \\
		\mathsf{C}_{12} & \mathsf{C}_{12} & \mathsf{C}_{11} & 0 & 0 & 0 \\
		0 & 0 & 0 & \mathsf{C}_{44} & 0 & 0 \\
		0 & 0 & 0 & 0 & \mathsf{C}_{44} & 0 \\
		0 & 0 & 0 & 0 & 0 & \mathsf{C}_{44}
	\end{bmatrix}
\end{equation}
for the three-dimensional case. The elasticity tensor in~\eqref{eq.modulus.cubic}
has cubic symmetry. The degree of anisotropy is often characterized by
the parameter
\begin{equation}
A = \frac{2\,\mathsf{C}_{44}}{\mathsf{C}_{11}-\mathsf{C}_{12}},
\end{equation}
where $A = 1$ for isotropy.

\subsubsection{3D input parameters}
An example of input parameters for three-dimensional analysis is shown 
below:
\begin{inputfile}
####### cubic model
3     # material code
####### common material parameters
0.0    0.0    1.0
0      0.0
####### cubic moduli
100.0 # C11
 10.0 # C12
 50.0 # C44
\end{inputfile}
After the common parameters described in 
Section~\ref{sect.material.solid.common}, the 
model requires specification of the cubic moduli
$\mathsf{C}_{11}$, $\mathsf{C}_{12}$, and
$\mathsf{C}_{44}$.

\subsubsection{2D input parameters}
An example of input parameters for two-dimensional analysis is shown 
below:
\begin{inputfile}
####### cubic model
3     # material code
####### common material parameters
0.0    0.0    1.0
0      0.0
####### cubic moduli
100.0 # C11
 10.0 # C12
 50.0 # C44
####### orientation
0     # is rotated
0.0   # rotation angle 
####### 2D parameters
1.0   # thickness
2     # constraint option
\end{inputfile}
After the common parameters described in 
Section~\ref{sect.material.solid.common}, the 
model requires specification of the cubic moduli
$\mathsf{C}_{11}$, $\mathsf{C}_{12}$, and
$\mathsf{C}_{44}$.
The cubic moduli are followed by specification of the lattice 
rotation, consisting of a boolean flag indicating whether the lattice 
is rotated and the angle rotation.
The material thickness and 
constraint option from Table~\ref{tab.2Dconstraint} 
follow the parameters for three-dimensional analysis.

\subsubsection{Output values}
This constitutive model does not have any output values.

\subsection{Material 4: finite deformation cubic}
\label{sect.mat.FDCubic}
This model is the finite deformation version of the material model
described in Section~\ref{sect.mat.SSCubic}.
The stress response of this isotropic model is 
governed by the strain energy function
\begin{equation}
\Phi\farg{\mathbf{E}}=
\frac{1}{2} E_{ij}\,\mathsf{C}_{ijkl}\,E_{kl},
\end{equation}
where $\mathbf{E}$ is the Green Lagrangian strain tensor
\begin{equation}
	\mathbf{E} = \frac{1}{2}\rbrkt{
	\mathbf{F}^{\text{T}}\mathbf{F} - \mathbf{1}},
\end{equation}
where $\mathbf{F}$ is the deformation gradient
and $\mathvec{\mathsf{C}}$ is the tensor of 
elastic constants given by~\eqref{eq.modulus.cubic}.

\subsubsection{3D input parameters}
An example of input parameters for three-dimensional analysis is shown 
below:
\begin{inputfile}
####### cubic model
4     # material code
####### common material parameters
0.0    0.0    1.0
0      0.0
####### cubic moduli
100.0 # C11
 10.0 # C12
 50.0 # C44
\end{inputfile}
After the common parameters described in 
Section~\ref{sect.material.solid.common}, the 
model requires specification of the cubic moduli
$\mathsf{C}_{11}$, $\mathsf{C}_{12}$, and
$\mathsf{C}_{44}$.

\subsubsection{2D input parameters}
An example of input parameters for two-dimensional analysis is shown 
below:
\begin{inputfile}
####### cubic model
4     # material code
####### common material parameters
0.0    0.0    1.0
0      0.0
####### cubic moduli
100.0 # C11
 10.0 # C12
 50.0 # C44
####### orientation
0     # is rotated
0.0   # rotation angle 
####### 2D parameters
1.0   # thickness
2     # constraint option
\end{inputfile}
After the common parameters described in 
Section~\ref{sect.material.solid.common}, the 
model requires specification of the cubic moduli
$\mathsf{C}_{11}$, $\mathsf{C}_{12}$, and
$\mathsf{C}_{44}$.
The cubic moduli are followed by specification of the lattice 
rotation, consisting of a boolean flag indicating whether the lattice 
is rotated and the angle rotation.
The material thickness and 
constraint option from Table~\ref{tab.2Dconstraint} 
follow the parameters for three-dimensional analysis.

\subsubsection{Output values}
This constitutive model does not have any output values.

\subsection{Material 5: uncoupled volumetric/deviatoric}
\label{sect.mat.SimoIso}
This model is the finite deformation, isotropic
constitutive model proposed by 
Simo \etal\cite{Simo1985} for
which the response to volumetric and deviatoric
deformations is uncoupled.
The hyperelastic stress response is derived from the
stored energy function
\begin{align}
\Phi &= U\farg{J} + \bar{\Phi}\farg{\bar{\mathbf{b}}} 
\label{Simo.iso.1}\\
U\farg{J} &= \frac{1}{2}\,\kappa 
    \sbrkt{
        \frac{1}{2} \rbrkt{J^{2}-1}-\ln J 
    } \label{Simo.iso.2}\\
\bar{\Phi}\farg{\bar{\mathbf{b}}} &=
    \frac{1}{2}\,\mu \rbrkt{\text{tr}\,\bar{\mathbf{b}} - 
    3}\label{Simo.iso.3},
\end{align}
where $\kappa$ is the bulk modulus, $\mu$ is the shear modulus,
$\bar{\mathbf{b}}=J^{-2/3}\mathbf{b}$ is the deviatoric part of the
left Cauchy-Green stretch tensor
\begin{equation}
	\mathbf{b} = \mathbf{F}\,\mathbf{F}^{\text{T}},
\end {equation}
and $J = \det \mathbf{F}$. As a result of the form of 
the volumetric portion 
of the stored energy function
$U\farg{J}$~\eqref{Simo.iso.2}, this model is 
stable under severe compression.

The bulk and shear moduli are related to Young's modulus $E$
and Poisson's ratio $\nu$ by
\begin{equation}
\label{eq.kappa}
\kappa = \frac{E}{3 \rbrkt{1 - 2\,\nu}}
\end{equation}
and
\begin{equation}
\mu = \frac{E}{2 \rbrkt{1 + \nu}}.
\tag{\ref{eq.mu}}
\end{equation}

\subsubsection{3D input parameters}
An example of input parameters for three-dimensional analysis is shown 
below:
\begin{inputfile}
####### uncoupled volumetric/deviatoric model
5     # material code
####### common material parameters
0.0    0.0    1.0
0      0.0
####### moduli
100.0 # Young's modulus
0.25  # Poisson's ratio
\end{inputfile}
After the common parameters described in 
Section~\ref{sect.material.solid.common}, the 
model requires specification of Young's modulus 
$E$ and Poisson's ratio $\nu$.

\subsubsection{2D input parameters}
An example of input parameters for two-dimensional analysis is shown 
below:
\begin{inputfile}
####### uncoupled volumetric/deviatoric model
5     # material code
####### common material parameters
0.0    0.0    1.0
0      0.0
####### moduli
100.0 # Young's modulus
0.25  # Poisson's ratio
####### 2D parameters
1.0   # thickness
2     # constraint option (override)
\end{inputfile}
After the common parameters~\ref{sect.material.solid.common}, the 
model requires specification of Young's modulus $E$ and Poisson's 
ratio $\nu$. The material thickness and 
constraint option from Table~\ref{tab.2Dconstraint} 
follow the parameters for three-dimensional analysis.
This two-dimensional model is formulated in plane strain,
so other values for the constraint option are silently
overridden.

\subsubsection{Output values}
This constitutive model does not have any output values.

\subsection{Material 6: quadratic-logarithmic (Simo)}
\label{sect.mat.QuadLog.Simo}
This model is the finite deformation, isotropic
constitutive model proposed by 
Simo \etal\cite{Simo1992} expressed in
principal stretch space.
The hyperelastic stress response is derived from the
stored energy function expressed in terms of the
logarithms of the principal stretches as
\begin{equation}
\label{eq.quad.log}
\Phi\farg{e_{1}, e_{2}, e_{3}}=
  \frac{1}{2}\,\lambda \rbrkt{e_{1} + e_{2} + e_{3}}^{2} +
  \mu \rbrkt{e_{1}^{2} + e_{2}^{2} + e_{3}^{2}},
\end{equation}
where
\begin{equation}
	e_{A} = \log{\lambda_{A}}, \qquad A = 1, 2, 3.
\end{equation}
The Lam\a'e constants $\lambda$ and 
$\mu$ are related to Young's modulus $E$ 
and Poisson's ratio $\nu$ by
\begin{equation}
\tag{\ref{eq.lambda}}
\lambda = \frac{\nu\,E}{\rbrkt{1 + \nu}\rbrkt{1 - 2\,\nu}}
\end{equation}
and
\begin{equation}
\tag{\ref{eq.mu}}
\mu = \frac{E}{2 \rbrkt{1 + \nu}}.
\end{equation}
The implementation of the model follows the spectral formulation
developed by Simo and Taylor~\cite{Simo1991}. Notably, the principal
values of the Kirchhoff stress are linearly related to the logarithmic 
strains
\begin{equation}
\label{eq.quad.log.elastic}
\cbrkt{
\begin{matrix}
\tau_{1} \\
\tau_{2} \\
\tau_{3}
\end{matrix}} = 
\begin{bmatrix}
\kappa + \frac{4}{3} \mu & \kappa - \frac{2}{3} \mu & \kappa - \frac{2}{3} \mu \\
\kappa - \frac{2}{3} \mu & \kappa + \frac{4}{3} \mu & \kappa - \frac{2}{3} \mu \\
\kappa - \frac{2}{3} \mu & \kappa - \frac{2}{3} \mu & \kappa + \frac{4}{3} \mu
\end{bmatrix}
\cbrkt{
\begin{matrix}
e_{1} \\
e_{2} \\
e_{3}
\end{matrix}}
\end{equation}
for elastic deformations. As a result of~\eqref{eq.quad.log.elastic},
the return mapping algorithm is readily transferred to stress space,
leading to a highly efficient numerical implementation.

The model described in
Section~\ref{sect.mat.QuadLog.Ogden} uses the stored energy 
function~\eqref{eq.quad.log} with the spectral
formulation developed by Ogden~\cite{Ogden1983}. The Ogden formulation 
allows more straightforward handling of repeated principal stretches.

\subsubsection{3D input parameters}
An example of input parameters for three-dimensional analysis is shown 
below:
\begin{inputfile}
####### quad-log (Simo)
6     # material code
####### common material parameters
0.0    0.0    1.0
0      0.0
####### moduli
100.0 # Young's modulus
0.25  # Poisson's ratio
\end{inputfile}
After the common parameters described in 
Section~\ref{sect.material.solid.common}, the 
model requires specification of Young's modulus 
$E$ and Poisson's ratio $\nu$.

\subsubsection{2D input parameters}
An example of input parameters for two-dimensional analysis is shown 
below:
\begin{inputfile}
####### quad-log (Simo)
6     # material code
####### common material parameters
0.0    0.0    1.0
0      0.0
####### moduli
100.0 # Young's modulus
0.25  # Poisson's ratio
####### 2D parameters
1.0   # thickness
2     # constraint option (override)
\end{inputfile}
After the common parameters~\ref{sect.material.solid.common}, the 
model requires specification of Young's modulus $E$ and Poisson's 
ratio $\nu$. The material thickness and 
constraint option from Table~\ref{tab.2Dconstraint} 
follow the parameters for three-dimensional analysis.
This two-dimensional model is formulated in plane strain,
so other values for the constraint option are silently
overridden.

\subsubsection{Output values}
This constitutive model does not have any output values.

\subsection{Material 7: quadratic-logarithmic (Ogden)}
\label{sect.mat.QuadLog.Ogden}
This model is the finite deformation, isotropic
constitutive model proposed by 
Simo \etal\cite{Simo1992} expressed in
principal stretch space.
The hyperelastic stress response is derived from the
stored energy function given by~\eqref{eq.quad.log}, 
expressed in terms of the
logarithms of the principal stretches.
This model differs from the model described in 
Section~\ref{sect.mat.QuadLog.Ogden} only in the underlying spectral
formulation. Instead of the formulation proposed by
Simo and Taylor~\cite{Simo1991}, this model use the spectral
formulation developed by Ogden~\cite{Ogden1983}.
This formulation allows more straightforward handling of 
repeated principal stretches.

\subsubsection{3D input parameters}
An example of input parameters for three-dimensional analysis is shown 
below:
\begin{inputfile}
####### quad-log (Ogden)
7     # material code
####### common material parameters
0.0    0.0    1.0
0      0.0
####### moduli
100.0 # Young's modulus
0.25  # Poisson's ratio
\end{inputfile}
After the common parameters described in 
Section~\ref{sect.material.solid.common}, the 
model requires specification of Young's modulus 
$E$ and Poisson's ratio $\nu$.

\subsubsection{2D input parameters}
An example of input parameters for two-dimensional analysis is shown 
below:
\begin{inputfile}
####### quad-log (Ogden)
7     # material code
####### common material parameters
0.0    0.0    1.0
0      0.0
####### 2D parameters
1.0   # thickness
2     # constraint option (override)
####### moduli
100.0 # Young's modulus
0.25  # Poisson's ratio
\end{inputfile}
The material thickness and 
constraint option from Table~\ref{tab.2Dconstraint} 
follow the the common 
parameters~\ref{sect.material.solid.common}.
Finally, the model requires specification of 
Young's modulus $E$ and Poisson's ratio $\nu$. 
This two-dimensional model is formulated in plane strain,
so other values for the constraint option are silently
overridden.

\subsubsection{Output values}
This constitutive model does not have any output values.

\subsection{Material 8: small deformation $J_{2}$ plasticity}
\label{sect.mat.SS.J2.LinearHardening}
This model combines the isotropic stored energy function from 
Section~\ref{sect.mat.SSKStV} with a $J_{2}$ yield
condition incorporating linear kinematic and 
isotropic hardening for infinitesimal strain analysis. 
This model is due to Simo and Hughes~\cite{Simo1998}. 
The yield condition is formulated as
\begin{equation}
	f\farg{\boldsymbol{\sigma}, \mathbf{q}} := 
	||\boldsymbol{\eta}|| 
	- \sqrt{\frac{2}{3}} K\farg{\alpha} \le 0,
\end{equation}
where the internal variables 
$\mathbf{q} = \cbrkt{\alpha, \bar{\boldsymbol{\beta}}}$ are
the equivalent plastic strain $\alpha$ and the center of
the yield surface $\bar{\boldsymbol{\beta}}$ in deviatoric
stress space. The relative stress $\boldsymbol{\eta}$ is
\begin{equation}
\boldsymbol{\eta} = \textrm{dev}\sbrkt{\boldsymbol\sigma} - 
\bar{\boldsymbol{\beta}},
\end{equation}
where $\boldsymbol{\sigma}$ is the Cauchy stress. The associative flow rule
and hardening laws are
\begin{align}
\Dot{\boldsymbol{\epsilon}}^{p} &= \gamma\,
   \frac{\boldsymbol{\eta}}{||\boldsymbol{\eta}||}, \\
\Dot{\Bar{\boldsymbol{\beta}}} &= \gamma\, \frac{2}{3} H'\farg{\alpha}
   \frac{\boldsymbol{\eta}}{||\boldsymbol{\eta}||}, \\
\Dot{\alpha} &= \gamma\,\sqrt{\frac{2}{3}},
\end{align}
where $\gamma$ is the consistency parameter from the Kuhn-Tucker
complementary conditions
\begin{equation}
\label{eq.Kuhn-Tucker.1}
\gamma \ge 0, \qquad f\farg{\boldsymbol{\sigma}, \mathbf{q}} \le 0,
\end{equation}
and
\begin{equation}
\label{eq.Kuhn-Tucker.2}
\gamma\,f\farg{\boldsymbol{\sigma}, \mathbf{q}} = 0.
\end{equation}
The combined linear kinematic and isotropic hardening functions are
defined as
\begin{equation}
	H\farg{\alpha} = \rbrkt{1 - \theta} \bar{H} \alpha
\end{equation}
and
\begin{equation}
	K\farg{\alpha} = \sigma_{y} + \theta \bar{H} \alpha,
\end{equation}
where $\sigma_{y}$ is the initial yield stress and
$\bar{H}$ is the single hardening modulus.
The parameter $0 \le \theta \le 1$ determines the mixity of 
isotropic to kinematic hardening, with $\theta = 1$ indicating purely 
isotropic hardening.

\subsubsection{3D input parameters}
An example of input parameters for three-dimensional analysis is shown 
below:
\begin{inputfile}
####### small strain J2 plasticity
8     # material code
####### common material parameters
0.0    0.0    1.0
0      0.0
####### moduli
100.0 # Young's modulus
0.25  # Poisson's ratio
####### plasticity
0     # isotropic hardening function (0:linear)
0.25  # yield stress
0.01  # hardening parameter
\end{inputfile}
After the common parameters described in 
Section~\ref{sect.material.solid.common}, the 
model requires specification of Young's modulus 
$E$ and Poisson's ratio $\nu$. Finally, the user defines the specification for the isotropic hardening function $K\farg{\alpha}$, the initial 
yield stress $\sigma_{y}$, and the hardening parameter $\bar{H}$. All available hardening functions are listed in Table~\ref{tab.hardening.functions.iso}. An example of implementing the general cubic spline is detailed in Section~\ref{sect.mat.FD.J2.C0.Hardening}. 

\subsubsection{2D input parameters}
An example of input parameters for two-dimensional analysis is shown 
below:
\begin{inputfile}
####### small strain J2 plasticity
8     # material code
####### common material parameters
0.0    0.0    1.0
0      0.0
####### moduli
100.0 # Young's modulus
0.25  # Poisson's ratio
####### plasticity
0     # isotropic hardening function
0.25  # yield stress
0.01  # hardening parameter
####### 2D parameters
1.0   # thickness
2     # constraint option (override)
\end{inputfile}
After the common parameters described in 
Section~\ref{sect.material.solid.common}, the 
model requires specification of Young's modulus 
$E$ and Poisson's ratio $\nu$. Following these parameters, 
the user defines the specification for the isotropic hardening function $K\farg{\alpha}$, the initial 
yield stress $\sigma_{y}$, and the hardening parameter $\bar{H}$. All available hardening functions are listed in Table~\ref{tab.hardening.functions.iso}. An example of implementing the general cubic spline is detailed in Section~\ref{sect.mat.FD.J2.C0.Hardening}. 
The material thickness and 
constraint option from Table~\ref{tab.2Dconstraint} 
follow the parameters for three-dimensional analysis.
This two-dimensional model is formulated in plane strain,
so other values for the constraint option are silently
overridden.

\subsubsection{Output values}
This constitutive model has three scalar output values listed in
Table~\ref{tab.mat.SS.J2.LinearHardening.output}.
\begin{table}[h]
\caption{\label{tab.mat.SS.J2.LinearHardening.output} Output values 
for a small strain $J_2$ plasticity model.}
\begin{center}
\begin{tabular}[c]{|l|c|c|}
\hline
 \parbox[c]{0.75in}{\centering \textbf{var.}}
&\parbox[c]{3.0in}{\raggedright \textbf{description}}
&\parbox[c]{1.0in}{\raggedright  \textbf{labels}}\\
\hline
\parbox[c]{0.75in}{\centering $\alpha$} & 
\parbox[c]{3.0in}{\raggedright equivalent plastic strain} &
\parbox[c]{1.0in}{\raggedright \texttt{alpha}}\\
\hline
\parbox[c]{0.75in}{\centering $||\textrm{dev}\sbrkt{\boldsymbol{\sigma}}||$}  & 
\parbox[c]{3.0in}{\raggedright second invariant of the deviatoric stress} &
\parbox[c]{1.0in}{\raggedright \texttt{VM}}\\
\hline
\parbox[c]{0.75in}{\centering $p$}  & 
\parbox[c]{3.0in}{\raggedright pressure} &
\parbox[c]{1.0in}{\raggedright \texttt{press}}\\
\hline
\end{tabular}
\end{center}
\end{table}

\subsection{Material 9: finite deformation $J_{2}$ plasticity}
\label{sect.mat.FD.J2.C0.Hardening}
This model is based on a formulation due to Simo~\cite{Simo1988a,Simo1988b}
for finite strain $J_{2}$ plasticity. The formulation includes
both linear isotropic and kinematic hardening; however, the 
implementation allows specification of an arbitrary isotropic hardening 
function, but does not include kinematic hardening.
The isotropic stored energy function for this model is taken from the 
model described in Section~\ref{sect.mat.FDKStV}. 
The yield condition is formulated in terms of the
Kirchhoff stress $\boldsymbol{\tau}$ as
\begin{equation}
	f\farg{\boldsymbol{\tau}, \mathbf{q}} := 
	||\bar{\boldsymbol{\eta}}|| 
	- \sqrt{\frac{2}{3}} K\farg{\alpha} \le 0,
\end{equation}
where the internal variables 
$\mathbf{q} = \cbrkt{\alpha, \bar{\boldsymbol{\beta}}}$ are
the equivalent plastic strain $\alpha$ and the center of
the yield surface $\bar{\boldsymbol{\beta}}$ in deviatoric
Kirchhoff stress space. The deviatoric relative stress 
$\bar{\boldsymbol{\eta}}$ is
\begin{equation}
\bar{\boldsymbol{\eta}} = \textrm{dev}\sbrkt{\boldsymbol\tau} - 
\bar{\boldsymbol{\beta}}.
\end{equation}
As in the small strain case, the hardening variable is assumed
to evolve as
\begin{equation}
\dot{\alpha} = \sqrt{\frac{2}{3}}\,\gamma,	
\end{equation}
where $\gamma$ is the consistency parameter 
in~\eqref{eq.Kuhn-Tucker.1} and~\eqref{eq.Kuhn-Tucker.2}.
The flow rule rule and kinematic hardening law are derived
assuming a multiplicative split of the deformation gradient
as $\mathbf{F} = \mathbf{F}^{\textrm{e}} \mathbf{F}^{\textrm{p}}$.
The isotropic hardening functions available for this model are 
described in Section~\ref{sect.hardening.functions}.

NOTE: The implementation of the consistent tangent moduli for this model 
has not been verified. The moduli are not used in the stress-update 
equations, but do affect the rate of convergence for Newton-type 
solvers. Finite strain, $J_{2}$ plasticity is also implemented in the model
described in Section~\ref{sect.mat.J2.QL.LinearHardening}.

\subsubsection{3D input parameters}
An example of input parameters for two-dimensional analysis is shown 
below:
\begin{inputfile}
####### finite strain J2 plasticity
 9     # material code
####### common material parameters
0.0    0.0    1.0
0      0.0
####### moduli
100.0 # Young's modulus
0.25  # Poisson's ratio
####### plasticity
2     # isotropic hardening function
####### spline hardening function
4     # number of points
# [alpha] [K(alpha)]
0.00  0.25
0.01  0.255
0.05  0.26
0.10  0.30
\end{inputfile}
After the common parameters described in 
Section~\ref{sect.material.solid.common}, the 
model requires specification of Young's modulus 
$E$ and Poisson's ratio $\nu$. 
The moduli are followed by the specification of the isotropic 
hardening function $K\farg{\alpha}$. In the example, the function specification
\texttt{2} indicates the general 
cubic spline. All available hardening functions are listed in 
Table~\ref{tab.hardening.functions.iso}.
The points defining the spline follow.

\subsubsection{2D input parameters}
An example of input parameters for two-dimensional analysis is shown 
below:
\begin{inputfile}
####### finite strain J2 plasticity
9     # material code
####### common material parameters
0.0    0.0    1.0
0      0.0
####### moduli
100.0 # Young's modulus
0.25  # Poisson's ratio
####### 2D parameters
1.0   # thickness
2     # constraint option (override)
####### plasticity
2     # isotropic hardening function
####### spline hardening function
4     # number of points
# [alpha] [K(alpha)]
0.00  0.25
0.01  0.255
0.05  0.26
0.10  0.30
\end{inputfile}
After the common parameters described in 
Section~\ref{sect.material.solid.common}, the 
model requires specification of Young's modulus 
$E$ and Poisson's ratio $\nu$. 
The material thickness and 
constraint option from Table~\ref{tab.2Dconstraint} 
follow the parameters for three-dimensional analysis.
This two-dimensional model is formulated in plane strain,
so other values for the constraint option are silently
overridden.
The parameters for 2D analysis 
are followed by the specification of the isotropic 
hardening function $K\farg{\alpha}$. In the example, the function specification
\texttt{2} indicates the general 
cubic spline. All available hardening functions are listed in 
Table~\ref{tab.hardening.functions.iso}.
The points defining the spline follow.

\subsubsection{Output values}
This constitutive model has three scalar output values listed in
Table~\ref{tab.mat.FD.J2.C0.Hardening.output}.
\begin{table}[h]
\caption{\label{tab.mat.FD.J2.C0.Hardening.output} Output values 
for a small strain $J_2$ plasticity model.}
\begin{center}
\begin{tabular}[c]{|l|c|c|}
\hline
 \parbox[c]{1.00in}{\centering \textbf{var.}}
&\parbox[c]{3.0in}{\raggedright \textbf{description}}
&\parbox[c]{1.0in}{\raggedright  \textbf{labels}}\\
\hline
\parbox[c]{1.00in}{\centering $\alpha$} & 
\parbox[c]{3.0in}{\raggedright equivalent plastic strain} &
\parbox[c]{1.0in}{\raggedright \texttt{alpha}}\\
\hline
\parbox[c]{1.00in}{\centering $||\bar{\boldsymbol{\beta}}||$} & 
\parbox[c]{3.0in}{\raggedright norm of kinematic hardening} &
\parbox[c]{1.0in}{\raggedright \texttt{norm\_beta}}\\
\hline
\parbox[c]{1.00in}{\centering 
$\sqrt{\frac{3}{2}}||\textrm{dev}\sbrkt{\boldsymbol{\tau}}||$}  & 
\parbox[c]{3.0in}{\raggedright second invariant of the deviatoric stress} &
\parbox[c]{1.0in}{\raggedright \texttt{VM\_Kirch}}\\
\hline
\parbox[c]{1.00in}{\centering $p$}  & 
\parbox[c]{3.0in}{\raggedright pressure} &
\parbox[c]{1.0in}{\raggedright \texttt{press}}\\
\hline
\end{tabular}
\end{center}
\end{table}

\subsubsection{Hardening functions}
\label{sect.hardening.functions}
The isotropic hardening function for this model is specified by the 
user. The functions available are listed 
in Table~\ref{tab.hardening.functions.iso}.
\begin{table}[h]
\caption{\label{tab.hardening.functions.iso} Isotropic hardening 
functions.}
\begin{center}
\begin{tabular}[c]{|l|c|c|}
\hline
 \parbox[b]{0.75in}{\centering \textbf{code}}
&\parbox[b]{2.5in}{\raggedright \textbf{description}}\\
\hline
\parbox[b]{0.75in}{\centering 0} & 
\parbox[b]{2.5in}{\raggedright linear}\\
\hline
\parbox[b]{0.75in}{\centering 1} & 
\parbox[b]{2.5in}{\raggedright linear-exponential saturation}\\
\hline
\parbox[b]{0.75in}{\centering 2}  & 
\parbox[b]{2.5in}{\raggedright general cubic spline}\\
\hline
\end{tabular}
\end{center}
\end{table}
The input parameters for each function are described in the sections 
that follow. For the 
nonlinear hardening functions, it is possible to specify the function 
parameters to produce hardening behavior that is not strictly convex. 
The local Newton iteration used to compute the return mapping during 
plastic loading is not guaranteed to converge under these conditions.

\paragraph{linear}
The linear hardening function has the form
\begin{equation}
K\farg{\alpha} = \sigma_{y} + K\,\alpha,
\end{equation}
where $\sigma_{y}$ is the initial yield stress and $K$ is the 
hardening modulus. An example of input parameters is shown below:
\begin{inputfile}
####### linear hardening function
0.25  # initial yield stress
0.01  # hardening modulus
\end{inputfile}	

\paragraph{linear-exponential saturation}
The linear-exponential saturation hardening function has the form
\begin{equation}
K\farg{\alpha} = \sigma_{y} + K\,\alpha 
+ \Delta K_{\infty} \rbrkt{1 - \exp \sbrkt{-\frac{\alpha}{\delta}}},
\end{equation}
where $\sigma_{y}$ is the initial yield stress and $K$ is the 
linear hardening modulus. The saturation hardening has a magnitude of
$\Delta K_{\infty}$ and occurs at a characteristic 
equivalent plastic strain of $\delta$. An example of input parameters is shown below:
\begin{inputfile}
####### linear-exponential saturation
0.25  # initial yield stress
0.10  # linear hardening modulus
0.05  # saturation hardening
0.05  # saturation strain
\end{inputfile}	

\paragraph{general cubic spline}
The cubic spline hardening function has the general form
\begin{equation}
K\farg{\alpha} =
\begin{cases}
K^{(1)}\farg{\alpha_{1}} + \rbrkt{\alpha - \alpha_{1}} {K^{(1)}}'\farg{\alpha_{1}}
& \textrm{if~} \alpha < \alpha_{1}, \\
K^{(i)}\farg{\alpha}
& \textrm{if~} \alpha_{i} < \alpha <  \alpha_{i+1} \textrm{~for~} i=1,\ldots,n-1, \\
K^{(n)}\farg{\alpha_{n}} + \rbrkt{\alpha - \alpha_{n}} {K^{(n)}}'\farg{\alpha_{n}}
& \textrm{if~}\alpha > \alpha_{n},
\end{cases}	
\end{equation}
where the $n$ is the number of points used to define the spline. Over 
each sub-interval, the spline takes the form
\begin{equation}
K^{(i)}\farg{\alpha} = a_{0}^{(i)} 
+ a_{1}^{(i)} \Delta \alpha 
+ a_{2}^{(i)} \Delta \alpha^{2} 
+ a_{3}^{(i)} \Delta \alpha^{3},
\end{equation}	
where $\Delta \alpha = \alpha - \alpha_{i}$. The coefficients 
$\mathbf{a}^{(i)}$ 
are computed from the spline points under the conditions that 
spline pass through the data and that the first derivative of the 
spline is continuous between sub-intervals. Additionally, the end 
conditions for the spline are specified as 
\begin{equation}
{K^{(1)}}''\farg{\alpha_{1}} = {K^{(n)}}''\farg{\alpha_{n}} = 0
\end{equation}
to match the linear extensions assumed beyond the range of the spline 
data. A specific example of input to define a hardening function is 
shown below:
\begin{inputfile}
####### spline hardening function
4     # number of spline points
# [alpha] [K(alpha)]
0.00  0.25
0.01  0.255
0.05  0.26
0.10  0.30	
\end{inputfile}
Generally, the input has the form
\begin{align}
& \texttt{\#\#\#\#\#\#\# spline hardening function} \notag \\
& \sbrkt{n_{pts} = \textit{number of spline points}} \notag \\
& \begin{matrix}
\sbrkt{\alpha_{1}} & \sbrkt{K\farg{\alpha_{1}}} \\
\vdots \\
\sbrkt{\alpha_{n_{pts}}} & \sbrkt{K\farg{\alpha_{n_{pts}}}}
\end{matrix} \notag
\end{align}
Although not required or enforced, one typically selects
$\alpha_{1} = 0$ in order to define the initial yield 
stress specifically with $K\farg{\alpha_{1}}$.

\subsection[Material 10: finite deformation $J_{2}$ plasticity in 
principal stretches]
{Material 10: finite deformation $J_{2}$ plasticity in 
principal\\ stretches}
\label{sect.mat.J2.QL.LinearHardening}
This model combines the isotropic stored energy function from 
Section~\ref{sect.mat.QuadLog.Simo} with
a $J_{2}$ yield condition incorporating
linear kinematic and 
isotropic hardening for finite strain analysis. 
This model is due to Simo~\cite{Simo1992}.
The yield condition is formulated in terms of the
Kirchhoff stress $\boldsymbol{\tau}$ as
\begin{equation}
	f\farg{\boldsymbol{\tau}, \mathbf{q}} := 
	||\bar{\boldsymbol{\eta}}|| 
	- \sqrt{\frac{2}{3}} K\farg{\alpha} \le 0,
\end{equation}
where the internal variables 
$\mathbf{q} = \cbrkt{\alpha, \bar{\boldsymbol{\beta}}}$ are
the equivalent plastic strain $\alpha$ and the center of
the yield surface $\bar{\boldsymbol{\beta}}$ in deviatoric
Kirchhoff stress space. The relative stress $\bar{\boldsymbol{\eta}}$ is
\begin{equation}
\bar{\boldsymbol{\eta}} = \textrm{dev}\sbrkt{\boldsymbol\tau} - 
\bar{\boldsymbol{\beta}}.
\end{equation}
As in the small strain case, the hardening variable is assumed
to evolve as
\begin{equation}
\dot{\alpha} = \sqrt{\frac{2}{3}}\,\gamma,	
\end{equation}
where $\gamma$ is the consistency parameter 
in~\eqref{eq.Kuhn-Tucker.1} and~\eqref{eq.Kuhn-Tucker.2}.
The flow rule rule and kinematic hardening law are derived
assuming a multiplicative split of the deformation gradient.
These will not be outlined here; however,
the linear kinematic and isotropic hardening functions
\begin{equation}
	H\farg{\alpha} = \rbrkt{1 - \theta} \bar{H} \alpha
\end{equation}
and
\begin{equation}
	K\farg{\alpha} = \tau_{y} + \theta \bar{H} \alpha
\end{equation}
functional analogously to the small strain model described
in Section~\ref{sect.mat.SS.J2.LinearHardening}.
$\tau_{y}$ is the initial Kirchhoff yield stress and
$\bar{H}$ is the hardening modulus.
The parameter $0 \le \theta \le 1$ determines the mixity of 
isotropic to kinematic hardening, with $\theta = 1$ indicating purely 
isotropic hardening.
This model is formulated in principal stretches which preserves the 
classical return mapping schemes from the infinitesimal theory. An 
important results of the spectral treatment is that the consistent 
tangent is symmetric even during plastic loading.

\subsubsection{3D input parameters}
An example of input parameters for three-dimensional analysis is shown 
below:
\begin{inputfile}
####### finite strain J2 plasticity
10    # material code
####### common material parameters
0.0    0.0    1.0
0      0.0
####### moduli
100.0 # Young's modulus
0.25  # Poisson's ratio
####### plasticity
0.25  # yield stress
0.01  # hardening parameter
1.0   # kinematic/isotropic mixity
\end{inputfile}
After the common parameters described in 
Section~\ref{sect.material.solid.common}, the 
model requires specification of Young's modulus 
$E$ and Poisson's ratio $\nu$. Finally, the user defines the initial 
yield stress $\sigma_{y}$, the hardening parameter $\bar{H}$, and the 
kinematic/isotropic hardening mixity parameter $\theta$, 
for which $\theta = 1$ produces purely isotropic hardening.

\subsubsection{2D input parameters}
An example of input parameters for two-dimensional analysis is shown 
below:
\begin{inputfile}
####### finite strain J2 plasticity
10    # material code
####### common material parameters
0.0    0.0    1.0
0      0.0
####### moduli
100.0 # Young's modulus
0.25  # Poisson's ratio
####### 2D parameters
1.0   # thickness
2     # constraint option (override)
####### plasticity
0.25  # yield stress
0.01  # hardening parameter
1.0   # kinematic/isotropic mixity
\end{inputfile}
After the common parameters described in 
Section~\ref{sect.material.solid.common}, the 
model requires specification of Young's modulus 
$E$ and Poisson's ratio $\nu$. 
The material thickness and 
constraint option from Table~\ref{tab.2Dconstraint} 
follow the parameters for three-dimensional analysis.
This two-dimensional model is formulated in plane strain,
so other values for the constraint option are silently
overridden.
Following these parameters, 
the user defines the initial 
yield stress $\sigma_{y}$, the hardening parameter $\bar{H}$, and the 
kinematic/isotropic hardening mixity parameter $\theta$, 
for which $\theta = 1$ produces purely isotropic hardening.

\subsubsection{Output values}
This constitutive model has five scalar output values listed in
Table~\ref{tab.mat.J2.QL.LinearHardening.output}.
\begin{table}[h]
\caption{\label{tab.mat.J2.QL.LinearHardening.output} Output values 
for a finite strain $J_2$ plasticity model.}
\begin{center}
\begin{tabular}[c]{|l|c|c|}
\hline
 \parbox[c]{0.75in}{\centering \textbf{var.}}
&\parbox[c]{3.0in}{\raggedright \textbf{description}}
&\parbox[c]{1.0in}{\raggedright  \textbf{labels}}\\
\hline
\parbox[c]{0.75in}{\centering $\alpha$} & 
\parbox[c]{3.0in}{\raggedright equivalent plastic strain} &
\parbox[c]{1.0in}{\raggedright \texttt{alpha}}\\
\hline
\parbox[c]{0.75in}{\centering $||\boldsymbol{\tau}||$}  & 
\parbox[c]{3.0in}{\raggedright second invariant of the Kirchhoff stress} &
\parbox[c]{1.0in}{\raggedright \texttt{VM\_Kirch}}\\
\hline
\parbox[c]{0.75in}{\centering $p$}  & 
\parbox[c]{3.0in}{\raggedright pressure} &
\parbox[c]{1.0in}{\raggedright \texttt{press}}\\
\hline
\parbox[c]{0.75in}{\centering $\sigma_{\textrm{max}}$}  & 
\parbox[c]{3.0in}{\raggedright maximum principal Cauchy stress} &
\parbox[c]{1.0in}{\raggedright \texttt{s\_max}}\\
\hline
\parbox[c]{0.75in}{\centering $\sigma_{\textrm{min}}$}  & 
\parbox[c]{3.0in}{\raggedright minimum principal Cauchy stress} &
\parbox[c]{1.0in}{\raggedright \texttt{s\_min}}\\
\hline
\end{tabular}
\end{center}
\end{table}

\subsection{Material 11: small deformation Drucker-Prager plasticity}
\label{sect.mat.SS.DP.LinearHardening}
This model combines the isotropic stored energy function from 
Section~\ref{sect.mat.SSKStV} with
a pressure-dependent yield condition incorporating
linear kinematic
hardening for infinitesimal strain analysis. 
The parameters for this model follow the presentation of
Regueiro and Borja~\cite{Regueiro1999}. 
The pressure-dependent yield condition is formulated as
\begin{equation}
	f\farg{\boldsymbol{\sigma}, \alpha_{1}, \alpha_{2}} := 
	\sqrt{\frac{3}{2}}\,||\textrm{dev}\sbrkt{\boldsymbol{\sigma}}|| 
	+ \sqrt{3} \rbrkt{\zeta + 
	\beta\,\frac{\textrm{tr}\sbrkt{\boldsymbol{\sigma}}}{3}} \le 0,
\end{equation}
where $\boldsymbol{\sigma}$ is the Cauchy stress and
\begin{equation}
\zeta = -\bar{\alpha} + b\,\alpha_{1} + \frac{\alpha_{2}}{\sqrt{3}}.
\end{equation}
The internal variables
\begin{equation}
\cbrkt{
\begin{matrix}
\alpha_{1} \\
\alpha_{2}
\end{matrix}} = -
\begin{bmatrix}
K' & 0 \\
0  & H'
\end{bmatrix}
\cbrkt{
\begin{matrix}
v^{p} \\
e^{p}
\end{matrix}}
\end{equation}
are related to the total volumetric and deviatoric plastic strains
$v^{p}$ and $e^{p}$, respectively, through the hardening moduli
$K'$ and $H'$. The material constants $\bar{\alpha}$ and $\beta$ may 
be defined in terms of the cohesive $\bar{c}$ and friction angle
$\bar{\phi}$ as
\begin{equation}
\bar{\alpha} = \frac{6\,\bar{c}\,\cos{\bar{\phi}}}
                    {\sqrt{3}\rbrkt{3+A\,\sin{\bar{\phi}}}}
\end{equation}
and
\begin{equation}
\beta = \frac{6\,\sin{\bar{\phi}}}
             {\sqrt{3}\rbrkt{3 + A\,\sin{\bar{\phi}}}},	
\end{equation}
where $-1 \le A \le 1$.
The plastic potential function is
\begin{equation}
\varphi\farg{\boldsymbol{\sigma}, \alpha_{1}, \alpha_{2}}=
\sqrt{\frac{3}{2}}\,||\textrm{dev} \sbrkt{\boldsymbol{\sigma}}|| +
\sqrt{3}\rbrkt{\zeta + b\,\frac{\textrm{tr}\sbrkt{\boldsymbol{\sigma}}}{3}}.
\end{equation}
Associative plasticity is recovered if $b=\beta$, where $b$ is the 
material dilation constant, and $J_{2}$ plasticity is recovered if
$b =\beta=0$.

\subsubsection{3D input parameters}
An example of input parameters for three-dimensional analysis is shown 
below:
\begin{inputfile}
####### small strain Drucker-Prager plasticity
11    # material code
####### common material parameters
0.0    0.0    1.0
0      0.0
####### moduli
100.0 # Young's modulus
0.25  # Poisson's ratio
####### plasticity
0.15  # cohesive-like strength parameter
0.04  # friction-like parameter
0.0	  # dilation parameter
0.0	  # deviatoric hardening parameter
0.0	  # volumetric hardening parameter
\end{inputfile}
After the common parameters described in 
Section~\ref{sect.material.solid.common}, the 
model requires specification of Young's modulus 
$E$ and Poisson's ratio $\nu$.
The cohesion-like strength parameter 
parameter $\bar{\alpha}$, the friction-like
parameter $\beta$, the dilatation
parameter $b$, and the hardening moduli follow.

\subsubsection{2D input parameters}
An example of input parameters for two-dimensional analysis is shown 
below:
\begin{inputfile}
####### small strain Drucker-Prager plasticity
11    # material code
####### common material parameters
0.0    0.0    1.0
0      0.0
####### moduli
100.0 # Young's modulus
0.25  # Poisson's ratio
####### plasticity
####### plasticity
0.15  # cohesive-like strength parameter
0.04  # friction-like parameter
0.0	  # dilation parameter
0.0	  # deviatoric hardening parameter
0.0	  # volumetric hardening parameter
####### 2D parameters
1.0   # thickness
2     # constraint option (override)
\end{inputfile}
After the common parameters described in 
Section~\ref{sect.material.solid.common}, the 
model requires specification of Young's modulus 
$E$ and Poisson's ratio $\nu$.
The cohesion-like strength parameter 
parameter $\bar{\alpha}$, the friction-like
parameter $\beta$, the dilatation
parameter $b$, and the hardening moduli follow.
The material thickness and 
constraint option from Table~\ref{tab.2Dconstraint} 
follow the parameters for three-dimensional analysis.
This two-dimensional model is formulated in plane strain,
so other values for the constraint option are silently
overridden.

\subsubsection{Output values}
This constitutive model has four scalar output values listed in
Table~\ref{tab.mat.DP.LinearHardening.output}.
\begin{table}[h]
\caption{\label{tab.mat.DP.LinearHardening.output} Output values 
for the small strain Drucker-Prager plasticity model.}
\begin{center}
\begin{tabular}[c]{|l|c|c|}
\hline
 \parbox[c]{0.75in}{\centering \textbf{var.}}
&\parbox[c]{3.0in}{\raggedright \textbf{description}}
&\parbox[c]{1.0in}{\raggedright  \textbf{labels}}\\
\hline
\parbox[c]{0.75in}{\centering $\alpha_{2}$} & 
\parbox[c]{3.0in}{\raggedright \vspace{2pt}
deviatoric part of stress-like plastic variable
\vspace{2pt}} &
\parbox[c]{1.0in}{\raggedright \texttt{alpha\_dev}}\\
\hline
\parbox[c]{0.75in}{\centering $\alpha_{1}$} & 
\parbox[c]{3.0in}{\raggedright \vspace{2pt}
volumetric part of stress-like plastic variable
\vspace{2pt}} &
\parbox[c]{1.0in}{\raggedright \texttt{alpha\_vol}}\\
\hline
\parbox[c]{0.75in}{\centering $||\textrm{dev}\sbrkt{\boldsymbol{\sigma}}||$}  & 
\parbox[c]{3.0in}{\raggedright second invariant of the deviatoric stress} &
\parbox[c]{1.0in}{\raggedright \texttt{VM}}\\
\hline
\parbox[c]{0.75in}{\centering $p$}  & 
\parbox[c]{3.0in}{\raggedright pressure} &
\parbox[c]{1.0in}{\raggedright \texttt{press}}\\
\hline
\end{tabular}
\end{center}
\end{table}

\subsection{Material 12: 2D hexagonal Lennard-Jones lattice}
\label{sect.mat.LJTr2D}
\label{sect.Cauchy-Born}
\begin{figure}[h]
\centerline{\includegraphics[scale = 1.0]{\dirfilepath{\figpath}{C-B.pdf}}}
\hangcaption
[The Cauchy-Born rule]
{Deformation of the underlying microstructure as
prescribed by the Cauchy-Born rule.\label{fig.CB}}
\end{figure}
\todo{this is a pure plane stress material}
The stress response of this material is calculated from an
underlying lattice structure and bonding potential by
making use of the Cauchy-Born rule.
The method was originally described 
by Born~\cite{Born1940} as a means for estimating the theoretical strength
of crystals and for assessing the stability of cubic crystal configurations
subject simple deformations.
The Cauchy-Born rule is a multiscale assumption about how the motion
of many atoms can be related to continuum deformation measures. Under
the assumption, the atoms in an crystal subject to a homogeneous
deformation move according to a single mapping from the undeformed
to the deformed configuration. From the continuum level, this mapping is
taken to be the deformation gradient $\mathbf{F}\farg{\mathbf{X},t}$. 
As shown in Figure~\ref{fig.CB}, the continuum region surrounding a point 
$\mathbf{X}$ in the undeformed configuration distorts as described by the 
deformation gradient. The coordinate of the continuum point
in the deformed configuration is denoted by $\mathbf{x}$.
At a microstructural length scale below the continuum level, we assume the
deformation of an underlying crystal lattice undergoes the same homogeneous
transformation. The deformation gradient at a point at the continuum length scale 
is assumed to be constant over a boundless crystal at the microstructural
length scale. The microstructural description of a crystalline material 
and the corresponding continuum constitutive properties are linked using an
equivalence in strain energy density. The continuum-level strain energy density 
is equated to the energy in a representative volume of the
microstructure. Since the deformation is homogeneous, the behavior of an
arbitrarily large crystal can be studied by considering only the representative
volume subject to periodic boundary conditions. For crystalline materials 
of a single species with a primitive unit cell, this representative volume is 
the atomic volume. The change in energy per unit volume for a given
deformation predicted by this procedure corresponds to the bulk behavior 
of the crystal at zero Kelvin.
There are no surface or temperature-dependent effects. The atoms
in the crystal interact as determined by assumed bonding potential functions. 
Taken from atomistic calculations, these semi-empirical
relationships may be in the form of pairwise potentials, 
multi-body potentials, or embedded atom method potentials. 
The number of bonds contributing to the 
energy in the representative volume depends on the range of 
influence of the bonding potentials.

For a lattice bound by pair potentials, the strain energy density is
\labeleq{eq.3.1}{
\Phi_2 \farg{ \mathbf{C} }=
{1 \over {\Omega _0}}\sum\limits_{i=1}^{n_b}
{U^{(i)} \farg{ {r^{(i)} \farg{ \mathbf{C} }} }},
}
where $\mathbf{C} = \mathbf{F}^{\mathrm{T}}\mathbf{F}$ 
is the right Cauchy-Green stretch tensor,
$n_b$ is the number of bonds, $\Omega_0$ is the representative volume, 
$r^{(i)}$ is the deformed bond length, and 
$U^{(i)}\farg{r}$ is a pairwise bond potential.
Typically, all the bonds are assumed to be governed by the same
potential function $U\farg{r}$. Since the deformation 
is assumed homogeneous over the crystal, 
the transformation defined by $\mathbf{F}$ can be applied to 
vectors of finite length. For each bond in the undeformed unit cell, 
we define a bond vector
\labeleq{eq.3.2}{
\mathbf{R} = R\,\boldsymbol{\Xi},
}
where $R$ and $\boldsymbol{\Xi}$ are the undeformed bond length and direction,
respectively. Assuming the deformation gradient $\mathbf{F}$ is
homogeneous over the representative volume,
we can express the deformed bond length as
\labeleq{eq.3.3}{
r\farg{\mathbf{C}} = R \,
\sqrt{\boldsymbol{\Xi} \cdot \mathbf{C} \, \boldsymbol{\Xi}}.
}
Given the strain energy density function~\eqref{eq.3.1}, we can derive a complete 
description of the stress response and elastic tangent moduli 
using Green elastic theory~\cite{Ogden1983}.

\begin{figure}[h]
\centerline{\includegraphics[scale = 1.0]{\dirfilepath{\figpath}{C-Bcell.pdf}}}
\hangcaption
[Deformation of a unit cell under the Cauchy-Born rule]
{Deformation of the unit cell of a two-dimensional
hexagonal lattice subject the Cauchy-Born rule.\label{fig.CBcell}}
\end{figure}
Figure~\ref{fig.CBcell} shows
the primitive unit cell of a two-dimensional, hexagonal lattice with nearest neighbor
bonding for which $n_b = 3$. In the undeformed configuration, the bond
vectors are given by
\begin{equation}
\mathbf{R}^{(1)} = 
a_0
\left\{
\begin{matrix}
-1/2 \\
-\sqrt{3}/2
\end{matrix}
\right\}
\quad
\mathbf{R}^{(2)} = 
a_0
\left\{ 
\begin{matrix}
1/2 \\
-\sqrt{3}/2
\end{matrix}
\right\} \quad
\mathbf{R}^{(3)} = 
a_0 \left\{ 
\begin{matrix}
1 \\
0
\end{matrix}
\right\},
\label{eq.hexabonds}
\end{equation}
where $a_0$ is the undeformed lattice parameter.

Following Green elastic theory, the Cauchy stress and spatial tangent modulus are given by
\begin{gather}
\boldsymbol{\sigma} =  \sigma \mathbf{1}, \\
\intertext{where the magnitude of the equibiaxial stress is}
\sigma = {3 \over 2}{a_0 \over \lambda \Omega_0} U'\farg{\lambda a_0}, \\
\intertext{and}
\mathvec{\mathsf{c}} = \mu \sbrkt{\mathbf{1} \otimes \mathbf{1} + 2\, \mathbf{I}}, \\
\intertext{where the instantaneous shear modulus is}
\mu = {3 \over 8\,\Omega_0}
   \sbrkt{a_0^2\,U''\farg{\lambda a_0} - 
        {a_0 \over \lambda}\,U'\farg{\lambda a_0}}.
   \label{eq.3.14}
\end{gather}
With instantaneous isotropic response under the restriction of Cauchy symmetry, 
the complete description of the modulus involves only the 
single parameter $\mu$ from~\eqref{eq.3.14}. Cauchy symmetry for this
two-dimensional model implies a fixed Poisson's ratio
$\nu = \frac{1}{3}$.

The earliest potentials used in lattice modeling were constructed to display the
essential features of cohesive interactions between atoms with parameters fitted to
available experimental data, such as lattice parameters and elastic properties
in the undeformed state. Born~\cite{Born1940} employed an ``inverse power'' 
potential of the form
\begin{equation}
\label{eq.3.15}
U\farg{r} = D {nm \over {n - m}}
  \sbrkt{{1 \over n}\rbrkt{a_0 \over r}^n - {1 \over m}\rbrkt{a_0 \over r}^m},
\end{equation}
where $n > m$, $D$, and $a_0$ were chosen to match experimental data. In particular, 
the well-known Lennard-Jones potential results from~\eqref{eq.3.15} with 
$m = 6$ and $n = 12$.

\subsubsection{3D input parameters}
This model is formulated in two-dimensions. Other models use the
Cauchy-Born rule to calculate the stress response of single crystals
in three dimensions.

\subsubsection{2D input parameters}
An example of input parameters for two-dimensional analysis is shown 
below:

\subsubsection{Output values}
This constitutive model does not have any output values.

\subsection{Material 13: plane strain FCC Lennard-Jones lattice}
\label{sect.mat.LJFCC111}
see~\ref{sect.Cauchy-Born}
With instantaneous isotropic response under the restriction of Cauchy symmetry, 
the complete description of the modulus involves only the 
single parameter $\mu$ from~\eqref{eq.3.14}. Cauchy symmetry for a
three-dimensional lattice implies a fixed Poisson's ratio
$\nu = \frac{1}{4}$.

\begin{table}[h]
\caption{\label{tab.FCC.orientations} FCC lattice orientation
options.}
\begin{center}
\begin{tabular}[c]{|l|c|c|}
\hline
 \parbox[b]{0.75in}{\centering \textbf{code}}
&\parbox[b]{2.0in}{\raggedright \textbf{orientation}}\\
\hline
\parbox[c]{0.75in}{\centering 0} & 
\parbox[c]{2.0in}{\raggedright
\vspace{2pt}
$\mathbf{e}_{1}$: [ 1\,0\,0 ] \\
$\mathbf{e}_{3}$: [ 0\,0\,1 ]
\vspace{2pt}
}\\
\hline
\parbox[c]{0.75in}{\centering 1} & 
\parbox[c]{2.0in}{\raggedright
\vspace{2pt}
$\mathbf{e}_{1}$: [ 1\,0\,$\bar{1}$ ] \\
$\mathbf{e}_{3}$: [ 1\,0\,1 ]
\vspace{2pt}
}\\
\hline
\parbox[c]{0.75in}{\centering 2}  & 
\parbox[c]{2.0in}{\raggedright
\vspace{2pt}
$\mathbf{e}_{1}$: [ 1\,0\,$\bar{1}$ ] \\
$\mathbf{e}_{3}$: [ 1\,1\,1 ]
\vspace{2pt}
}\\
\hline
\end{tabular}
\end{center}
\end{table}

\subsubsection{3D input parameters}
Although the underlying lattice for this material is three-dimensional,
it has only be implemented for plane strain deformation.
Other models use the
Cauchy-Born rule to calculate the stress response of single crystals
in three dimensions.

\subsubsection{2D input parameters}
An example of input parameters for two-dimensional analysis is shown 
below:

\subsubsection{Output values}
This constitutive model does not have any output values.

\subsection{Material 14: FCC lattice with EAM potentials}
\label{sect.mat.FCC.EAM}
The stress response of this model is calculated by applying the 
Cauchy-Born, described in Section~\ref{sect.Cauchy-Born},
rule together with embedded atom method potentials.
Application of the Cauchy-Born rule dates back to the work
of Born~\cite{Born1940}.
More recently, Tadmor \etal~\cite{Tadmor1996}, used the method
with the embedded atom method potentials as a means to derive 
nonlinear, hyperelastic constitutive models
for use in numerical simulations to predict stress and deformation.
The embedded atom method~\cite{Daw1984} (EAM) was developed in an attempt
to incorporate more physical understanding into the bond potential functions.
Originally motivated by the view of metals as nuclei in a sea of electrons, the 
total energy of any atom in the crystal is comprised of a contribution due to the nuclear 
repulsion and an embedding energy representing the attraction of the nucleus to the 
background electron density. In fitting a set of EAM potentials to a particular
material, one must define three functions: a pair potential describing nuclear interactions, 
the electron density distribution around each nucleus, and a function describing the
interaction between a nucleus and the background electron density. Although quite
different from simple pair interactions, EAM potentials may be incorporated
into the procedures for deriving constitutive properties
without any modifications to the assumptions of the Cauchy-Born rule.

Using EAM potentials, the strain energy density has two contributions,
\begin{equation}
\label{eq.3.17}
\Phi_{\mathrm{EAM}} = \Phi_2 + \Phi_e,
\end{equation}
where $\Phi_2$ is the contribution from the nuclear interactions given by the
pair potential expression~\eqref{eq.3.1} and $\Phi_e$ is the embedding energy. 
This additive split in the total
energy results in additive contributions to the stress response and tangent
moduli. Therefore, all expressions derived from~\eqref{eq.3.1} still apply,
and we only need to discuss the additional contribution due to the embedding
energy. The contribution to the total energy~\eqref{eq.3.17} due to the
embedding energy may be written as
\begin{equation}
\label{eq.3.18}
\Phi_e={1 \over {\Omega_0}} \, U_e \farg{\overline{\rho}_e}
\end{equation}
where $\overline{\rho}_{e}$ is the background electron density, that is,
the electron density at the nucleus of a particular atom 
resulting from all neighbors of the atom, 
but not the atom itself. As a first approximation, the electron density is assumed
to be centrosymmetric
\begin{equation}
\label{eq.3.19}
\overline{\rho}_e = \sum\limits_{i=1}^{n_b} \rho_{e} \farg{r^{(i)}}.
\end{equation}

\begin{table}[h]
\caption{\label{tab.EAM.glue} Embedded atom method potentials.}
\begin{center}
\begin{tabular}[c]{|l|c|c|}
\hline
 \parbox[b]{0.75in}{\centering \textbf{code}}
&\parbox[b]{3.0in}{\raggedright \textbf{potential}}\\
\hline
\parbox[c]{0.75in}{\centering 0} & 
\parbox[c]{3.0in}{\raggedright Al: Ercolessi and Adams~\cite{Ercolessi1994}}\\
\hline
\parbox[c]{0.75in}{\centering 1} & 
\parbox[c]{3.0in}{\raggedright Al: Voter and Chen~\cite{Voter1987}}\\
\hline
\parbox[c]{0.75in}{\centering 2}  & 
\parbox[c]{3.0in}{\raggedright Cu: Voter and Chen~\cite{Voter1994}}\\
\hline
\parbox[c]{0.75in}{\centering ~3{\scriptsize \dag}} & 
\parbox[c]{3.0in}{\raggedright Au: Foiles, Baskes, and Daw~\cite{FBD}}\\
\hline
\end{tabular}
\parbox[c]{4.0in}{\raggedright \vspace{2pt} \footnotesize \textsuperscript{\dag}
This potential requires the
file \texttt{auu3} with the potential data in \textsf{ParaDyn}~\cite{PARADYN} format.}
\end{center}
\end{table}

\todo{print tables of modulus values and units for each potential.}

\subsubsection{3D input parameters}
An example of input parameters for three-dimensional analysis is shown 
below:

\subsubsection{2D input parameters}
An example of input parameters for two-dimensional analysis is shown 
below:

\subsubsection{Output values}
This constitutive model does not have any output values.

\subsection{Material 15: Stillinger-Weber diamond cubic}
\label{sect.mat.DiamondCubic.SW}
The cubic unit cell for the diamond cubic structure is shown in 
Figure~\ref{fig.DCunitcell}.
\begin{figure}[h]
\centerline{\includegraphics[scale = 0.80]{\dirfilepath{\figpath}{DCunitcell.pdf}}}
\hangcaption
[Cubic unit cell of the diamond cubic lattice]
{Cubic unit cell of the diamond cubic lattice.\label{fig.DCunitcell}}
\end{figure}
The unit cell can be constructed from two interlaced face-centered cubic
lattices. The lattice $B$ in the figure is positioned relative to lattice
$A$ by a displacement of $\cbrkt{1/4,1/4,1/4}$ of the cube edge dimensions.
Stakgold\cite{Stakgold1950} discussed how such structure, which 
Wiener\cite{Weiner1983} called a composite lattice, should be treated, 
though he limits his discussion to pair potentials and his approach does not
strictly guarantee objectivity of the resulting strain energy density
function. In general terms, symmetry constraints require that each constituent
sub-lattice deforms homogeneously, but the sub-lattices can move relative to each other
by a rigid body translation which depends on the interatomic potentials. The rigid
body motion is determined by minimizing the energy of the crystal as a function
of the translational degrees of freedom. By enforcing this condition, static
equilibrium is satisfied for all atoms in the lattice.

Keeping in mind that the degrees of freedom will be introduced in a manner which 
maintains objectivity, we can express a modified strain energy density function
as
\labeleq{eq.3.35}{
\Phi \farg{\mathbf{C}} = 
   \check{\Phi} \farg{\mathbf{C},\boldsymbol{\Xi} \farg{\mathbf{C}}},
}
where $\boldsymbol{\Xi}$ represents a vector of all the degrees of freedom of
$\check{\Phi}$. The degrees of freedom are implicit functions of the stretch $\mathbf{C}$.
Given a strain energy density function of the form~\eqref{eq.3.35}, we can apply
the standard relations from Green elastic theory to derive expressions for the
stress response and tangent moduli. The internal degrees of freedom are
added as
\begin{equation}
\mathbf{x} = \mathbf{F} \rbrkt{\mathbf{X} + \boldsymbol{\Xi}}	
\end{equation}

Though perhaps not
the most accurate for all types of simulations, the original potentials
by Stillinger and Weber~\cite{Stillinger1985} 
are still highly attractive due to their
simplicity. With these potentials, the strain energy density is composed
of two contributions,
\labeleq{eq.3.45}{
\Phi_{\textup{SW}} = \Phi_2 + \Phi_3.
}
Here, $\Phi_2$ represents the contribution from standard inverse-power pair potentials,
and $\Phi_3$ represents the contribution from three-body
potentials which are needed to stabilize the tetrahedral structures that compose
the lattice. For a range of deformations around the
undeformed state, these potentials involve only nearest neighbor bonds.
In the original formulation\cite{Stillinger1985}, the two-body potential is
given by
\labeleq{eq.3.46}{
U_2 \farg{r} =
   \begin{cases}
   A \rbrkt{B\,r^{-p} - r^{-q}} \exp \sbrkt{\delta \over r - r_{cut}} &
   r < r_{cut} \\
   0 & r \geq r_{cut},
   \end{cases}
}
and the three-body potential is
\begin{multline} \label{eq.3.47}
U_3 \farg{r^{(ji)},r^{(jk)},\theta^{(ijk)}} \\
=
   \begin{cases}
   \lambda \rbrkt{{1 \over 3} + \cos \theta^{(ijk)}}^2 
   \exp \sbrkt{\gamma \over r^{(ji)} - r_{cut}}
   \exp \sbrkt{\gamma \over r^{(jk)} - r_{cut}} &
   r^{(ji)} , r^{(jk)} < r_{cut} \\
   0 &
   r^{(ji)} , r^{(jk)} \geq r_{cut}.
   \end{cases}
\end{multline}
Both potentials make 
use of an exponential cut-off term which causes the potentials, and
all derivatives of the potentials, to vanish smoothly when the atomic spacing $r$ 
reaches $r_{cut}$. This cut-off term is advantageous because it clearly defines the
distance over which atoms interact and provides a smooth termination for all the
potentials through to the second derivatives required to compute the moduli.
$U_3$~\eqref{eq.3.47} is clearly designed to favor the ``ideal'' tetrahedral angle of
$\theta_t = \cos^{-1} \farg{-1/3}$, though the functional
form is not motivated by any detailed attributes of the $sp^{3}$ orbitals which
produce this configuration. In terms of~\eqref{eq.3.46} and~\eqref{eq.3.47}, 
we can express the strain energy density~\eqref{eq.3.45} as
\labeleq{eq.3.48}{
\Phi_{\text{SW}} = {1 \over \Omega_0} \sbrkt{
\sum\limits_{(ij) = 1}^{8} \hspace{-2pt} U_2 \farg{r^{(ij)}} + 
\sum\limits_{(ijk) = 1}^{12} \hspace{-2pt} 
                      U_3 \farg{\tilde{\mathbf{r}}^{(ijk)}}},
}
where we represent the arguments to the three-body term as a vector composed of
the length of two bond vectors and the cosine of the angle they subtend,
\labeleq{eq.3.49}{
\tilde{\mathbf{r}}^{(ijk)} = \cbrkt{
\begin{matrix}
r^{(ji)} \\
r^{(jk)} \\
\cos \theta^{(ijk)}
\end{matrix}
  }.
}

\begin{table}[h]
\caption{\label{tab.DC.orientations} Diamond cubic lattice orientation
options.}
\begin{center}
\begin{tabular}[c]{|l|c|c|}
\hline
 \parbox[b]{0.75in}{\centering \textbf{code}}
&\parbox[b]{2.0in}{\raggedright \textbf{orientation}}\\
\hline
\parbox[c]{0.75in}{\centering 0} & 
\parbox[c]{2.0in}{\raggedright
\vspace{2pt}
$\mathbf{e}_{1}$: [ 1\,0\,0 ] \\
$\mathbf{e}_{3}$: [ 0\,0\,1 ]
\vspace{2pt}
}\\
\hline
\parbox[c]{0.75in}{\centering 1} & 
\parbox[c]{2.0in}{\raggedright
\vspace{2pt}
$\mathbf{e}_{1}$: [ 1\,0\,$\bar{1}$ ] \\
$\mathbf{e}_{3}$: [ 1\,0\,1 ]
\vspace{2pt}
}\\
\hline
\parbox[c]{0.75in}{\centering 2}  & 
\parbox[c]{2.0in}{\raggedright
\vspace{2pt}
$\mathbf{e}_{1}$: [ 1\,0\,$\bar{1}$ ] \\
$\mathbf{e}_{3}$: [ 1\,1\,1 ]
\vspace{2pt}
}\\
\hline
\end{tabular}
\end{center}
\end{table}

\todo{with and without internal equilibration.}

\subsubsection{3D input parameters}
An example of input parameters for three-dimensional analysis is shown 
below:

\subsubsection{2D input parameters}
An example of input parameters for two-dimensional analysis is shown 
below:

\subsubsection{Output values}
This constitutive model does not have any output values.

\subsection{Material 16: Virtual Internal Bond model}
\label{sect.mat.VIB}
The general features of the VIB model,
as well as more detailed study of its localization 
behavior under conditions of
dynamic crack propagation, are described
in greater detail 
elsewhere\cite{Gao1998,Klein1998,Klein1999,Klein2000}.
The VIB model is developed within the framework of hyperelasticity.
The current, or deformed, configuration of a body is described
as $\mathbf{x} = \boldsymbol{\varphi} \farg{\mathbf{X}}$, by
a mapping $\boldsymbol{\varphi}$ of the undeformed configuration
$\mathbf{X}$.
The arrangement of cohesive interactions among material particles 
is described by a spatial bond density function. The strain energy 
density is computed 
by integrating the bond density in space in a continuous analog to the sum
over discrete lattice neighbors for the case of crystalline materials. 
The VIB form of the
strain energy density function is
\labeleq{eq.4.1}{
\Phi ={1 \over {\Omega _0}}\int \limits_{\Omega_0^{*}} 
{U \farg{l}} \, 
D_\Omega \,d\Omega,
}
where $\Omega_{0}$ is the undeformed representative volume, 
$l$ is the deformed virtual bond length, $U \farg{l}$ is 
the bonding potential, $D_{\Omega}$ is the volumetric bond density function,
and $\Omega_0^{*}$ is the integration volume defined by the range of influence
of $U$.

\todo{show results for the fracture energy, cohesive strength, and 
modulus in terms of potential, and then show specific results for the
Smith-Ferrante potential. Is the potential selectable?}

Unlike constitutive models based on the
Cauchy-Born rule for crystalline
materials described in Section~\ref{sect.mat.FCC.EAM}, 
the VIB model may suffer significant numerical errors
when evaluating integrals of the bond
distribution to determine the strain energy density~\eqref{eq.4.1}
and the resulting stress response and moduli.
Many different schemes could be developed for evaluating the VIB
model integrals. As with any numerical procedure, the optimal
scheme represents a compromise between accuracy and speed
of execution. The primary concern with the implementation of the
isotropic VIB model is that the response in numerical calculations 
is indeed lacking in any orientational dependence resulting from 
the integration procedure.
For the plane stress, isotropic
VIB model, the integral of the volumetric bond density 
function reduces to an integral over the interval $-\pi \le \phi \le \pi$.
A simple scheme for evaluating these integrals is shown in 
Figure~\ref{fig.VIB2Derrorplot}.
\begin{figure}[h]
\centerline{\includegraphics[scale = 1.0]
{\dirfilepath{\figpath}{VIB2Derrorplot.pdf}}}
\hangcaption
[Convergence to isotropic response for the plane stress, VIB model]
{Convergence to isotropic response in
   the numerical implementation of the plane stress, 
   isotropic VIB model.\label{fig.VIB2Derrorplot}}
\end{figure}
The uniform bond distribution is sampled at fixed intervals $\Delta \phi$.
The resulting values of the stress and modulus should be independent of the
orientation of the ``microstructure'', represented by the offset angle
$\hat{\phi}_0$ in positioning the integration points. Clearly, the value of the
VIB integrals for the stress and modulus will not be independent of $\hat{\phi}_0$
if an insufficient number of sampling points is used. The variation of the
$\mathsf{C}_{1111}$ component of the spatial tangent modulus for an
arbitrarily chosen state of deformation is shown in 
Figure~\ref{fig.VIB2Derrorplot} for the case of $N=3$ sampling points.
As is evident from the figure, the value displays a six-fold symmetric
variation. The $N=3$ configuration corresponds to the bonding
arrangement in a two-dimensional, close-packed lattice. This is the lattice
arrangement on the close-packed planes of the face-centered cubic (FCC) lattice
and on the basal plane of the hexagonal close-packed (HCP) lattice. For
calculations at infinitesimal deformations, these planes are commonly
assumed to be isotropic. However, only the instantaneous response of the
undeformed lattice lacks orientational dependence. For deformations at
finite strains, this crystal plane displays six-fold symmetric anisotropy.
Figure~\ref{fig.VIB2Derrorplot} also shows how the numerical response
converges toward isotropy as the number of integration points is increased.
The magnitude of the  variation in the modulus $\Delta \mathsf{C}_{1111}$
relative to the ``exact'' value for the applied deformation 
$\mathsf{C}_{1111}^{\infty}$ drops to below machine precision with only
11 integration points. Therefore, the computational effort to produce
essentially isotropic response in the numerical implementation
does not present significant difficulties. The evenly-spaced
integration point arrangement seems to be a highly efficient scheme for evaluating
the VIB integrals. For a given number of points, the results using integration
schemes based on evenly-spaced points display much
smaller dependencies on orientation than do the results using 
integration schemes based on Gaussian quadrature.

For the three-dimensional isotropic VIB model, the task of generating
isotropic response is significantly more difficult. This case
requires evaluating the VIB integrals over a sphere.
Unlike the two-dimensional case, there are no schemes for arranging
an arbitrary number of points on a sphere in a evenly-spaced manner.
The icosahedron with its 20 equilateral triangular faces presents a
starting point from which more dense arrangements of points can be
constructed. Integration rules with 80 and 320 points in arrangements
which are symmetric about the ``equator'' are shown in
Figure~\ref{fig.icosahedralpoints}.
\begin{figure}[h]
\centerline{\includegraphics[scale = 1.0]
{\dirfilepath{\figpath}{icosahedralpoints.pdf}}}
\hangcaption
[Integration point arrangements in three dimensions]
{Integration point arrangements in
   three dimensions based on the icosahedron.\label{fig.icosahedralpoints}
   \todo{get picture of lat-long points.}}
\end{figure}
These arrangements are created by successively bisecting the edges of the
triangular facets and projecting the vertex points onto the sphere. 
The center of each facet is used as an integration point
with a weight selected to produce the correct result when integrating a 
constant over the surface of the sphere. 
For a rotation about an arbitrarily selected axis, the
variation in the
$\mathsf{C}_{1111}$ component of the spatial tangent modulus is shown in
Figure~\ref{fig.VIBerrorplot3D}
\begin{figure}[h]
\centerline{\includegraphics[scale = 1.0]
{\dirfilepath{\figpath}{VIBerrorplot3D.pdf}}}
\hangcaption
[Orientational dependence of $\mathsf{C}_{1111}$]
{Orientational dependence of 
   $\mathsf{C}_{1111}$ for the three-dimensional isotropic VIB model
   in a deformed state.\label{fig.VIBerrorplot3D}}
\end{figure}
for the integration schemes depicted in Figure~\ref{fig.icosahedralpoints}.
The figure shows that the modulus displays noticeable anisotropic behavior
although a much larger number of points is used than would be required in
two dimensions to produce essentially no error. 
Ba$\check{\textrm{z}}$ant\cite{Bazant1986} has developed more
sophisticated schemes for numerical integration over the surface
over of a sphere, but even these schemes produce a significant
increase in computational effort with the extension from two to
three dimensions.

\subsubsection{3D input parameters}
An example of input parameters for three-dimensional analysis is shown 
below:
\begin{inputfile}
####### VIB
16    # material code
####### common material parameters
0.0    0.0    1.0
0      0.0
####### potential parameters
1     # potential code          
4.775 # A 
0.0643# B
####### integration parameters
1     # integration rule      
40    # number of integration points
\end{inputfile}
	
\subsubsection{2D input parameters}
An example of input parameters for two-dimensional analysis is shown 
below:
\begin{inputfile}
####### VIB
16    # material code
####### common material parameters
0.0    0.0    1.0
0      0.0
####### 2D parameters
1.0   # thickness
1     # constraint option (override)
####### potential parameters
1     # potential code          
4.775 # A 
0.0643# B
####### integration parameters
0     # integration rule      
11    # number of integration points
\end{inputfile}

\subsubsection{Output values}
This constitutive model does not have any output values.

\subsection{Material 17: isotropic Virtual Internal Bond model (Simo)}
\label{sect.mat.VIB.Simo}
This model combines the VIB strain energy density from
Section~\ref{sect.mat.VIB} with the
spectral formulation developed by Simo and Taylor~\cite{Simo1991}.
Expressed in principal stretches, the model displays
exactly isotropic response regardless of the accuracy of the
scheme used to integrate the bond distribution.
\todo{However, the model will not display the modulus, fracture
energy, or cohesive strength developed from analysis of the
model.}

\subsubsection{3D input parameters}
An example of input parameters for three-dimensional analysis is shown 
below:

\subsubsection{2D input parameters}
An example of input parameters for two-dimensional analysis is shown 
below:

\subsubsection{Output values}
This constitutive model does not have any output values.

\subsection{Material 18: isotropic Virtual Internal Bond model (Ogden)}
\label{sect.mat.VIB.Ogden}
This model combines the VIB strain energy density from
Section~\ref{sect.mat.VIB} with the
spectral formulation developed by Ogden~\cite{Ogden1983}.
Expressed in principal stretches, the model displays
exactly isotropic response regardless of the accuracy of the
scheme used to integrate the bond distribution.
\todo{However, the model will not display the modulus, fracture
energy, or cohesive strength developed from analysis of the
model.}

\subsubsection{3D input parameters}
An example of input parameters for three-dimensional analysis is shown 
below:

\subsubsection{2D input parameters}
An example of input parameters for two-dimensional analysis is shown 
below:

\subsubsection{Output values}
This constitutive model does not have any output values.

\subsection{Material 19: isotropic Virtual Internal Bond model
with $J_{2}$ plasticity}
\label{sect.mat.VIB.Simo.J2}
This model combines the VIB strain energy density from
Section~\ref{sect.mat.VIB} with the
spectral formulation developed by Simo and Taylor~\cite{Simo1991}
and the finite strain framework of $J_{2}$ plasticity
from Simo~\cite{Simo1992} described in 
Section~\ref{sect.mat.J2.QL.LinearHardening}.
Expressed in principal stretches, the model displays
exactly isotropic response regardless of the accuracy of the
scheme used to integrate the bond distribution.
\todo{However, the model will not display the modulus, fracture
energy, or cohesive strength developed from analysis of the
model.}

The yield condition is formulated in terms of the
Kirchhoff stress $\boldsymbol{\tau}$ as
\begin{equation}
	f\farg{\boldsymbol{\tau}, \mathbf{q}} := 
	||\boldsymbol{\eta}|| 
	- \sqrt{\frac{2}{3}} K\farg{\alpha} \le 0,
\end{equation}
where the internal variables 
$\mathbf{q} = \cbrkt{\alpha, \bar{\boldsymbol{\beta}}}$ are
the equivalent plastic strain $\alpha$ and the center of
the yield surface $\Bar{\boldsymbol{\beta}}$ in deviatoric
Kirchhoff stress space. The relative stress $\boldsymbol{\eta}$ is
\begin{equation}
\boldsymbol{\eta} = \textrm{dev}\sbrkt{\boldsymbol\tau} - 
\bar{\boldsymbol{\beta}}.
\end{equation}
As in the small strain case, the hardening variable is assumed
to evolve as
\begin{equation}
\dot{\alpha} = \sqrt{\frac{2}{3}}\,\gamma,	
\end{equation}
where $\gamma$ is the consistency parameter 
in~\eqref{eq.Kuhn-Tucker.1} and~\eqref{eq.Kuhn-Tucker.2}.
The flow rule rule and kinematic hardening law are derived
assuming a multiplicative split of the deformation gradient.
These will not be outlined here; however,
the linear kinematic and isotropic hardening functions
\begin{equation}
	H\farg{\alpha} = \rbrkt{1 - \theta} \bar{H} \alpha
\end{equation}
and
\begin{equation}
	K\farg{\alpha} = \tau_{y} + \theta \bar{H} \alpha
\end{equation}
functional analogously to the small strain model described
in Section~\ref{sect.mat.SS.J2.LinearHardening}.
$\tau_{y}$ is the initial Kirchhoff yield stress and
$\bar{H}$ is the hardening modulus.
The parameter $0 \le \theta \le 1$ determines the mixity of 
isotropic to kinematic hardening, with $\theta = 1$ indicating purely 
isotropic hardening.

\subsubsection{Output values}
\todo{output values}

% jump to section 80 for ABAQUS/Standard UMAT materials
\setcounter{subsection}{79}
% jump to section 80 for ABAQUS/Standard UMAT materials
\subsection{Material 80: \textsf{ABAQUS/Standard UMAT} BCJ}
\label{sect.mat.ABAQUS.BCJ}
This model makes of the the \textsf{ABAQUS/Standard}~\cite{ABAQUSv56}
\textsf{UMAT} interface in \tahoe. This model is due to Bammann, 
Chiesa, and Johnson~\cite{BammannNNNN}.
The \textsf{ABAQUS/Standard UMAT} interface in \tahoe parses
\textsf{ABAQUS}-format input. The specifications for
\textsf{UMAT}'s in \tahoe may be taken directly from an
\textsf{ABAQUS Standard} input deck.

Like all material models, the materials using the 
\textsf{ABAQUS/Standard UMAT} interface may be used for 
analysis types not supported by \textsf{ABAQUS/Standard}. Namely, the 
materials may be applied to dynamic analysis using explicit time 
integration. Using this time integration scheme, the \textsf{UMAT}'s
will calculate both the stress and modulus, unless specifically 
rewritten to detect whether both or just one is actually needed for 
the analysis. Unmodified \textsf{UMAT}'s will perform unnecessary 
calculations, leading to longer simulations times.
This model requires an installation of \tahoe that include the 
\textsf{f2c}~\cite{f2c} module. 

\subsubsection{3D input parameters}
An example of input parameters for three-dimensional analysis is shown 
below:
\begin{verbatim}
####### ABAQUS/Standard BCJ	
80    # material code
####### common material parameters
0.0    0.0    1.0
0      0.0
####### ABAQUS input deck
*****************************************
*MATERIAL, NAME=A356
**
** user subroutine is dmg1.f using metric units
** material name = alum  356                                               
** creation date = 08/08/96                                                    
** data fit by   =  Jim Lathrop for uscar A356 temp-strain rate
*USER MATERIAL, CONSTANTS=52
**
**G, a,Bulk,b,melttemp,C1, C2, C3
** C4, C5, C6, C7, C8, C9, C10, C11
**C12,C13,C14,C15,C16,C17, C18, C19
**C20,Ca,Cb,init.temp,heat gen.coeff,void growth exp,initial rad,tors constant a
**nuc const b,nuc const c,nuc coeff,fract.tough,part.size,part.vol.fract,cd1,cd2
**
 2.592E+04 1.000E+00 6.763E+04 0.000E+00 5.556E+03 5.309E+01 9.453E+02 1.559E+02
 1.105E+02 1.000E-05 0.0000000 1.128E-03 -1796.200 4.820E+03 1.094E+01 2.385E-03
 1.441E+03 1.674E-03 0.000E+00 2.818E+03 4.622E+00 0.000E+00 0.000E+00 0.000E+00
 0.000E+00 -5.000000 -0.389490 297.00000 0.0000000 0.3000000 0.0002000 615460.00
 58640.000 30011.000 86.600000 17.300000 4.000E-06 0.0700000 4.5000000 40.000000
 20.00E-00 20.00E-00 0.0509000 0.0010000 0.0000000 0.0000000 0.0000000 0.0000000
 0.0000000 0.0000000 0.0035000 0.0090000
*DEPVAR
 25
\end{verbatim}
	
\todo{This is for A356}

\subsubsection{2D input parameters}
The parameters for using the 
\todo{same as for three-dimensional analysis. Since this model is 
formulated in three-dimensions, the thing will be plane strain when 
used for two dimensional analysis. Mention this above in the general 
description of the \textsf{UMAT} interface.}

\subsubsection{Output values}
\todo{output values determined by the implementation of the 
\textsf{UMAT}. Show output parameters for A356.}


