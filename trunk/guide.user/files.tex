%&LaTeX
% $Id: files.tex,v 1.2 2001-03-19 16:46:24 paklein Exp $

\section{Geometry and results files}
\label{sect.files}
The user input consists of a analysis parameters
and a geometry specification. 
Analysis parameters are specified in a plain text file. Several 
options exist for the specification of the geometry.
Table~\ref{tab.input.file.type} lists the
file types available to specify the geometry.
These will be described in greater detail in 
Section~\ref{sect.file.geometry}.
Table~\ref{tab.output.file.type} lists the
file types available for the output of results.
Note that some of the file types may be used for both
input and results, while others are limited to one or the other.
\begin{table}[h]
\caption{\label{tab.input.file.type} Geometry input file formats.}
\begin{center}
\begin{tabular}[c]{|l|c|c|}
\hline
 \parbox[b]{0.75in}{\centering \textbf{code}}
&\parbox[b]{2.0in}{\raggedright \textbf{description}}\\
\hline
\parbox[b]{0.75in}{\centering 0} & 
\parbox[b]{2.0in}{\raggedright TahoeI}\\
\hline
\parbox[b]{0.75in}{\centering 1}  & 
\parbox[b]{2.0in}{\raggedright TahoeII}\\
\hline
\parbox[b]{0.75in}{\centering 5}  & 
\parbox[b]{2.0in}{\raggedright \textsf{ExodusII}~\cite{ExodusII}}\\
\hline
\end{tabular}
\end{center}
\end{table}
\begin{table}[h]
\caption{\label{tab.output.file.type} Result output file formats.}
\begin{center}
\begin{tabular}[c]{|l|c|c|}
\hline
 \parbox[c]{0.75in}{\centering \textbf{code}}
&\parbox[c]{2.0in}{\raggedright \textbf{description}}\\
\hline
\parbox[c]{0.75in}{\centering 0} & 
\parbox[c]{2.0in}{\raggedright TahoeI}\\
\hline
\parbox[c]{0.75in}{\centering 2}  & 
\parbox[c]{2.0in}{\raggedright \textsf{TecPlot}~\cite{TecPlot} version 7}\\
\hline
\parbox[c]{0.75in}{\centering 3}  & 
\parbox[c]{2.0in}{\raggedright \vspace{2pt} \textsf{EnSight}~\cite{EnSight} Gold ASCII 
version 6 \vspace{2pt}}\\
\hline
\parbox[c]{0.75in}{\centering 4}  & 
\parbox[c]{2.0in}{\raggedright \vspace{2pt} \textsf{EnSight} Gold 
Binary version 6 \vspace{2pt}}\\
\hline
\parbox[c]{0.75in}{\centering 5}  & 
\parbox[c]{2.0in}{\raggedright \textsf{ExodusII}~\cite{ExodusII}}\\
\hline
\end{tabular}
\end{center}
\end{table}

\subsection{Geometry file formats}
\label{sect.file.geometry}
Input data is allowed to be one of three database formats listed in
Table~\ref{tab.input.file.type}: 
\begin{itemize}
\item[(1)] TahoeI: With this format, geometry data may be supplied 
embedded in the parameters file or in files separate from, or external 
to, the parameters files. When specified in external files, the block 
of geometry data in the parameters file is replaced by the name of 
the external file where the data resides. The format of the data 
itself is the same in both cases.
\item[(2)] TahoeII: This format allows random access to the geometry 
data which is stored in one or more plain text files.
\item[(3)] \textsf{ExodusII}: 
This format allows random access to the geometry 
data which is stored in platform independent binary file.
\end{itemize}
All \tahoe plain text input files are white space delineated, that is, 
values are seperated by white space, but the arrangement of white space
is ignored.
The \# symbol denotes that the remainder of a given line of text 
should be ignored, or treated as a comment.

\subsubsection{TahoeI format}
\label{sect.file.tahoeI}

\subsubsection{TahoeII format}
\label{sect.file.tahoeII}

\subsubsection{\textsf{ExodusII/Genesis} format}
\label{sect.file.exodusII}
The \textsf{ExodusII}~\cite{ExodusII} finite element geometry and results file 
format was developed at Sandia National Laboratories. This file format 
is export-controlled, and therefore is not available to all users.
A given database should contain a coordinate list, element blocks,
and any node set or sides sets needed in the input file (i.e. a Genesis
file) [21].  Material data should not be included and will be ignored. 
Node and element maps are currently ignored and items are assumed to be
consecutively numbered.  To refer to a specific set or block in the
database use the ID value.  The blocks and sets do not need to have
consecutive ID values.
